\documentclass[UKenglish]{lipics-v2016}

\usepackage{amsmath,amssymb,amsthm}
\usepackage{latexsym}
\usepackage{microtype}
\usepackage{proof}
\usepackage{stmaryrd}
\usepackage{xargs}

\usepackage{graphicx}

\usepackage{willemtools}
\usepackage{proofnet}

\bibliographystyle{plainurl}


% ============================== MACROS

\makeatletter

% ===== Definitions

\newcommand\defn[1]{\textit{\textbf{#1}}}


% ===== General maths

\newcommand\floor[1]{\lfloor#1\rfloor}

% ===== Sets

\newcommand\varA{\textsc{var}^\forall}
\newcommand\varE{\textsc{var}^\exists}
\newcommand\terms{\textsc{term}}
\newcommand\termsA{\textsc{term}^\forall}
\newcommand\atom{\textsc{atom}}
\newcommand\all{\textsc{all}}

\newcommand\subs[1]{\textsc{sub}(#1)}
\newcommand\dom[1]{\textsc{dom}(#1)}

% ===== Formulas

\newcommand\+{+}
\renewcommand\*{\times}
\newcommand\dual[1]{\overline{#1}}

\newcommand\sub{\leq}
\newcommand\seq[3][]{{\vdash_{#1}}#2,#3}
\newcommand\fv{\textsc{fv}}

% ===== Proofs

\newcommand\prf[3]{#1\vdash\!#2,#3}

% ===== Nets

\newcommand\net[3]{#1\triangleright #2,#3}

\newcommand\deseq[4][\sigma]{[#2]^{#1}_{#3,#4}}
\newcommand\Deseq[4][\sigma]{\left[\vcenter{#2}\right]^{#1}_{#3,#4}}

%\newcommandx\deseq[4][2=\sigma]{\floor{#1}_{#2}^{#3,#4}}
%\newcommandx\Deseq[4][2=\sigma]{\left\lfloor\vcenter{#1}\right\rfloor_{#2}^{#3,#4}}

% ===== Witness maps

\newcommand\mgu{\textsc{mgu}}

\newcommand\ult{\subseteq}
\newcommand\uco{\smallfrown}
\newcommand\ujn{\cup}

\newcommand\vlt{\le}
\newcommand\vco{\mathrel{\makebox[5pt][c]{\raisebox{-1ex}{$\widehat{}$}}}}
\newcommand\vjn{\vee}

\newcommand\join\bowtie

\newcommand\comp\frown
%\newcommand\ubot\sqcup

%\newlength{\cuplength}
%\settowidth{\cuplength}{$\cup$}
%\newcommand\ubot{\mathbin{\ooalign{$\cup$\cr\hss\scalebox{.9}{\raisebox{0.6ex}{\tiny$\mathbf\bot$}}\hss}}}


%\newcommand\inst[1]{\llparenthesis#1\rrparenthesis}

% ===== Slices


% ===== Coalescence

\newcommand\clink[3][\sigma]{(#2,#3)_{#1}}
\newcommand\minus{\mathop{/\mathchoice{\kern-3pt}{\kern-3pt}{\kern-2.5pt}{\kern-2pt}/}}


\newcommand\ucoal{\leadsto}
\newcommand\vcoal{\rightsquigarrow}

\newcommand\urr[1]{
  \ifx#1+\expandafter\@urr\else
  \ifx#1*(\*)\else
  \ifx#1!(\forall)\else
  \ifx#1?(\exists)\else
  (#1)
  \fi\fi\fi\fi
}
\newcommand\@urr[1]{(+_{#1})}
\newcommand\vrr[1]{
  \ifx#1+\expandafter\@vrr\else
  \ifx#1*\langle\*\rangle\else
  \ifx#1!\langle\forall\rangle\else
  \ifx#1?\langle\exists\rangle\else
  \langle#1\rangle
  \fi\fi\fi\fi
}
\newcommand\@vrr[1]{\langle+_{#1}\rangle}



\newcommand\coals{\rightsquigarrow}
\newcommand\coal{\coals}%{\stackrel1\rightsquigarrow}

\newcommandx\crule[4][1=\sigma,2=\lambda]{#2\coals(#3,#4)_{#1}}

%\newcommandx\xlink[4][1=\sigma,2=\lambda]{#2\triangleright(#3,#4)_{#1}}


% ===== Derivations

\newcommand\udn{\mathrel{\rotatebox[origin=c]{-90}{$\ucoal$}\kern1pt}}
\newcommand\vdn{\mathrel{\rotatebox[origin=c]{-90}{$\vcoal$}}}

\newcommandx\ujg[4][2=\sigma]{#1\udn\clink[#2]{#3}{#4}}	%{#2\udn_{#1}{#3}\mathbin,{#4}}
\newcommandx\vjg[4][2=\sigma]{#1\vdn\clink[#2]{#3}{#4}}	%{#2\vdn_{#1}{#3}\mathbin,{#4}}

\makeatother

% ============================== TITLE & AUTHORS

\title{Proof nets for first-order additive linear logic}
%\titlerunning{Proof nets for ALL1}


\author[1]{Willem B.\ Heijltjes}
\author[2]{Dominic J.D.\ Hughes}
\author[3]{Lutz Stra\ss burger}
\affil[1]{University of Bath, United Kingdom\\
  \texttt{w.b.heijltjes@bath.ac.uk}}
\affil[2]{
  \texttt{}}
\affil[3]{INRIA \&\ \'Ecole Polytechnique, Palaiseau, France\\
  \texttt{lutz.strassburger@inria.fr}}
\authorrunning{W.B.\ Heijltjes, D.J.D.\ Hughes, and L.\ Stra\ss burger}

\Copyright{Willem B.\ Heijltjes, Dominic J.D.\ Hughes, and Lutz Stra\ss burger}

\subjclass{Dummy classification -- please refer to \url{http://www.acm.org/about/class/ccs98-html}}% mandatory: Please choose ACM 1998 classifications from http://www.acm.org/about/class/ccs98-html . E.g., cite as "F.1.1 Models of Computation". 
\keywords{Linear logic, Proof nets}


% ============================== CONTENT

\begin{document}


% --------------------------------------------------

\subsection{Sequentialization}

\[
	\infer{\seq P{\dual P}}{}
\qquad
	\infer{\seq A{B_0\+B_1}}{\seq A{B_i}}
\qquad
	\infer{\seq A{B\*C}}{\seq AB && \seq AC}
\qquad
	\infer{\seq A{\exists x.B}}{\seq A{B[t/x]}}
\qquad
	\infer[\!\!\scriptstyle{a\,\notin\,\textsc{fv}(A)}]{\seq A{\forall a.B}}{\seq AB}
\]


\[
	\infer[\!\!\scriptstyle{Q\sigma=\dual R\sigma}]{\ujg{\clink QR}QR}{}
\]

\[
	\infer[\!\!\scriptstyle{\sigma\,=\,\sigma(Q,R)}]{\vjg{\clink[]QR}QR}{}
\]

\[
	\infer{\ujg\lambda A{B_0\+B_1}}{\ujg\lambda A{B_i}}
\]

\[
	\infer{\vjg\lambda A{B_0\+B_1}}{\vjg\lambda A{B_i}}
\]

\[
	\infer[\!\!\scriptstyle{\sigma\,\uco\,\tau}\vphantom)]{\ujg{\lambda\cup\kappa}[\sigma\ujn\tau]A{B\*C}}{\ujg\lambda[\sigma]AB && \ujg\kappa[\tau]AC}
\]

\[
	\infer[\!\!\scriptstyle{\sigma\,\vco\,\tau}\vphantom)]{\vjg{\lambda\cup\kappa}[\sigma\vjn\tau]A{B\*C}}{\vjg\lambda[\sigma]AB && \vjg\kappa[\tau]AC}
\]

\[
	\infer{\ujg\lambda[\sigma\!\minus\!x]A{\exists x.B}}{\ujg\lambda AB}
\]

\[
	\infer{\vjg\lambda[\sigma\!\minus\!x]A{\exists x.B}}{\vjg\lambda AB}
\]

\end{document}






\maketitle

\begin{abstract}

\end{abstract}

\subsection{Proof identity}

\[
	\infer={\exists x.p(x),\exists y.\dual p(y)}{\infer{p(t),\dual p(t)}{}}
	\quad
	\stackrel?\equiv
	\quad
	\infer={\exists x.P(x),\exists y.\dual p(y)}{\infer{p(s),\dual p(s)}{}}
\]

\[
	\infer{\exists x.p,\dual p}{\infer{p[t/x],\dual p}{}}
	\quad
	\stackrel?\equiv
	\quad	
	\infer{\exists x.p,\dual p}{\infer{p[s/x],\dual p}{}}
\]

\[
	\infer{\exists x.p\+q(x),\dual p}{\infer{p\+q(t)}{\infer{p,\dual p}{}}}
	\quad
	\stackrel?\equiv
	\quad	
	\infer{\exists x.p\+q(x),\dual p}{\infer{p\+q(s)}{\infer{p,\dual p}{}}}
\]

\[
\begin{tikzpicture}[net]
\formula[y=1]{\exists x.\forall a.~p(a,x)}
\formula[y=0]{\exists y.~{\dual p}(y,s)\*{\dual p}(y,t)}
\Vlink[red,label={$\scriptstyle\sigma~$},l]{[-2]8,5}
\Vlink[red,label={$~\scriptstyle\tau$}]{[2]8,12}
\end{tikzpicture}
\]

\[
\begin{tikzpicture}[net]
\formula[y=1]{\exists x.\forall a.~p(a,x)}
\formula[y=0]{\exists y.~{\dual p}(y,s)\*{\dual p}(y,t)}
\Vlink[red,label={$~\scriptstyle\upsilon$}]{8,11}
\end{tikzpicture}
\]

\[
\begin{tikzpicture}[net]
\formula[y=1]{\exists x.\forall a.~p(a,x)}
\formula[y=0]{\exists y.~{\dual p}(y,s)\*{\dual p}(y,t)}
\Vlink[red,label={$\quad\scriptstyle{\upsilon\!\minus\!y}$}]{8,1}
\end{tikzpicture}
\]

\[
\begin{tikzpicture}[net]
\formula[y=1]{\exists x.\forall a.~p(a,x)}
\formula[y=0]{\exists y.~{\dual p}(y,s)\*{\dual p}(y,t)}
\Vlink[red,label={$~\scriptstyle{\upsilon\!\minus\!y}$}]{4,1}
\end{tikzpicture}
\]

\[
\begin{tikzpicture}[net]
\formula[y=1]{\exists x.\forall a.~p(a,x)}
\formula[y=0]{\exists y.~{\dual p}(y,s)\*{\dual p}(y,t)}
\Vlink[red,label={$~\scriptstyle{\rho}$}]{1,1}
\end{tikzpicture}
\]

\[
	\sigma = [y\mapsto a, x\mapsto s]
	\qquad
	\tau = [y\mapsto a, x\mapsto t]
\]
\[
	\upsilon = \rho[y\mapsto a, x\mapsto u]
	\qquad
	\rho = \mgu(s,t)
	\qquad
	 u=s\rho=t\rho
\]


% ==================================================

\section{Proof nets for first-order additive linear logic}

% ..................................................

\begin{figure}
\[
	\infer{\seq P{\dual P}}{}
\qquad
	\infer{\seq A{B_0\+B_1}}{\seq A{B_i}}
\qquad
	\infer{\seq A{B\*C}}{\seq AB && \seq AC}
\qquad
	\infer{\seq A{\exists x.B}}{\seq A{B[t/x]}}
\qquad
	\infer[\!\!\scriptstyle{a\,\notin\,\textsc{fv}(A)}]{\seq A{\forall a.B}}{\seq AB}
\]
\caption{A sequent calculus for $\all1$}
\label{fig:sequent calculus}
\end{figure}

% ..................................................

% --------------------------------------------------

\subsection{First-order additive linear logic}

First-order terms and the formulas of first-order $\all$ are generated by the following grammars.
%
\setMidspace{5pt}
\[
\begin{array}{@{}l@{}l}
	t &\Coloneqq a \Mid x \Mid f(t_1,\dots,t_n)
\\[10pt]
	P &\Coloneqq p(t_1,\dots,t_n) \Mid \dual p(t_1,\dots,t_n)
\\[10pt]
	A &\Coloneqq P \Mid A\+A \Mid A\*A \Mid \exists x.A \Mid \forall a.A
\end{array}
\]
%
Negation $(\dual{\,\cdot\,})$ is applied to predicate symbols, $\dual p$, as a matter of convenience. The \defn{dual} $\dual A$ of an arbitrary formula $A$ is given by DeMorgan. We use the following notational conventions.
%
\[
\begin{tabular}{@{}lll@{}}
	$a,b,c$ & $\in \varA$ 		& universally quantified variables (and free variables)\\
	$x,y,z$ & $\in \varE$		& existentially quantified variables \\
	$f,g,h$ & $\in \Sigma_f$	& $n$-ary $(n\geq 0)$ function symbols from a fixed alphabet $\Sigma_f$\\
	$p,q,r$ & $\in \Sigma_p$	& $n$-ary $(n\geq 0)$ predicate symbols from a fixed alphabet $\Sigma_p$ \\
	$s,t,u$ & $\in \terms$ 		& first-order terms over $\varA$, $\varE$, $\Sigma_f$, and $\Sigma_p$ \\
	$P,Q,R$ & $\in \atom$		& atomic propositions \\
	$A,B,C$ & $\in \all1$		& formulas \\
\end{tabular}
\]
%
A \defn{sequent} $\seq AB$ is a pair of formulas $A$ and $B$. A sequent calculus for \all1 is given in Figure~\ref{fig:sequent calculus}, where each rule has a symmetric counterpart that applies to the first formula in the sequent. We write $\prf\Pi AB$ for a proof $\Pi$ with conclusion sequent $\seq AB$. Anticipating the requirements of proof nets, we will assume the following convention for \all1 proofs.

\begin{description}
	\item
[\defn{Barendregt's convention}] All quantifiers in a sequent $\seq AB$ have a distinct binding variable, and no bound variable shares a name with a free one.
%	\item
%% ALTERNATIVE: "STATIC WITNESSING"
%[\defn{Universal witnessing}] All existential witnesses in a proof are \defn{universal terms} $\termsA\subset\terms$, generated by the following grammar.
%\[
%	\termsA: \qquad t \Coloneqq a \Mid f(t_1,\dots,t_n)
%\]
\end{description}

% --------------------------------------------------

\subsection{Proof nets}

By a \defn{subformula} we will mean a subformula \defn{occurrence}. For instance, a formula $A\*A$ has two subformulas $A$, one on the left and one on the right. The \defn{subformulas} $\subs A$ of a formula are defined as follows; we write $B\sub A$ if $B$ is a subformula of $A$, i.e.\ if $B\in\subs A$.
\[
	\subs A = \{A\} \cup
	\left\{\begin{array}{ll}
		\subs B\uplus\subs C	& \text{if $A=B\+C$ or $A=B\*C$} \\[5pt]
		\subs B					& \text{if $A=\exists x.B$ or $A=\forall a.B$}
	\end{array}\right.
\]

A \defn{witness map} $\sigma\colon \varE \rightharpoonup \terms$ is a substitution map which assigns terms to existential variables, given as a partial function. The \defn{empty} witness map $\varnothing$ is the everywhere undefined partial function. The map $\sigma[x\mapsto t]$ substitutes $t$ for $x$ and is as $\sigma$ everywhere else.

A \defn{link} $\clink CD$ on a sequent $\seq AB$ is a pair of subformulas $C\leq A$ and $D\leq B$ along with a witness map $\sigma$. We assume the existential variables occurring in the range of $\sigma$ to be fresh, i.e.\ distinct from any existentially quantified variable in $\seq AB$. A \defn{linking} $\lambda$ on the sequent $\seq AB$ is a set of links on it.% $\seq AB$.

\begin{definition}
A \defn{pre-net} $\net\lambda AB$ is a sequent $\seq AB$ with a linking $\lambda$ on it.
\end{definition}

An \defn{axiom} link is a link $\clink PQ$, and a linking consisting of only axiom links an \defn{axiom linking}.
%
A link is \defn{clean} if its witness map is empty, $\clink[\varnothing]CD$; it may then be rendered simply $(C,D)$. A linking or pre-net is \defn{clean} if all its links are. We write $\floor\lambda$ respectively $\floor{\net\lambda AB}$ for the clean version of a linking or pre-net, which forgets the witness map for each link.
%
A link is \defn{strict} if its domain consists of exactly the existentially quantified variables that its formulas are in scope of. That is, a link $\clink CD$ in a pre-net $\net\lambda AB$ is strict if
\[
	\dom\sigma = \{\,x \mid C < \exists x.X \leq A\,\}
			\cup \{\,y \mid D < \exists y.Y \leq B\,\}~.
\]
A linking or pre-net is \defn{strict} if all its links are.

\begin{definition}
The \defn{de-sequentialization} $[\Pi]$ of a sequent proof $\prf\Pi AB$ is the pre-net $\net{\deseq[\varnothing]\Pi AB} AB$ where $\deseq-AB$ is defined as follows.
%
\[
\begin{array}{ccl}
	\Deseq{\infer{\seq P{\dual P}}{}}QR &=& \{\clink QR\}
\\ \\
	\Deseq{\infer {\seq A {B_0\+B_1}} {\deduce [\rule{0pt}{1pt}] {\seq A{B_i}} \Pi }}
	  {A'}{B_0'\+B'_1} 
	&=& \Deseq{\deduce[\rule{0pt}{1pt}]{\seq A{B_i}}\Pi}{A'}{B_i'}
\\ \\
	\Deseq{\infer{\seq A{B\*C}}{
	  \deduce[\rule{0pt}{1pt}]{\seq AB}\Pi 
	  && 
	  \deduce[\rule{0pt}{1pt}]{\seq AC}\Sigma}}{A'}{B'\*C'}
	&=& \Deseq{\deduce[\rule{0pt}{1pt}]{\seq AB}\Pi}{A'}{B'}
	  \cup
	  \Deseq{\deduce[\rule{0pt}{1pt}]{\seq AC}\Sigma}{A'}{C'}
\\ \\
	\Deseq{\infer{\seq A{\exists x.B}}{\deduce[\rule{0pt}{1pt}]{\seq A{B[t/x]}}\Pi}}
	  {A'}{\exists x.B'}
	&=&
	 \Deseq[{\sigma[x\mapsto t]}]{\deduce[\rule{0pt}{1pt}]{\seq A{B[t/x]}}\Pi}
	  {A'}{B'}
\\ \\
	\Deseq{\infer[\!\!\scriptstyle{a\,\notin\,\textsc{fv}(A)}]
	  {\seq A{\forall a.B}}{\deduce[\rule{0pt}{1pt}]{\seq AB}\Pi}}{A'}{\forall a.B'}
	&=&
	  \Deseq{\deduce[\rule{0pt}{1pt}]{\seq AB}\Pi}{A'}{B'}
\end{array}
\]
The \defn{clean} de-sequentialization $\floor\Pi$ of $\prf\Pi AB$ is $\floor{[\Pi]}$.
\end{definition}

Observe that the pre-net $[\Pi]$ has a strict axiom linking and $\floor\Pi$ a clean axiom linking.

\subsection{Witness maps}

We will write $A\sigma$ for the application of the witness map $\sigma\colon\varE\rightharpoonup\terms$ to the formula $A$ as a substitution map. The \defn{composition} of two maps is written $\sigma\tau$, where $A(\sigma\tau)=(A\sigma)\tau$. The map $\sigma\minus x$ is undefined on $x$ and otherwise as $\sigma$, and $\sigma|X$ is as $\sigma$ on the set of variables $X$ and undefined otherwise. We define the following further operations on witness maps.

\[
\begin{tabular}{@{}lp{.75\textwidth}@{}}
	$\sigma\ult\tau$
&
	A witness map is \defn{included} in another, $\sigma\ult\tau$,
	if for every $x\in\varE$, $\sigma(x)=t$ implies $\tau(x)=t$.
\\[5pt]	
	$\sigma\uco\tau$
&
	Two witness maps are \defn{coherent}, $\sigma\uco\tau$,
	if for every $x\in\varE$, if $\sigma(x)$ and $\tau(x)$
	are both defined then $\sigma(x)=\tau(x)$.
\\[5pt]
	$\sigma\ujn\tau$
&
	The \defn{union} of two witness maps $\sigma\ujn\tau$
	is defined if they are coherent, and is the map taking
	$x$ to $\sigma(x)$ or $\tau(x)$ if either is defined.
\\[5pt]
	$\sigma\vlt\tau$
&
	A witness map is \defn{more general} than another, 
	$\sigma\vlt\tau$, if there is a map $\rho$ such that 
	$\sigma\rho=\tau$.
\\[5pt]	
	$\sigma\vco\tau$
&
	Two witness maps are \defn{compatible}, $\sigma\vco\tau$,
	if there is a map $\rho$ such that $\sigma\rho=\tau\rho$.
\\[5pt]
	$\sigma\vjn\tau$
&
	The \defn{join} of two witness maps $\sigma\vjn\tau$ is defined
	if they are compatible, and is the least map $\rho$ such that
	$\sigma\vlt\rho$ and $\tau\vlt\rho$.
\end{tabular}%
\]

% --------------------------------------------------

\subsection{Coalescence}

We will define two \defn{coalescence} relations, one \defn{compounding} $(\ucoal)$ and one \defn{unifying} $(\vcoal)$. 

\begin{definition}
\defn{Compounding coalescence} ($\ucoal$) is the rewrite relation on pre-nets generated by the following rules, which apply symmetrically to the first formula in the sequent.
%
\renewcommand{\+}{{+}}
\renewcommand{\*}{{\times}}
\begin{align}
  	\net{\{\clink AB\}}A{B\+C} 
& ~\ucoal~
	\net{\{\clink A{B\+C}\}}A{B\+C}
\tag{$\+_1$}
\\[5pt]
        \net{\{\clink AC\}}A{B\+C}
& ~\ucoal~  
	\net{\{\clink A{B\+C}\}}A{B\+C}
\tag{$\+_2$}
\\[5pt]
	\net{\{\clink AB,\clink[\tau]AC\}}A{B\*C}
& ~\ucoal~ 
	\net{\{\clink[\sigma\ujn\tau]A{B\*C}\}}A{B\*C}\quad\text{if }\sigma\uco\tau
\tag{$\*$}
\\[5pt]	
	\net{\{\clink AB\}}A{\exists x.B}
& ~\ucoal~ 
	\net{\{\clink[\sigma\!\minus\!x]A{\exists x.B}\}}A{\exists x.B}
\tag{$\exists$}
\\[5pt]
	\net{\{\clink AB\}}A{\forall a.B}
& ~\ucoal~ 
	\net{\clink A{\forall a.B}}A{\forall a.B} \quad \text{if }a\,\notin\,\fv(A\sigma)
\tag{$\forall$}
\end{align}
%
\defn{Unifying coalescence} ($\vcoal$) is the rewrite relation generated by the following rules:
\[
	\vrr+1~,~\vrr+2~,~\vrr?~,~\vrr!
	\quad
	\text{are respectively as}
	\quad
	\urr+1~,~\urr+2~,~\urr?~,~\urr!
\]
plus the rule
\renewcommand{\+}{{+}}
\renewcommand{\*}{{\times}}
\begin{align}
	\net{\{\clink AB,\clink[\tau]AC\}}A{B\*C}
& ~\vcoal~ 
	\net{\{\clink[\sigma\vjn\tau]A{B\*C}\}}A{B\*C}\quad\text{if }\sigma\vco\tau~.
\tag*{$\vrr*$}
\end{align}
%
A pre-net $\net\lambda AB$ \defn{coalesces}, in either relation $(\ucoal)$ or $(\vcoal)$, if it coalesces to a single clean link $\net{(A,B)}AB$. It \defn{strongly coalesces} if every coalescence sequence terminates at $\net{(A,B)}AB$.
\end{definition}

To be explicit, to generate coalescence as a rewrite relation means the following: a pre-net coalesces to another, $\net\lambda AB\ucoal \net{\lambda'}AB$ if and only if there is a rewrite rule $\net\kappa CD\ucoal\net{\kappa'} CD$ such that $C\leq A$, $D\leq B$, $\kappa\subseteq\lambda$, $\kappa'\subseteq\lambda'$, and $\lambda-\kappa=\lambda'-\kappa'$; and likewise for $(\vcoal)$.

\begin{definition}
An \all1 \defn{witnessing proof net} or \defn{witness net} is a pre-net $\net\lambda AB$ with a strict axiom linking $\lambda$, that compounding-coalesces.
\end{definition}

Coalescence on a clean axiom linking is initiated by assigning each link $(P,Q)$ an \defn{initial witness map} $\sigma(P,Q)$ which, if it exists, is the most general witness map $\sigma$ such that $P\sigma=\dual Q\sigma$. In other words, $\sigma(P,Q)$ is the most general unifier ($\textsc{mgu}$) of $P$ and $\dual Q$. If it exists, we call the link $(P,Q)$ \defn{compatible}, and a linking is \defn{compatible} if all its links are. For a compatible axiom linking $\lambda$ we define the \defn{initial witness assignment} $\lambda_\star$ to be:
\[
	\lambda_\star = \{~\clink PQ~\mid~(P,Q)\in\lambda~,~\sigma=\sigma(P,Q)~\}~.
\]

\begin{definition}
An \all1 \defn{unification proof net} or \defn{unification net} is a pre-net $\net\lambda AB$ with a clean, compatible axiom linking $\lambda$, such that $\net{\lambda_\star}AB$ unifying-coalesces.
\end{definition}

% --------------------------------------------------

\subsection{Sequentialization}





\end{document}


% --------------------------------------------------


\subsection{Geometric correctness}

For a labelled linking $\lambda$ and an existential variable $x$, write $\lambda(x)\subseteq\terms$ for the \defn{witness set} of $x$, which collects the terms assigned to $x$ by the links in $\lambda$:
\[
	\lambda(x) = \{~t~\mid~\sigma(x) = t~,~\clink AB\in\lambda~\}~.
\]
%
Given a labelled linking $\lambda$, a \defn{slice} $S$ of a formula $A$ is a set of formulas given by: $S=\{A\}\cup S'$ where $S'$ is as follows; if $A$ is of the form
%
\begin{itemize}
	\item
\makebox[40pt][l]{$P$:}				$S'=\varnothing$.
	\item
\makebox[40pt][l]{$B\+C$:} 			$S'=S_B\cup S_C$ where $S_B$ is a slice of $B$ and $S_C$ a slice of $C$.
	\item
\makebox[40pt][l]{$B\*C$:}			$S'$ is a slice of $B$ or a slice of $C$.
	\item
\makebox[40pt][l]{$\exists x.B$:}	$S'$ is a slice of $B$ if the quantifier is vacuous, $x\notin\fv(B)$, and
	\\[\itemsep]
	\hspace*{40pt}					$S'=\uplus_{t\,\in\,\lambda(x)}S_t$ where each $S_t$ is a slice of $B[t/x]$ otherwise.
	\item
\makebox[40pt][l]{$\forall a.B$:}	$S'$ is a slice of $B$.
\end{itemize}
%
A \defn{slice} of a labelled pre-net $\net\lambda AB$ is a set of links
\[
	\{~\clink CD\in \lambda~\mid~(C\sigma,D\sigma)\in S_A\times S_B~\}
\]
where $S_A$ is a slice of $A$ and $S_B$ a slice of $B$. A labelled pre-net is \defn{correct} if every slice is a singleton. A compatible pre-net $\net\lambda AB$ is \defn{correct} if its initial labelling $\net{\lambda_\star}AB$ is. Observe that for a \emph{labelled} pre-net correctness and compatibility are independent, while a non-labelled pre-net must be compatible to be correct, since $\lambda_\star$ is otherwise not defined.

\begin{definition}
An \all1 \defn{proof net} is a correct pre-net. An \all1 \defn{labelled proof net} is a correct and compatible labelled pre-net.
\end{definition}

\begin{lemma}
If a labelled pre-net coalesces, it is correct. 
\end{lemma}

\begin{proof}
By induction on the coalescence path we will show that if $\net\lambda AB$ coalesces to $\net{\{\clink AB\}}AB$ then $\net{\{\clink[\varnothing]{A\sigma}{B\sigma}\}}{A\sigma}{B\sigma}$ is correct. The statement is then the special case $\sigma=\varnothing$.

The result $\net{\{\clink[\varnothing]{A\sigma}{B\sigma}\}}{A\sigma}{B\sigma}$ is correct: its only slice is $\{\clink[\varnothing]{A\sigma}{B\sigma}\}$, since $A\sigma$ is in any slice $S_{A\sigma}$ of $A\sigma$, and $B\sigma$ is in any slice $S_{B\sigma}$ of $B\sigma$.

\renewcommand{\+}{{+}}
\renewcommand{\*}{{\times}}
\begin{align}
	\net{\{\clink AB,\clink[\tau]AC\}}A{B\*C}
& ~\coal~ 
	\net{\{\clink[\sigma\join\tau]A{B\*C}\}}A{B\*C}\quad\text{if }\sigma\comp\tau
\tag{$\*$}
\end{align}

Let $\rho=\sigma\join\tau$, and assume that $\net{\{\clink[\varnothing]{A\rho}{B\rho\*C\rho}\}}{A\rho}{B\rho\*C\rho}$ is correct. 

\end{proof}



Coalescence is effectively sequentialization. We will make this explicit by organising the rewrite rules as inference rules. 


%
%As an intermediary between the sequent calculus and labelled proof nets, we will introduce a \defn{labelled sequent calculus}. A \defn{labelled sequent}
%\[
%	\xlink AB
%\]
%is a sequent that carries a linking $\lambda$ and a witness map $\sigma$. 
%
%
%for the statement that $\lambda\seq{A\sigma}{B\sigma}$ coalesces to a single labelled 
%


\begin{figure}
\hrule
\par\bigskip
\[
\begin{array}{c@{\hspace{40pt}}ccc}
	\vc{\infer{\crule [\sigma(P,\dual Q)] [\{(P,Q)\}] P Q}{}}
&    
    \vc{\begin{tikzpicture}[net]
    	\formula[y=2]{P}
    	\formula[y=1]{Q}
    	\Vlink[blue]{1,1}
    \end{tikzpicture}}
&\rightsquigarrow&    
    \vc{\begin{tikzpicture}[net]
    	\formula[y=2]{P}
    	\formula[y=1]{Q}
    	\Vlink[red,label={$~\scriptstyle{\sigma(P,\dual Q)}$}]{1,1}
    \end{tikzpicture}}
\\ \\
	\vc{\infer{\crule A {B\+C}} {\crule AB}}
&   
    \vc{\begin{tikzpicture}[net]
    	\formula[y=2]{A}
    	\formula[y=1]{B\+C}
    	\Vlink[red,label={$\scriptstyle\sigma~$},l]{1,1}
    \end{tikzpicture}}
&\rightsquigarrow& 
    \vc{\begin{tikzpicture}[net]
    	\formula[y=2]{A}
    	\formula[y=1]{B\+C}
    	\Vlink[red,label={~$\scriptstyle\sigma$}]{1,2}
    \end{tikzpicture}}
\\ \\ 
    \vc{\infer[\!\!\scriptstyle{\sigma\comp\tau}]
    		{\crule [\sigma\cup\tau] [\lambda\cup\kappa] A {B\*C}}
    		{\crule AB && \crule[\tau][\kappa]AC}}
&     
    \vc{\begin{tikzpicture}[net]
    	\formula[y=2]{A}
    	\formula[y=1]{B\*C}
    	\Vlink[red,label={$\scriptstyle\sigma~$},l]{[-1]1,1}
    	\Vlink[red,label={~$\scriptstyle\tau$}]{[1]1,3}
    \end{tikzpicture}}
&\stackrel{\sigma\comp\tau}\rightsquigarrow&   
	\vc{\begin{tikzpicture}[net]
    	\formula[y=2]{A}
    	\formula[y=1]{B\*C}
    	\Vlink[red,label={~$\scriptstyle{\sigma\cup\tau}$}]{1,2}
    \end{tikzpicture}}
    
\\ \\
    \vc{\infer{\crule[\sigma\minus x]A{\exists x.B}}{\crule AB}}
&
    \vc{\begin{tikzpicture}[net]
    	\formula[y=2]{A}
    	\formula[y=1]{\exists x.B}
    	\Vlink[red,label={~$\scriptstyle\sigma$}]{1,4}
    \end{tikzpicture}}
&\rightsquigarrow&
    \vc{\begin{tikzpicture}[net]
    	\formula[y=2]{A}
    	\formula[y=1]{\exists x.B}
    	\Vlink[red,label={~$\scriptstyle{\sigma\!\minus\!x}$}]{1,1}
    \end{tikzpicture}}
\\ \\
    \vc{\infer[\!\!\scriptstyle{a\,\notin\,\fv(A\sigma)}]{\crule A{\forall a.B}}{\crule AB}}
&    
    \vc{\begin{tikzpicture}[net]
    	\formula[y=2]{A}
    	\formula[y=1]{\forall a.B}
    	\Vlink[red,label={~$\scriptstyle\sigma$}]{1,4}
    \end{tikzpicture}}
&\stackrel{a\,\notin\,\fv(A\sigma)}\rightsquigarrow&
    \vc{\begin{tikzpicture}[net]
    	\formula[y=2]{A}
    	\formula[y=1]{\forall a.B}
    	\Vlink[red,label={~$\scriptstyle\sigma$}]{1,1}
    \end{tikzpicture}}
\end{array}
\]
\par\bigskip\hrule
\caption{Coalescence derivation rules}
\label{fig:coalescence}
\end{figure}


\begin{lemma}
Given a proof $\prf\Pi AB$, formulas $A'$, $B'$, and a witness map $\sigma$ such that $A'\sigma=A$ and $B'\sigma=B$, the de-sequentialization $\net{\deseq\Pi{A'}{B'}}AB$ coalesces to $\net{\{\clink AB\}}AB$. 
\end{lemma}

\begin{lemma}
Given a proof $\prf\Pi AB$ and formulas $A'$ and $B'$, if a smallest witness map $\tau$ exists such that $A=A'\tau$ and $B=B'\tau$, then $\deseq\Pi{A'}{B'}$ coalesces to $\clink{A'}{B'}$ with $\sigma\leq\tau$.
\end{lemma}

\begin{proof}
By induction on $\Pi$.

\newlength\prooftablewidth
\setlength\prooftablewidth{-80pt}
\addtolength\prooftablewidth{\textwidth}

\noindent
\begin{tabular}{@{}p{80pt}@{}p{\prooftablewidth}@{}}
	\hfill \raisebox{-20pt}{\infer{\seq P{\dual P}}{\vphantom B}} \hfill\hfill
&
	Let $A'=Q$ and $B'=R$, and suppose $\tau$ is the smallest witness map such that $Q\tau=P$ and $R\tau=\dual P$. The desequentialization $\deseq\Pi QR$ is the $\net{\{Q,R\}}QR$, which coalesces to $\clink QR$ if $\sigma=\sigma(Q,{\dual R})$ exists. Since $\tau$ is a unifier of $Q$ and $\dual R$, $\sigma\leq\tau$.	
\\ \\
%	$\infer{\seq A{B_0\+B_1}}{\seq A{B_i}}$
%&
%\\ \\
	\hfill \raisebox{-20pt}{\infer{\seq A{B\*C}}{\seq AB && \seq AC}}\hfill\hfill
&
	Let $\seq{\prf\Pi AB}{A'}{B'}$ coalesce to the two subproofs coalesce to $\clink{A'}{B'}$ and $\clink[\rho]{A'}{B'}$ 
%	\infer{\seq A{\exists x.B}}{\seq A{B[t/x]}}
%	\infer[\!\!\scriptstyle{a\,\notin\,\textsc{fv}(A)}]{\seq A{\forall a.B}}{\seq AB}
\end{tabular}

\end{proof}


\end{document}


%\[
%	\infer{\xlink [\sigma(P,Q)] [\{(P,Q)\}] P Q}{}
%\]
%\par
%\[
%	\infer{\xlink A {B_0\+B_1}} {\xlink A {B_i}}
%\]
%\par
%\[
%	\infer[\!\!\scriptstyle{\sigma\comp\tau}]
%		{\xlink [\sigma\cup\tau] [\lambda\cup\kappa] A {B\*C}}
%		{\xlink AB && \xlink[\tau][\kappa]AC}
%\]
%\par
%\[
%	\infer{\xlink[\sigma\minus x]A{\exists x.B}}{\xlink AB}
%\]
%\par
%\[
%	\infer[\!\!\scriptstyle{a\,\notin\,\fv(A\sigma)}]{\xlink A{\forall a.B}}{\xlink AB}
%\]








% --------------------------------------------------

%\subsection{Existential witnesses}

%
%
%For two first-order terms $s$ and $t$, and a set of variables $V$, let a $V$-\defn{unifier} be a unifying substitution map
%\[
%	\sigma\colon V\rightharpoonup \terms
%\]
%(that is, $s\sigma=t\sigma$). A $V$-unifier for atomic propositions $P$ and $Q$ is defined analogously. For example, 
%
%
%The \defn{most general $V$-unifier} $\sigma_V(P,Q)$ of $P$ and $Q$ is the most general unifying
%
%let the \defn{initial witness map} $\sigma(P,Q)$ be the most general unifier of $P$ and $Q$, if it exists, where only existential variables can be substituted for, and universal variables are considered constant. That is, the only unifier of $p(x,y)$ and $p(a,a)$ is $\{x\mapsto a,y\mapsto a\}$, while $p(x,x)$ and $p(a,b)$ have no unifier.
%
%Let $\sigma\join\tau$ denote the combination of two witness maps $\sigma$ and $\tau$ as unifiers, if it exists. If it does, we call $\sigma$ and $\tau$ \defn{compatible}, expressed by $\sigma\comp\tau$. The initial witness map $\sigma(A,B)$ of two formulae $A$ and $B$, if it exists, is then given by:
%
%\begin{align}
%	\left.\begin{array}{r}
%		\sigma_V(\,A\+B\,,\,C\+D\,) \\
%		\sigma_V(\,A\*B\,,\,C\*D\,)
%	\end{array}\right\}
%	& = \sigma_V(A,C)\join\sigma_V(B,D)
%\\	
%	\sigma_V(\,\forall a.A\,,\,\forall a.B\,)
%		 \kern12.5pt & = \sigma_V(A,B)
%\\	\sigma_V(\,\exists x.A\,,\,\exists x.B\,)
%		 \kern12.5pt & = \sigma_{V-\{x\}}(A,B)
%\end{align*}
%
%
%Two witness maps $\sigma$ and $\tau$ are \defn{compatible}, written $\sigma\comp\tau$, if they agree wherever they are defined:
%\[
%	\sigma\comp\tau \quad\iff\quad \forall x\in\varE.~\big(\sigma(x)\neq\bot~\wedge~\tau(x)\neq\bot\big)~\Rightarrow~\sigma(x)=\tau(x)~.
%\]
%Two compatible witness maps can be combined into the map $\sigma\cup\tau$ that sends $x$ to $\sigma(x)$ or $\tau(x)$, whichever is defined, and is undefined when both are. To combine witness maps that may not be compatible we will write $\sigma\ubot\tau$, which is $\sigma\cup\tau$ if they are, and undefined ($\bot$) otherwise; if either $\sigma$ or $\tau$ is already undefined, $\sigma\ubot\tau$ also reverts to $\bot$.
%\[
%	\sigma\ubot\tau\quad=\quad
%	\left\{\begin{array}{ll}
%		\sigma\cup\tau	& \text{if $\sigma\neq\bot$, $\tau\neq\bot$, and $\sigma\comp\tau$} \\
%		\bot			& \text{otherwise} \\
%	\end{array}\right.
%\] 
%Note that $\bot$ is not a witness map, and different from the empty substitution map, $\sigma(x)=\bot$ for all $x$. For two atomic formulas $P,Q$ (or terms $s,t$) the \defn{initial witness map} $\sigma(P,Q)$ is the smallest witness map $\sigma$ such that $P\sigma=\dual Q\sigma$ and $P\sigma$ has no free existential variables. If it exists, it is defined inductively as follows (omitting the symmetric case for $\dual p$ and $p$). 
%
%\begin{align*}
%	\left.\begin{array}{r}
%		\sigma(\,x\,,\,t\,) \\
%	 	\sigma(\,t\,,\,x\,)
%	\end{array}\right\}
%	& = \{ x\mapsto t \}\quad \text{if $t\in\termsA$}
% \\
%	\left.\begin{array}{r}
%		\sigma(\,p(s_1,\dots,s_n)\,,\,\dual p(t_1,\dots,t_n)\,) \\
%		\sigma(\,f(s_1,\dots,s_n)\,,\,f(t_1,\dots,t_n)\,)
%	\end{array}\right\}
%	& = \sigma(s_1,t_1)\ubot\dots\ubot\sigma(s_n,t_n)
%\end{align*}
%
%\begin{lemma}
%\label{lem:minimal witness map}
%For first-order terms $s$ and $t$ the witness map $\sigma(s,t)$ is the smallest such that $s\sigma=t\sigma$ and $s\sigma$ contains no existential variables.
%\end{lemma}
%
%\begin{proof}
%By induction on $s$ and $t$. If $s=x$ then $\sigma(x,t)$ exists if and only if $t\in\termsA$, 
%\end{proof}

%\begin{align*}
%	\sigma_V(\,a\,,\,a\,) \kern12.5pt & = \varnothing
%\\	\sigma_V(\,x\,,\,x\,) \kern12.5pt & = \varnothing\quad \text{if $x\in V$}
%\\
%	\left.\begin{array}{r}
%		\sigma_V(\,x\,,\,t\,) \\
%	 	\sigma_V(\,t\,,\,x\,)
%	\end{array}\right\}
%	& = \{ x\mapsto t \}\quad \text{if $x\notin V$ and $t\in\termsA$}
% \\
%	\left.\begin{array}{r}
%		\sigma_V(\,p(s_1,\dots,s_n)\,,\,p(t_1,\dots,t_n)\,) \\
%		\sigma_V(\,f(s_1,\dots,s_n)\,,\,f(t_1,\dots,t_n)\,)
%	\end{array}\right\}
%	& = \sigma_V(s_1,t_1)\ubot\dots\ubot\sigma_V(s_n,t_n)
%\\	
%	\left.\begin{array}{r}
%		\sigma_V(\,A\+B\,,\,C\+D\,) \\
%		\sigma_V(\,A\*B\,,\,C\*D\,)
%	\end{array}\right\}
%	& = \sigma_V(A,C)\ubot\sigma_V(B,D)
%\\	
%	\sigma_V(\,\forall a.A\,,\,\forall a.B\,)
%		 \kern12.5pt & = \sigma_V(A,B)
%\\	\sigma_V(\,\exists x.A\,,\,\exists x.B\,)
%		 \kern12.5pt & = \sigma_{V\cup\{x\}}(A,B)
%\end{align*}
%
%Observe that if the witness map $\sigma=\sigma(A,B)$ exists then $A\sigma=B\sigma$.

% --------------------------------------------------


\begin{lemma}
Given a proof $\prf\Pi AB$ and formulas $A'$ and $B'$, if a smallest witness map $\tau$ exists such that $A=A'\tau$ and $B=B'\tau$, then $\deseq\Pi{A'}{B'}$ coalesces to $\clink{A'}{B'}$ with $\sigma\leq\tau$.
\end{lemma}

\begin{proof}
By induction on $\Pi$.

\newlength\prooftablewidth
\setlength\prooftablewidth{-80pt}
\addtolength\prooftablewidth{\textwidth}

\noindent
\begin{tabular}{@{}p{80pt}@{}p{\prooftablewidth}@{}}
	\hfill \raisebox{-20pt}{\infer{\seq P{\dual P}}{\vphantom B}} \hfill\hfill
&
	Let $A'=Q$ and $B'=R$, and suppose $\tau$ is the smallest witness map such that $Q\tau=P$ and $R\tau=\dual P$. The desequentialization $\deseq\Pi QR$ is the $\net{\{Q,R\}}QR$, which coalesces to $\clink QR$ if $\sigma=\sigma(Q,{\dual R})$ exists. Since $\tau$ is a unifier of $Q$ and $\dual R$, $\sigma\leq\tau$.	
\\ \\
%	$\infer{\seq A{B_0\+B_1}}{\seq A{B_i}}$
%&
%\\ \\
	\hfill \raisebox{-20pt}{\infer{\seq A{B\*C}}{\seq AB && \seq AC}}\hfill\hfill
&
	Let $\seq{\prf\Pi AB}{A'}{B'}$ coalesce to the two subproofs coalesce to $\clink{A'}{B'}$ and $\clink[\rho]{A'}{B'}$ 
%	\infer{\seq A{\exists x.B}}{\seq A{B[t/x]}}
%	\infer[\!\!\scriptstyle{a\,\notin\,\textsc{fv}(A)}]{\seq A{\forall a.B}}{\seq AB}
\end{tabular}

\end{proof}




\begin{description}
	\item[$\sigma\ult\tau$] 
A witness map is \defn{included} in another, $\sigma\ult\tau$, if for every $x\in\varE$, $\sigma(x)=t$ implies $\tau(x)=t$.
	\item[$\sigma\uco\tau$]
Two witness maps are \defn{coherent}, $\sigma\uco\tau$, if for every $x\in\varE$, if $\sigma(x)$ and $\tau(x)$ are both defined then $\sigma(x)=\tau(x)$.
	\item[$\sigma\ujn\tau$]
The \defn{union} of two witness maps $\sigma\ujn\tau$ is defined if they are coherent, and is the map taking $x$ to $\sigma(x)$ or $\tau(x)$ if either is defined.
	\item[$\sigma\vlt\tau$]
A witness map is \defn{more general} than another, $\sigma\vlt\tau$, if there is a map $\rho$ such that $\sigma\rho=\tau$.
	\item[$\sigma\vco\tau$]
Two witness maps are \defn{compatible}, $\sigma\vco\tau$, if there is a map $\rho$ such that $\sigma\rho=\tau\rho$.
	\item[$\sigma\vjn\tau$]
The \defn{join} of two witness maps $\sigma\vjn\tau$ is defined if they are compatible, and is the least map $\rho$ such that $\sigma\vlt\rho$ and $\tau\vlt\rho$.
\end{description}