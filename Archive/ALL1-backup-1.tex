\documentclass{article}

\usepackage{amsmath,amssymb,amsthm}
\usepackage{proof}
\usepackage{stmaryrd}
\usepackage{xargs}

\usepackage{graphicx}

\usepackage{willemtools}
\usepackage{proofnet}

\theoremstyle{definition}
\newtheorem{definition}{Definition}
\theoremstyle{plain}
\newtheorem{lemma}[definition]{Lemma}


% ===== Definitions

\newcommand\defn[1]{\textit{\textbf{#1}}}


% ===== General maths

\newcommand\floor[1]{\lfloor#1\rfloor}

% ===== Sets

\newcommand\varA{\textsc{var}^\forall}
\newcommand\varE{\textsc{var}^\exists}
\newcommand\terms{\textsc{term}}
\newcommand\termsA{\textsc{term}^\forall}
\newcommand\atom{\textsc{atom}}
\newcommand\all{\textsc{all}}

\newcommand\subs[1]{\textsc{sub}(#1)}


% ===== Formulas

\newcommand\+{+}
\renewcommand\*{\times}
\newcommand\dual[1]{\overline{#1}}

\newcommand\sub{\leq}
\newcommand\seq[3][]{{\vdash_{#1}}#2,#3}
\newcommand\fv{\textsc{fv}}

% ===== Proofs

\newcommand\prf[3]{#1\vdash\!#2,#3}

% ===== Nets

\newcommand\net[3]{#1\triangleright #2,#3}

\newcommandx\deseq[3]{\floor{#1}_{#2,#3}}
\newcommandx\Deseq[3]{\left\lfloor\vcenter{#1}\right\rfloor_{#2,#3}}

%\newcommandx\deseq[4][2=\sigma]{\floor{#1}_{#2}^{#3,#4}}
%\newcommandx\Deseq[4][2=\sigma]{\left\lfloor\vcenter{#1}\right\rfloor_{#2}^{#3,#4}}

% ===== Witness maps


\newcommand\comp\frown
\newcommand\ubot\sqcup

%\newlength{\cuplength}
%\settowidth{\cuplength}{$\cup$}
%\newcommand\ubot{\mathbin{\ooalign{$\cup$\cr\hss\scalebox{.9}{\raisebox{0.6ex}{\tiny$\mathbf\bot$}}\hss}}}



%\newcommand\inst[1]{\llparenthesis#1\rrparenthesis}

% ===== Slices


% ===== Coalescence

\newcommand\clink[3][\sigma]{(#2,#3)_{#1}}
\newcommand\minus{\mathop{/\mathchoice{\kern-3pt}{\kern-3pt}{\kern-2.5pt}{\kern-2pt}/}}

\newcommand\coals{\rightsquigarrow}
\newcommand\coal{\stackrel1\rightsquigarrow}

\newcommandx\crule[4][1=\sigma,2=\lambda]{#2\coals(#3,#4)_{#1}}

%\newcommandx\xlink[4][1=\sigma,2=\lambda]{#2\triangleright(#3,#4)_{#1}}




\begin{document}



    
    
 

%\end{document}


% ==================================================

\section{Proof nets for first-order additive linear logic}

% ..................................................

\begin{figure}
\hrule
\par\bigskip
\[
	\infer{\seq P{\dual P}}{}
\qquad
	\infer{\seq A{B_0\+B_1}}{\seq A{B_i}}
\qquad
	\infer{\seq A{B\*C}}{\seq AB && \seq AC}
\]
\par
\[
	\infer{\seq A{\exists x.B}}{\seq A{B[t/x]}}
\qquad
	\infer[\!\!\scriptstyle{a\,\notin\,\textsc{fv}(A)}]{\seq A{\forall a.B}}{\seq AB}
\]
\par
\[
	\infer{\seq A{\dual A}}{}
\qquad
	\infer{\seq AC}{\seq AB && \seq{\dual B}C}
\]
\par\bigskip\hrule
\caption{A sequent calculus for $\all1$}
\label{fig:sequent calculus}
\end{figure}

% ..................................................

% --------------------------------------------------

\subsection{First-order additive linear logic}

First-order terms and the formulas of first-order $\all$ are generated by the following grammars.
%
\setMidspace{5pt}
\[
\begin{array}{l@{}l}
	t &\Coloneqq a \Mid x \Mid f(t_1,\dots,t_n)
\\[10pt]
	P &\Coloneqq p(t_1,\dots,t_n) \Mid \dual p(t_1,\dots,t_n)
\\[10pt]
	A &\Coloneqq P \Mid A\+A \Mid A\*A \Mid \exists x.A \Mid \forall a.A
\end{array}
\]
%
Negation $(\dual{\,\cdot\,})$ is applied to predicate symbols, $\dual p$, as a matter of convenience. The \defn{dual} $\dual A$ of an arbitrary formula $A$ is given by DeMorgan. We use the following notational conventions.
%
\\[2\itemsep]
%
\begin{tabular}{@{}lll@{}}
	$a,b,c$ & $\in \varA$ 		& universally quantified variables (and free variables)\\
	$x,y,z$ & $\in \varE$		& existentially quantified variables \\
	$f,g,h$ & $\in \Sigma_f$	& $n$-ary $(n\geq 0)$ function symbols from a fixed alphabet $\Sigma_f$\\
	$p,q,r$ & $\in \Sigma_p$	& $n$-ary $(n\geq 0)$ predicate symbols from a fixed alphabet $\Sigma_p$ \\
	$s,t,u$ & $\in \terms$ 		& first-order terms over $\varA$, $\varE$, $\Sigma_f$, and $\Sigma_p$ \\
	$P,Q,R$ & $\in \atom$		& atomic propositions \\
	$A,B,C$ & $\in \all1$		& formulas \\
\end{tabular}
%
\\[2\itemsep]
%
A \defn{sequent} $\seq AB$ is an ordered pair of formulas $A$ and $B$. A sequent calculus for \all1 is given in Figure~\ref{fig:sequent calculus}, where each rule has a symmetric counterpart that applies to the first formula in the sequent. We write $\prf\Pi AB$ for a proof $\Pi$ with conclusion sequent $\seq AB$. Anticipating the requirements of proof nets, we will assume the following conventions for \all1 proofs.

\begin{description}
	\item
[\defn{Barendregt's convention}] All quantifiers in a sequent $\seq AB$ have a distinct binding variable, and no bound variable shares a name with a free one.
	\item
% ALTERNATIVE: "STATIC WITNESSING"
[\defn{Universal witnessing}] All existential witnesses in a proof are \defn{universal terms} $\termsA\subset\terms$, generated by the following grammar.
\[
	\termsA: \qquad t \Coloneqq a \Mid f(t_1,\dots,t_n)
\]
\end{description}

% --------------------------------------------------

\subsection{Proof nets}

By a \defn{subformula} we will mean a subformula \defn{occurrence}. For instance, a formula $A\*A$ has two subformulas $A$, one on the left and one on the right. The \defn{subformulas} $\subs A$ of a formula are given by
\[
	\subs A = \{A\} \cup
	\left\{\begin{array}{ll}
		\subs B\uplus\subs C	& \text{if $A=B\+C$ or $A=B\*C$} \\[5pt]
		\subs B					& \text{if $A=\exists x.B$ or $A=\forall a.B$}
	\end{array}\right.
\]
We will write $B\sub A$ if $B$ is a subformula of $A$. In the context of a sequent $\seq AB$, a \defn{link} $(A',B')$ is a pair of subformulas $A'\leq A$ and $B'\leq B$. A link is an \defn{axiom link} if both formulas are atomic propositions. A \defn{linking} $\lambda$ on the sequent $\seq AB$ is a set of links on $\seq AB$. It is an \defn{axiom linking} if all links are axiom links.

\begin{definition}
A \defn{pre-net} $\net\lambda AB$ is a sequent $\seq AB$ with a linking $\lambda$ on it.
\end{definition}

The \defn{de-sequentialization} of a sequent proof $\prf\Pi AB$ is the labelled pre-net $\net\lambda AB$ where $\lambda=\deseq\Pi AB$ and $\deseq-AB$ is defined as follows.
%
\[
\begin{array}{ccl}
	\Deseq{\infer{\seq P{\dual P}}{}}QR &=& \{(Q,R)\}
\\ \\
	\Deseq{\infer {\seq A {B_0\+B_1}} {\deduce [\rule{0pt}{1pt}] {\seq A{B_i}} \Pi }}
	  {A'}{B_0'\+B'_1} 
	&=& \Deseq{\deduce[\rule{0pt}{1pt}]{\seq A{B_i}}\Pi}{A'}{B_i'}
\\ \\
	\Deseq{\infer{\seq A{B\*C}}{
	  \deduce[\rule{0pt}{1pt}]{\seq AB}\Pi 
	  && 
	  \deduce[\rule{0pt}{1pt}]{\seq AC}\Sigma}}{A'}{B'\*C'}
	&=& \Deseq{\deduce[\rule{0pt}{1pt}]{\seq AB}\Pi}{A'}{B'}
	  \cup
	  \Deseq{\deduce[\rule{0pt}{1pt}]{\seq AC}\Sigma}{A'}{C'}
\\ \\
	\Deseq{\infer{\seq A{\exists x.B}}{\deduce[\rule{0pt}{1pt}]{\seq A{B[t/x]}}\Pi}}
	  {A'}{\exists x.B'}
	&=&
	 \Deseq{\deduce[\rule{0pt}{1pt}]{\seq A{B[t/x]}}\Pi}
	  {A'}{B'}
\\ \\
	\Deseq{\infer[\!\!\scriptstyle{a\,\notin\,\textsc{fv}(A)}]
	  {\seq A{\forall a.B}}{\deduce[\rule{0pt}{1pt}]{\seq AB}\Pi}}{A'}{\forall a.B'}
	&=&
	  \Deseq{\deduce[\rule{0pt}{1pt}]{\seq AB}\Pi}{A'}{B'}
\end{array}
\]

% --------------------------------------------------

\subsection{Existential witnesses}

A \defn{witness map} $\sigma$ is a substitution map
\[
	\sigma\colon \varE \rightharpoonup \termsA
\]
which assigns universal terms to existential variables, given by a partial function. We write $\sigma(x)=\bot$ to indicate that $\sigma$ is undefined for $x$. The map $\sigma\minus x$ is undefined for $x$ and as $\sigma$ everywhere else. The application of a witness map $\sigma$ to a formula $A$ is written $A\sigma$. Two witness maps $\sigma$ and $\tau$ are \defn{compatible}, written $\sigma\comp\tau$, if they agree wherever they are defined:
\[
	\sigma\comp\tau \quad\iff\quad \forall x\in\varE.~\big(\sigma(x)\neq\bot~\wedge~\tau(x)\neq\bot\big)~\Rightarrow~\sigma(x)=\tau(x)~.
\]
Two compatible witness maps can be combined into the map $\sigma\cup\tau$ that sends $x$ to $\sigma(x)$ or $\tau(x)$, whichever is defined, and is undefined when both are. To combine witness maps that may not be compatible we will write $\sigma\ubot\tau$, which is $\sigma\cup\tau$ if they are, and undefined ($\bot$) otherwise; if either $\sigma$ or $\tau$ is already undefined, $\sigma\ubot\tau$ also reverts to $\bot$.
\[
	\sigma\ubot\tau\quad=\quad
	\left\{\begin{array}{ll}
		\sigma\cup\tau	& \text{if $\sigma\neq\bot$, $\tau\neq\bot$, and $\sigma\comp\tau$} \\
		\bot			& \text{otherwise} \\
	\end{array}\right.
\] 
Note that $\bot$ is not a witness map, and different from the empty substitution map, $\sigma(x)=\bot$ for all $x$. For two atomic formulas $P,Q$ (or terms $s,t$) the \defn{initial witness map} $\sigma(P,Q)$ is the smallest witness map $\sigma$ such that $P\sigma=\dual Q\sigma$ and $P\sigma$ has no free existential variables. If it exists, it is defined inductively as follows (omitting the symmetric case for $\dual p$ and $p$). 

\begin{align*}
	\left.\begin{array}{r}
		\sigma(\,x\,,\,t\,) \\
	 	\sigma(\,t\,,\,x\,)
	\end{array}\right\}
	& = \{ x\mapsto t \}\quad \text{if $t\in\termsA$}
 \\
	\left.\begin{array}{r}
		\sigma(\,p(s_1,\dots,s_n)\,,\,\dual p(t_1,\dots,t_n)\,) \\
		\sigma(\,f(s_1,\dots,s_n)\,,\,f(t_1,\dots,t_n)\,)
	\end{array}\right\}
	& = \sigma(s_1,t_1)\ubot\dots\ubot\sigma(s_n,t_n)
\end{align*}

\begin{lemma}
\label{lem:minimal witness map}
For first-order terms $s$ and $t$ the witness map $\sigma(s,t)$ is the smallest such that $s\sigma=t\sigma$ and $s\sigma$ contains no existential variables.
\end{lemma}

\begin{proof}
By induction on $s$ and $t$. If $s=x$ then $\sigma(x,t)$ exists if and only if $t\in\termsA$, 
\end{proof}

%\begin{align*}
%	\sigma_V(\,a\,,\,a\,) \kern12.5pt & = \varnothing
%\\	\sigma_V(\,x\,,\,x\,) \kern12.5pt & = \varnothing\quad \text{if $x\in V$}
%\\
%	\left.\begin{array}{r}
%		\sigma_V(\,x\,,\,t\,) \\
%	 	\sigma_V(\,t\,,\,x\,)
%	\end{array}\right\}
%	& = \{ x\mapsto t \}\quad \text{if $x\notin V$ and $t\in\termsA$}
% \\
%	\left.\begin{array}{r}
%		\sigma_V(\,p(s_1,\dots,s_n)\,,\,p(t_1,\dots,t_n)\,) \\
%		\sigma_V(\,f(s_1,\dots,s_n)\,,\,f(t_1,\dots,t_n)\,)
%	\end{array}\right\}
%	& = \sigma_V(s_1,t_1)\ubot\dots\ubot\sigma_V(s_n,t_n)
%\\	
%	\left.\begin{array}{r}
%		\sigma_V(\,A\+B\,,\,C\+D\,) \\
%		\sigma_V(\,A\*B\,,\,C\*D\,)
%	\end{array}\right\}
%	& = \sigma_V(A,C)\ubot\sigma_V(B,D)
%\\	
%	\sigma_V(\,\forall a.A\,,\,\forall a.B\,)
%		 \kern12.5pt & = \sigma_V(A,B)
%\\	\sigma_V(\,\exists x.A\,,\,\exists x.B\,)
%		 \kern12.5pt & = \sigma_{V\cup\{x\}}(A,B)
%\end{align*}
%
%Observe that if the witness map $\sigma=\sigma(A,B)$ exists then $A\sigma=B\sigma$.

% --------------------------------------------------

\subsection{Geometric correctness}

A \defn{labelled link} $\clink AB$ is a link carrying a witness map $\sigma$. A \defn{labelled linking} respectively \defn{labelled pre-net} is a linking or pre-net where every link is labelled.
%
For a labelled pre-net $\net\lambda AB$ and an existential variable $x\in\varE$, write $\lambda(x)\subseteq\termsA$ for the \defn{witness set} of $x$, which collects the terms assigned to $x$ by the links in $\lambda$:
\[
	\lambda(x) = \{~t~\mid~\sigma(x) = t~,~\clink AB\in\lambda~\}~.
\]
%
Given a labelled linking $\lambda$, a \defn{slice} $S$ of a formula $A$ is a set of formulas given by: $S=\{A\}\cup S'$ where $S'$ is as follows; if $A$ is of the form
%
\begin{itemize}
	\item
\makebox[40pt][l]{$P$:}				$S'=\varnothing$.
	\item
\makebox[40pt][l]{$B\+C$:} 			$S'=S_B\cup S_C$ where $S_B$ is a slice of $B$ and $S_C$ a slice of $C$.
	\item
\makebox[40pt][l]{$B\*C$:}			$S'$ is a slice of $B$ or a slice of $C$.
	\item
\makebox[40pt][l]{$\exists x.B$:}	$S'$ is a slice of $B$ if the quantifier is vacuous, $x\notin\fv(B)$, and
	\\[\itemsep]
	\hspace*{40pt}					$S'=\uplus_{t\,\in\,\lambda(x)}S_t$ where each $S_t$ is a slice of $B[t/x]$ otherwise.
	\item
\makebox[40pt][l]{$\forall a.B$:}	$S'$ is a slice of $B$.
\end{itemize}
%
A \defn{slice} of a labelled pre-net $\net\lambda AB$ is a set of links
\[
	\{~(C\sigma,D\sigma)~\mid~\clink CD\in \lambda~,~(C\sigma,D\sigma)\in S_A\times S_B~\}
\]
where $S_A$ is a slice of $A$ and $S_B$ a slice of $B$.

\begin{definition}
A labelled pre-net is \defn{correct} if every slice is a singleton.
\end{definition}

A pre-net is \defn{compatible} if every link is an axiom link $(P,Q)$ such that $\sigma(P,Q)$ exists. A compatible pre-net $\net\lambda AB$ is \defn{correct} if its \defn{labelling} $\net{\lambda_\star}AB$ is correct, where $\lambda_\star$ is defined as follows.
\[
	\lambda_\star=\{~\clink PQ~\mid~(P,Q)\in\lambda~,~\sigma=\sigma(P,Q)~\}
\]

\begin{definition}
An \all1 \defn{proof net} is a correct pre-net.
\end{definition}

% --------------------------------------------------

\subsection{Coalescence}


\begin{definition}
\defn{Coalescence} ($\coals$) is the rewrite relation on labelled pre-nets generated by the following rewrite rules, and their symmetric variants.
%
%
\begin{itemize}
	\item 
\makebox[40pt][l]{$B\+C$:} 
Replace a link $\clink AB$ or $\clink AC$ with $\clink A{B\+C}$.
	\item 
\makebox[40pt][l]{$B\*C$:} 
Replace links $\clink AB$ and $\clink[\tau]AC$ with $\clink[\sigma\cup\tau]A{B\*C} $ if $\sigma\comp\tau$.
	\item
\makebox[40pt][l]{$\exists x.B$:} 
Replace a link $\clink AB$ with $\clink[\sigma\minus x]A{\exists x.B}$.
	\item
\makebox[40pt][l]{$\forall a.B$:}
Replace a link $\clink AB$ with $\clink A{\forall a.B}$ if $a\,\notin\,\fv(A\sigma)$.
\end{itemize}
%
%
%\begin{itemize}
%	\item 
%\makebox[40pt][l]{$B\+C$:} 
%Replace a link $\clink AB$ or $\clink AC$ with $\clink A{B\+C}$.
%	\item 
%\makebox[40pt][l]{$B\*C$:} 
%Replace links $\clink AB$ and $\clink[\tau]AC$ with $\clink[\sigma\cup\tau]A{B\*C} $ if $\sigma\comp\tau$.
%	\item
%\makebox[40pt][l]{$\exists x.B$:} 
%Replace a link $\clink AB$ with $\clink[\sigma\minus x]A{\exists x.B}$.
%	\item
%\makebox[40pt][l]{$\forall a.B$:}
%Replace a link $\clink AB$ with $\clink A{\forall a.B}$ if $a\,\notin\,\fv(A\sigma)$.
%\end{itemize}
%
A labelled pre-net $\net\lambda AB$ \defn{coalesces} if it coalesces to a single link $\clink[\varnothing]AB$. A compatible pre-net \defn{coalesces} if $\net{\lambda_\star}AB$ coalesces. A pre-net \defn{strongly coalesces} if every coalescence sequence terminates at $\clink[\varnothing]AB$.
\end{definition}

Coalescence is effectively sequentialization. We will make this explicit by organising the rewrite rules as inference rules. 


\begin{lemma}
Given a proof $\prf\Pi AB$ and formulas $A'$ and $B'$, if a smallest witness map $\sigma$ exists such that $A=A'\sigma$ and $B=B'\sigma$, then $\deseq\Pi{A'}{B'}$ coalesces to $\clink{A'}{B'}$.
\end{lemma}

\begin{proof}
By induction on $\Pi$.

\newlength\prooftablewidth
\setlength\prooftablewidth{-40pt}
\addtolength\prooftablewidth{\textwidth}

\noindent
\begin{tabular}{@{}p{40pt}@{}p{\prooftablewidth}@{}}
	$\infer{\seq P{\dual P}}{}$
&
	Let $A'=Q$ and $B'=R$. Then $\sigma$ is the smallest witness map such that $R\sigma=P$ and $Q\sigma=\dual P$. 
	
	

	
		\left.\begin{array}{r}
		\sigma(\,x\,,\,t\,) \\
	 	\sigma(\,t\,,\,x\,)
	\end{array}\right\}
	& = \{ x\mapsto t \}\quad \text{if $t\in\termsA$}
 \\
	\left.\begin{array}{r}
		\sigma(\,p(s_1,\dots,s_n)\,,\,\dual p(t_1,\dots,t_n)\,) \\
		\sigma(\,f(s_1,\dots,s_n)\,,\,f(t_1,\dots,t_n)\,)
	\end{array}\right\}
	& = \sigma(s_1,t_1)\ubot\dots\ubot\sigma(s_n,t_n)

	
\\ \\
	$\infer{\seq A{B_0\+B_1}}{\seq A{B_i}}$
&
\\ \\
	$\infer{\seq A{B\*C}}{\seq AB && \seq AC}$
%\]
%\[
%	\infer{\seq A{\exists x.B}}{\seq A{B[t/x]}}
%\]
%\[
%	\infer[\!\!\scriptstyle{a\,\notin\,\textsc{fv}(A)}]{\seq A{\forall a.B}}{\seq AB}
%\]
\end{tabular}

\end{proof}



%
%As an intermediary between the sequent calculus and labelled proof nets, we will introduce a \defn{labelled sequent calculus}. A \defn{labelled sequent}
%\[
%	\xlink AB
%\]
%is a sequent that carries a linking $\lambda$ and a witness map $\sigma$. 
%
%
%for the statement that $\lambda\seq{A\sigma}{B\sigma}$ coalesces to a single labelled 
%


\begin{figure}
\hrule
\par\bigskip
\[
\begin{array}{c@{\hspace{40pt}}ccc}
	\vc{\infer{\crule [\sigma(P,\dual Q)] [\{(P,Q)\}] P Q}{}}
&    
    \vc{\begin{tikzpicture}[net]
    	\formula[y=2]{P}
    	\formula[y=1]{Q}
    	\Vlink[blue]{1,1}
    \end{tikzpicture}}
&\rightsquigarrow&    
    \vc{\begin{tikzpicture}[net]
    	\formula[y=2]{P}
    	\formula[y=1]{Q}
    	\Vlink[red,label={$~\scriptstyle{\sigma(P,\dual Q)}$}]{1,1}
    \end{tikzpicture}}
\\ \\
	\vc{\infer{\crule A {B_0\+B_1}} {\crule A {B_i}}}
&   
    \vc{\begin{tikzpicture}[net]
    	\formula[y=2]{A}
    	\formula[y=1]{B\+C}
    	\Vlink[red,label={~$\scriptstyle\sigma$}]{1,2}
    \end{tikzpicture}}
&\rightsquigarrow& 
\\ \\ 
    \vc{\infer[\!\!\scriptstyle{\sigma\comp\tau}]
    		{\crule [\sigma\cup\tau] [\lambda\cup\kappa] A {B\*C}}
    		{\crule AB && \crule[\tau][\kappa]AC}}
&     
    \vc{\begin{tikzpicture}[net]
    	\formula[y=2]{A}
    	\formula[y=1]{B\*C}
    	\Vlink[red,label={$\scriptstyle\sigma~$},l]{[-1]1,1}
    	\Vlink[red,label={~$\scriptstyle\tau$}]{[1]1,3}
    \end{tikzpicture}}
&\stackrel{\sigma\comp\tau}\rightsquigarrow&   
	\vc{\begin{tikzpicture}[net]
    	\formula[y=2]{A}
    	\formula[y=1]{B\*C}
    	\Vlink[red,label={~$\scriptstyle{\sigma\cup\tau}$}]{1,2}
    \end{tikzpicture}}
    
\\ \\
    \vc{\infer{\crule[\sigma\minus x]A{\exists x.B}}{\crule AB}}
&
    \vc{\begin{tikzpicture}[net]
    	\formula[y=2]{A}
    	\formula[y=1]{\exists x.B}
    	\Vlink[red,label={~$\scriptstyle\sigma$}]{1,4}
    \end{tikzpicture}}
&\rightsquigarrow&
    \vc{\begin{tikzpicture}[net]
    	\formula[y=2]{A}
    	\formula[y=1]{\exists x.B}
    	\Vlink[red,label={~$\scriptstyle{\sigma\!\minus\!x}$}]{1,1}
    \end{tikzpicture}}
\\ \\
    \vc{\infer[\!\!\scriptstyle{a\,\notin\,\fv(A\sigma)}]{\crule A{\forall a.B}}{\crule AB}}
&    
    \vc{\begin{tikzpicture}[net]
    	\formula[y=2]{A}
    	\formula[y=1]{\forall a.B}
    	\Vlink[red,label={~$\scriptstyle\sigma$}]{1,4}
    \end{tikzpicture}}
&\stackrel{a\,\notin\,\fv(A\sigma)}\rightsquigarrow&
    \vc{\begin{tikzpicture}[net]
    	\formula[y=2]{A}
    	\formula[y=1]{\forall a.B}
    	\Vlink[red,label={~$\scriptstyle\sigma$}]{1,1}
    \end{tikzpicture}}
\end{array}
\]
\par\bigskip\hrule
\caption{Coalescence derivation rules}
\label{fig:coalescence}
\end{figure}



\end{document}


%\[
%	\infer{\xlink [\sigma(P,Q)] [\{(P,Q)\}] P Q}{}
%\]
%\par
%\[
%	\infer{\xlink A {B_0\+B_1}} {\xlink A {B_i}}
%\]
%\par
%\[
%	\infer[\!\!\scriptstyle{\sigma\comp\tau}]
%		{\xlink [\sigma\cup\tau] [\lambda\cup\kappa] A {B\*C}}
%		{\xlink AB && \xlink[\tau][\kappa]AC}
%\]
%\par
%\[
%	\infer{\xlink[\sigma\minus x]A{\exists x.B}}{\xlink AB}
%\]
%\par
%\[
%	\infer[\!\!\scriptstyle{a\,\notin\,\fv(A\sigma)}]{\xlink A{\forall a.B}}{\xlink AB}
%\]





