\documentclass[UKenglish]{lipics-v2016}

\usepackage{amsmath,amssymb,amsthm}
\usepackage{latexsym}
\usepackage{microtype}
\usepackage{proof}
\usepackage{stmaryrd}
\usepackage{xargs}

\usepackage{graphicx}

\usepackage{tikz}
\usetikzlibrary{arrows}

\usepackage{willemtools}
\usepackage{proofnet}

\bibliographystyle{plainurl}

% ============================== FOR COMMENTS

\newcommand{\lutz}[1]{{\color{blue}     \noindent[\![\![{\bf Lutz: }#1]\!]\!]}}
\newcommand{\todo}[1]{{\color{red}     \noindent[\![\![{\bf TODO: }#1]\!]\!]}}
\newcommand{\willem}[1]{{\color{green!50!black}     \noindent[\![\![{\bf Willem: }#1]\!]\!]}}

% ============================== MACROS

\makeatletter

\theoremstyle{plain}
\newtheorem{proposition}[theorem]{Proposition}


% ===== Definitions

\newcommand\defn[1]{\textit{\textbf{#1}}}

% ===== General maths

\newcommand\zeroleft[1]{\makebox[0pt][r]{$#1$}}
\newcommand\zeroright[1]{\makebox[0pt][l]{$#1$}}

\newcommand\floor[1]{\lfloor#1\rfloor}

% ===== Sets

\newcommand\var{\textsc{var}}
\newcommand\varA{\textsc{var}^\forall}
\newcommand\varE{\textsc{var}^\exists}
\newcommand\terms{\textsc{term}}
\newcommand\termsA{\textsc{term}^\forall}
\newcommand\atom{\textsc{atom}}
\newcommand\form{\textsc{form}}
\newcommand\proofs{\textsc{proof}}
\newcommand\all{\textsc{all}}
\newcommand\mall{\textsc{mall}}

\newcommand\ex[2][{}]{\textsc{ev}_{#1}(#2)}

\newcommand\subs[1]{\textsc{sub}(#1)}
\newcommand\poss[1]{\textsc{pos}(#1)}
\newcommand\dom[1]{\textsc{dom}(#1)}

% ===== Formulas

\newcommand\+{+}
\renewcommand\*{\times}
\newcommand\dual[1]{\overline{#1}}

\newcommand\sub{\leq}
\newcommand\dep{\preccurlyeq}

\newcommand\seq[2]{{\vdash}#1,#2}
\newcommand\fv{\textsc{fv}}

% ===== Proofs

\newcommand\prf[3]{#1\vdash\!#2,#3}

\newcommand\Seq{\vphantom(\seq}
\newcommand\Prf[3]{\deduce{\Seq{#2}{#3}}{\vphantom(#1}}

% ===== Nets

\newcommand\net[3]{#1\triangleright #2,#3}

\newcommand\deseq[4][\sigma]{[#2]_{#1}^{#3,#4}}
\newcommand\Deseq[4][\sigma]{\left[\vcenter{#2}\right]_{#1}^{#3\,,\,#4}}

\newcommand\comp{\mathbin;}

% ===== Witness maps

\newcommand\mgu{\textsc{mgu}}

\newcommand\gen{\leq}
\newcommand\coh{\smallfrown}
\newcommand\join{\vee}
\newcommand\minus{\mathop{\!/\mathchoice{\kern-3pt}{\kern-3pt}{\kern-2.5pt}{\kern-2pt}/\!}}


\newcommand\res[1]{|_{#1}}
\newcommand\fix[2][2pt]{\overrightharpoon[#1]{#2}}
\newcommand\dcom{\mathbin\cdot}

\newcommand\init[2]{\mgu(#1,#2)}

\newcommand\subdual[1]{_{\smash{\,\dual{\!#1}}}}

% ===== Slices


% ===== Coalescence

\newcommand\link[3][\sigma]{(#2,#3)_{#1}}

\newcommand\scoal{\rightarrow} %{\leadsto}
\newcommand\ucoal{\rightsquigarrow}

\newcommand\Qrr{\!\!\scriptstyle\qrr}

\newcommand\qrr[1]{
  \ifx#1+\expandafter\@qrr\else
  \ifx#1*\*\mathrm R\else
  \ifx#1!\forall\mathrm R\else
  \ifx#1?\expandafter\@@qrr\else
  \ifx#11\mathrm{ax}\else
  \ifx#1.\mathrm{cut}\else
  #1\mathrm R
  \fi\fi\fi\fi\fi\fi
}
\newcommand\@qrr[1]{+\mathrm R,#1}
\newcommand\@@qrr[1]{\exists\mathrm R,#1}

\newcommand\srr[1]{
  \ifx#1+\expandafter\@srr\else
  \ifx#1*\*\mathrm S\else
  \ifx#1!\forall\mathrm S\else
  \ifx#1?\exists\mathrm S\else 
  \ifx#11\mathrm{axS}\else
  #1\mathrm S  \fi\fi\fi\fi\fi
}
\newcommand\@srr[1]{+\mathrm S,{#1}}

\newcommand\urr[1]{
  \ifx#1+\expandafter\@urr\else
  \ifx#1*\*\mathrm U\else
  \ifx#1!\forall\mathrm U\else
  \ifx#1?\exists\mathrm U\else 
  \ifx#11\mathrm{axU}\else
  #1\mathrm U
  \fi\fi\fi\fi\fi
}
\newcommand\@urr[1]{+\mathrm U,{#1}}

% ===== Derivations

\newcommand\sdown{\mathrel{\rotatebox[origin=c]{-90}{$\scoal$}\kern1pt}}
\newcommand\udown{\mathrel{\rotatebox[origin=c]{-90}{$\ucoal$}\kern1pt}}

\newcommandx\sdn[4][2=\sigma]{#1\sdown\link[#2]{#3}{#4}}	%{#2\sdown_{#1}{#3}\mathbin,{#4}}
\newcommandx\udn[4][2=\sigma]{#1\udown\link[#2]{#3}{#4}}	%{#2\udown_{#1}{#3}\mathbin,{#4}}

% ===== (De-)Sequentialization

\newcommand\QU{\Leftrightarrow}

% ===== Overset extensible harpoon

\DeclareFontFamily{U}{MnSymbolA}{}
\DeclareFontShape{U}{MnSymbolA}{m}{n}{
    <-6>  MnSymbolA5
   <6-7>  MnSymbolA6
   <7-8>  MnSymbolA7
   <8-9>  MnSymbolA8
   <9-10> MnSymbolA9
  <10-12> MnSymbolA10
  <12->   MnSymbolA12}{}
\DeclareFontShape{U}{MnSymbolA}{b}{n}{
    <-6>  MnSymbolA-Bold5
   <6-7>  MnSymbolA-Bold6
   <7-8>  MnSymbolA-Bold7
   <8-9>  MnSymbolA-Bold8
   <9-10> MnSymbolA-Bold9
  <10-12> MnSymbolA-Bold10
  <12->   MnSymbolA-Bold12}{}
\DeclareSymbolFont{MnSyA}{U}{MnSymbolA}{m}{n}
\SetSymbolFont{MnSyA}{bold}{U}{MnSymbolA}{b}{n}

\DeclareRobustCommand{\overleftharpoon}{\mathpalette{\overarrow@\leftharpoonfill@}}
\DeclareRobustCommand{\overrightharpoon}{\mathpalette{\overarrow@\rightharpoonfill@}}
\def\leftharpoonfill@{\arrowfill@\leftharpoondown\mn@relbar\mn@relbar}
\def\rightharpoonfill@{\arrowfill@\mn@relbar\mn@relbar\rightharpoonup}

\DeclareMathSymbol{\leftharpoondown}{\mathrel}{MnSyA}{'112}
\DeclareMathSymbol{\rightharpoonup}{\mathrel}{MnSyA}{'100}
\DeclareMathSymbol{\mn@relbar}{\mathrel}{MnSyA}{'320}

\renewcommand\overrightharpoon[2][2pt]{
\begin{tikzpicture}[baseline=0]
	\node[inner sep=0pt,outer sep=0pt,anchor=base] (x) at (0,0) {$#2$};
	\draw[-left to] ($(x.north west) + (0pt,#1)$) -- ($(x.north east) + (0pt,#1)$);
\end{tikzpicture}}

% ===== Permutations

\newcommand\downperm{\sim\rule[-8pt]{0pt}{22pt}}

\newcommand\permAB{
 \infer[\Qrr!]{\Seq{\forall x.A}{\forall y.B}}{
  \infer[\Qrr!]{\Seq A{\forall y.B}}{
   \Seq AB
}}}
\newcommand\permBA{
 \infer[\Qrr!]{\Seq{\forall x.A}{\forall y.B}}{
  \infer[\Qrr!]{\Seq{\forall x.A}B}{
   \Seq AB
}}}
\newcommand\permAE{
 \infer[\Qrr!]{\Seq{\forall x.A}{\exists y.B}}{
  \infer[\Qrr?t]{\Seq A{\exists y.B}}{
   \Seq A{B[t/y]}
}}}
\newcommand\permEA{
 \infer[\Qrr?t]{\Seq{\forall x.A}{\forall y.B}}{
  \infer[\Qrr!]{\Seq{\forall x.A}{B[t/y]}}{
   \Seq A{B[t/y]}
}}}
\newcommand\permEF{
 \infer[\Qrr?s]{\Seq{\exists x.A}{\exists y.B}}{
  \infer[\Qrr?t]{\Seq{A[s/x]}{\exists y.B}}{
   \Seq{A[s/x]}{B[t/y]}
}}}
\newcommand\permFE{
 \infer[\Qrr?t]{\Seq{\exists x.A}{\exists y.B}}{
  \infer[\Qrr?s]{\Seq{\exists x.A}B[t/y]}{
   \Seq{A[s/x]}{B[t/y]}
}}}
\newcommand\permAP{
 \infer[\Qrr!]{\Seq{\forall x.A}{B_1\+B_2}}{
  \infer[\Qrr+i]{\Seq A{B_1\+B_2}}{
   \Seq A{B_i}
}}}
\newcommand\permPA{
 \infer[\Qrr+i]{\Seq{\forall x.A}{B_1\+B_2}}{
  \infer[\Qrr!]{\Seq{\forall x.A}{B_i}}{
   \Seq A{B_i}
}}}
\newcommand\permEP{
 \infer[\Qrr?t]{\Seq{\exists x.A}{B_1\+B_2}}{
  \infer[\Qrr+i]{\Seq{A[t/x]}{B_1\+B_2}}{
   \Seq{A[t/x]}{B_i}
}}}
\newcommand\permPE{
 \infer[\Qrr+i]{\Seq{\exists x.A}{B_1\+B_2}}{
  \infer[\Qrr?t]{\Seq{\exists x.A}{B_i}}{
   \Seq{A[t/x]}{B_i}
}}}
\newcommand\permPQ{
 \infer[\Qrr+i]{\Seq{A_1\+A_2}{B_1\+B_2}}{
  \infer[\Qrr+j]{\Seq{A_i}{B_1\+B_2}}{
   \Seq{A_i}{B_j}
}}}
\newcommand\permQP{
 \infer[\Qrr+j]{\Seq{A_1\+A_2}{B_1\+B_2}}{
  \infer[\Qrr+i]{\Seq{A_1\+A_2}{B_j}}{
   \Seq{A_i}{B_j}
}}}
\newcommand\permAX{
 \infer[\Qrr!]{\Seq{\forall x.A}{B\*C}}{
  \infer[\Qrr*]{\Seq A{B\*C}}{
   \Seq AB & \Seq AC
}}}
\newcommand\permXA{
 \infer[\Qrr*]{\Seq{\forall x.A}{B\*C}}{
  \infer[\Qrr!]{\Seq{\forall x.A}B}{
   \Seq AB 
  }& 
  \infer[\Qrr!]{\Seq{\forall x.A}C}{
   \Seq AC
}}}
\newcommand\permEX{
 \infer[\Qrr?t]{\Seq{\exists x.A}{B\*C}}{
  \infer[\Qrr*]{\Seq{A[t/x]}{B\*C}}{
   \Seq{A[t/x]}B & \Seq{A[t/x]}C
}}}
\newcommand\permXE{
 \infer[\Qrr*]{\Seq{\exists x.A}{B\*C}}{
  \infer[\Qrr?t]{\Seq{\exists x.A}B}{
   \Seq{A[t/x]}B 
  }& 
  \infer[\Qrr?t]{\Seq{\exists x.A}C}{
   \Seq{A[t/x]}C
}}}
\newcommand\permPX{
 \infer[\Qrr+i]{\Seq{A_1\+A_2}{B\*C}}{
  \infer[\Qrr*]{\Seq{A_i}{B\*C}}{
   \Seq{A_i}B & \Seq{A_i}C
}}}
\newcommand\permXP{
 \infer[\Qrr*]{\Seq{A_1\+A_2}{B\*C}}{
  \infer[\Qrr+i]{\Seq{A_1\+A_2}B}{
   \Seq{A_i}B 
  }& 
  \infer[\Qrr+i]{\Seq{A_1\+A_2}C}{
   \Seq{A_i}C
}}}
\newcommand\permXY{
 \infer[\Qrr*]{\Seq{A\*B}{C\*D}}{
  \infer[\Qrr*]{\Seq A{C\*D}}{
   \Seq AC & \Seq AD
  }&
  \infer[\Qrr*]{\Seq B{C\*D}}{
   \Seq BC & \Seq BD
}}}
\newcommand\permYX{
 \infer[\Qrr*]{\Seq{A\*B}{C\*D}}{
  \infer[\Qrr*]{\Seq{A\*B}C}{
   \Seq AC & \Seq BC
  }&
  \infer[\Qrr*]{\Seq{A\*B}D}{
   \Seq AD & \Seq BD
}}}

% ===== Cut-permutations

\newcommand\permAC{
 \infer[\Qrr.]{\Seq{\forall x.A}C}{
  \infer[\Qrr!]{\Seq{\forall x.A}B}{
   \Seq AB 
  }& 
  \Seq{\dual B}C
}}
\newcommand\permCA{
 \infer[\Qrr!]{\Seq{\forall x.A}C}{
  \infer[\Qrr.]{\Seq AC}{
   \Seq AB & \Seq{\dual B}C
}}}
\newcommand\permEC{
 \infer[\Qrr.]{\Seq{\exists x.A}C}{
  \infer[\Qrr?t]{\Seq{\exists x.A}B}{
   \Seq{A[t/x]}B 
  }& 
  \Seq{\dual B}C
}}
\newcommand\permCE{
 \infer[\Qrr?t]{\Seq{\exists x.A}C}{
  \infer[\Qrr.]{\Seq{A[t/x]}C}{
   \Seq{A[t/x]}B & \Seq{\dual B}C
}}}
\newcommand\permPC{
 \infer[\Qrr.]{\Seq{{A_1\+A_2}}C}{
  \infer[\Qrr+i]{\Seq{{A_1\+A_2}}B}{
   \Seq {A_i}B 
  }& 
  \Seq{\dual B}C
}}
\newcommand\permCP{
 \infer[\Qrr+i]{\Seq{{A_1\+A_2}}C}{
  \infer[\Qrr.]{\Seq {A_i}C}{
   \Seq {A_i}B & \Seq{\dual B}C
}}}
\newcommand\permXC{
 \infer[\Qrr.]{\Seq{{A_1\*A_2}}C}{
  \infer[\Qrr*]{\Seq{{A_1\*A_2}}B}{
   \Seq {A_1}B & \Seq{A_2}B
  }& 
  \Seq{\dual B}C
}}
\newcommand\permCX{
 \infer[\Qrr*]{\Seq{{A_1\*A_2}}C}{
  \infer[\Qrr.]{\Seq {A_i}C}{
   \Seq {A_1}B & \Seq{\dual B}C
  }&
  \infer[\Qrr.]{\Seq {A_i}C}{
   \Seq {A_2}B & \Seq{\dual B}C
}}}
\newcommand\permCD{
 \infer[\Qrr.]{\Seq AD}{
  \infer[\Qrr.]{\Seq AC}{
   \Seq AB & \Seq{\dual B}C
  }& 
  \Seq{\dual C}D
}}
\newcommand\permDC{
 \infer[\Qrr.]{\Seq{{A_1\*A_2}}C}{
  \Seq A{\dual B}
  &
  \infer[\Qrr.]{\Seq BD}{
   \Seq B{\dual C} & \Seq CD
}}}

\makeatother

% ============================== TITLE & AUTHORS

\title{Proof nets for first-order additive linear logic}
%\titlerunning{Proof nets for ALL1}

\author[1]{Willem B.\ Heijltjes}
\author[2]{Dominic J.D.\ Hughes}
\author[3]{Lutz Stra\ss burger}
\affil[1]{University of Bath, United Kingdom\\
  \texttt{w.b.heijltjes@bath.ac.uk}}
\affil[2]{UC Berkeley, USA.\\
  \texttt{dominich@theory.stanford.edu}}
\affil[3]{INRIA \&\ \'Ecole Polytechnique, Palaiseau, France\\
  \texttt{lutz.strassburger@inria.fr}}
\authorrunning{W.B.\ Heijltjes, D.J.D.\ Hughes, and L.\ Stra\ss burger}

\Copyright{Willem B.\ Heijltjes, Dominic J.D.\ Hughes, and Lutz Stra\ss burger}

\subjclass{F.4.1: Mathematical logic--Proof theory}
\keywords{Linear logic, First-order logic, Proof nets}



% ============================== CONTENT


\begin{document}

\maketitle

\begin{abstract}
We extend proof nets for additive linear logic with first-order quantification. 

We present two versions of our proof nets. One, witness nets, retains explicit witnessing information to existential quantification. For the other, unification nets, this information is absent but can be reconstructed through unification. Unification nets embody a main contribution of the paper: first-order witness information can be left implicit, and reconstructed effectively. 

Witness nets are canonical for first-order additive sequent calculus. Unification nets in addition factor out any inessential choice for existential witnesses. 
Both notions of proof net are defined through coalescence (the additive version of multiplicative contractibility), and for witness nets a geometric correctness criterion is provided. Both feature a global composition operation.
\end{abstract}


% ==================================================

\section{Introduction}

%Proof nets aim to be canonical representations of proofs. As graphs, they abstract away from the rule commutations of the sequent calculus. 
%
%In the study of \emph{proof nets} \cite{Girard-1987}, canonical graphical representations of proofs, the additive fragment of linear logic provides several interesting challenges. At the propositional level, the challenge is to capture the merging of the contexts $\Gamma$ in the additive product rule:
%\[
%	\vc{\infer[\Qrr*]{\seq\Gamma{A\*B}}{\seq\Gamma A & \seq\Gamma B}}~.
%\]
%One approach, perfect for the multiplicative fragment, is to create proof nets from sequent proofs by separating rules from their context: each sequent rule becomes a node in the graph, connected only to the primary formulas of the rule, in this case $A$, $B$, and $A\*B$. The benefit is a direct interpretation from sequent calculus; 


Additive linear logic $(\all)$ is the logic of product $(\*)$ and sum $(\+)$. Proofs, over a sequent $A\vdash B$ or, here, $\seq{\dual A}B$, are built from the canonical morphisms: projections, injections, and diagonals. These are captured naturally in \emph{proof nets} \cite{Girard-1987,Hughes-vanGlabbeek-2005}, which consist of a set of \emph{links} between subformulas in $\dual A$ and in $B$, satisfying a simple geometric correctness condition.
\[
\begin{array}{c@{\qquad}c@{\qquad}c@{\qquad}c}
	\begin{tikzpicture}[net]
		\formula[y=1.2]{{\dual a}}
		\formula[y=0]{a\*a}
		\Vlink[red]{[-2]1,1}
		\Vlink[red]{[2]1,3}
	\end{tikzpicture}
&
	\begin{tikzpicture}[net]
		\formula[y=1.2]{{\dual a}}
		\formula[y=0]{a\+b}
		\Vlink[red]{1,1}
	\end{tikzpicture}
&
	\begin{tikzpicture}[net]
		\formula[y=1.2]{{\dual a}\+{\dual b}}
		\formula[y=0]{b\*a}
		\Vlink[red]{1,3}
		\Vlink[red]{3,1}
	\end{tikzpicture}
&
	\begin{tikzpicture}[net]
		\formula[y=1.2]{*{\phantom(}{\dual a}*{\phantom)}
		 {{\+}}*({\dual b}*{\phantom)}
		 {{\+}}*{\phantom(}{\dual c}*)}
		\formula[y=0]{*(a*{\phantom)}
		{{\*}}*{\phantom(}b*)
		{{\*}}*{\phantom(}c*{\phantom)}}
		\Vlink[red]{1,1}
		\Vlink[red]{3,3}
		\Vlink[red]{5,5}
	\end{tikzpicture}
\\
	\text{\small diagonal}
&	\text{\small injection}
&	\text{\small symmetry}
&	\text{\small associativity}
\end{array}
\]
%
The logic is combinatorially rich, yet well-behaved and tractable: proof search \cite{Galmiche-Marion-1995,Galmiche-2000} and proof net correctness \cite{Heijltjes-Hughes-2015} are linear in $|A|\times|B|$ (with $|A|$ the size of the syntax tree of $A$); proof nets remain canonical and equally tractable when extended with the two units \cite{Heijltjes-2011,Heijltjes-Hughes-2015}; and the first-order case is merely NP-complete \cite{Heijltjes-Hughes-2015}.

At the same time, the ubiquity of its main operations, product and sum, mean the ideas and results it garners are widely applicable. It serves as a microcosmos for important ideas and observations: it fostered the \emph{connections method} of proof search \cite{Galmiche-2000}; it demonstrated the \emph{Blass problem} \cite{Abramsky-2003} of game semantics, that sequential strategies do not in general have associative composition; and it is at the root of the study of fixed points in linear logic \cite{Santocanale-2002a}. The logic describes two-way communication~\cite{Cockett-Santocanale-2009}, and is a core part of \emph{session types}~\cite{Honda-1993}.

{\all} is of course part of {\mall} (multiplicative-additive linear logic), and its lessons are clearly visible in the second author's canonical proof nets \cite{Hughes-vanGlabbeek-2005} for that fragment, as well as the first and second author's locally canonical \emph{conflict nets}~\cite{Hughes-Heijltjes-2016}. But its proof nets also appear as the \emph{skew fibrations} in the second author's \emph{combinatorial proofs} for classical logic~\cite{Hughes-2006-proofs}, and in the third authors study of the \emph{medial rule} for classical logic~\cite{str:RTA07}. To prepare the ground for the extension of \emph{combinatorial proofs} to first-order classical logic is one motivation for the% current presentation.

We present proof nets for first-order additive linear logic (\all1). The main content of first-order proof, and our central challenge, is the \emph{witness assignment} to existential quantifiers. 
%
The standard approach to evaluate a formula $\exists x.A$ is by a substitution $A[t/x]$, which destroys the subformula $A$.%
%
\footnote{Given $A[t/x]$ and $t$, it is not possible to unambiguously recover $A$, which may contain other occurrences of $t$ not originating in the substitution $[t/x]$.}
%
But the current, successful formulation of additive proof nets, as a set of links over a two-formula sequent, requires subformulas to remain intact.
%
Moreover, the same quantification $\exists x.A$ may occur in different \emph{slices} (branches of a proof above a product), with different witnesses, as below. But the \emph{formulas + links} design has only one instance of $\exists x.A$.
\[
\vc{
	\infer[\Qrr*]{\Seq{\exists x.A}{B\*C}}{
	 \infer[\Qrr?s]{\Seq{\exists x.A}B}{\Seq{A[s/x]}B}
	 &
	 \infer[\Qrr?t]{\Seq{\exists x.A}C}{\Seq{A[t/x]}C}
   }}
~.
\]
%
We advance two solutions. \defn{Witness nets} replace the \emph{implicit} substitutions of sequent proofs with \emph{explicit} substitutions, recorded at each link. \defn{Unification nets} omit existential witnesses altogether, and reconstruct them by unification. Respective examples are below.
\[
\begin{tikzpicture}[net]
	\formula[y=1.2]{\forall*{x.}\exists*{y.~}{\dual P}*{(y)}\+{\dual Q}*{(x,y)}}
	\formula[y=0]{\exists*{z.~}P*{(f(z))}\*Q*{(z,t)}}
	\Vlink[red,l,label={$\scriptstyle[f(x)/y\,,\,x/z]$~~}]{3,2}
	\Vlink[blue,r,label={~$\scriptstyle[t/y\,,\,x/z]$}]{5,4}
\end{tikzpicture}
\qquad\qquad
\begin{tikzpicture}[net]
	\formula[y=1.2]{\forall*{x.}\exists*{y.~}{\dual P}*{(y)}\+{\dual Q}*{(x,y)}}
	\formula[y=0]{\exists*{z.~}P*{(f(z))}\*Q*{(z,t)}}
	\Vlink[red]{3,2}
	\Vlink[blue]{5,4}
\end{tikzpicture}
\]
%
Correctness and sequentialization for \all1 proof nets will be through \defn{coalescence}~\cite{Heijltjes-Hughes-2015,Hughes-Heijltjes-2016}, the additive counterpart to \defn{contractibility} for multiplicative linear logic~\cite{Danos-1990}. It is a simple graph rewrite relation that pushes links from the leaves towards the roots of a sequent, that is effectively top-down sequentialization (like its multiplicative counterpart~\cite{Guerrini-Masini-2001}). Figure~\ref{fig:main example} illustrates the process for the witness net above. In the example, a link between subformulas $A$ and $B$ carrying a substitution map $\sigma$ represents a sequent $\seq{A\sigma}{B\sigma}$. Each coalescence step ($\star S$) corresponds to a proof rule ($\star R$), by which the sequence generates the sequent proof.

In unification nets, existential witnesses are reconstructed during coalescence. The initial substitution map assigned to a link on atomic formulas $a=P(s_1,\dots,s_n)$ and $b=\dual P(t_1,\dots,t_n)$ is the most general unifier (\mgu) of $a$ and $\dual b$ (the first step in Figure~\ref{fig:unification example}). Coalescence then proceeds as for witness nets, except when joining two links from different slices with a product step, ($\urr*$) in Figure~\ref{fig:unification example}. Here, witness nets require both links to carry the same substitution map, while for unification nets a common, more general map is generated by unification.

We include the following results for our \all1 proof nets. Witness nets are \emph{canonical} for the permutations of the sequent calculus, and we provide a geometric correctness condition. \emph{Composition} of unification nets (over $\seq AB$ and $\seq{\dual BC}$ into one over $\seq AC$) is remarkedly simple, as the relational composition of both sets of links. For composition of witness nets, in addition the witness assignments of links must be composed, which is done through a simple process of \emph{interaction + hiding} similar to those of game semantics \cite{Mimram-2011}.

A companion paper \cite{Hughes-2018} explores unification nets for first-order \emph{multiplicative} linear logic, reporting drastic complexity improvements and equally straightforward proof net composition.

% Hyland-Ong-2000,Abramsky-Jagadeesan-Malacaria-2000

% ..................................................

\begin{figure}
\[
\begin{array}{c}
\begin{array}{@{}c@{~}c@{\qquad}c@{}}
  \vc{
  \begin{tikzpicture}[net]
	\formula[y=1.2]{\forall*{x.}\exists*{y.~}{\dual P}*{(y)}\+{\dual Q}*{(x,y)}}
	\formula[y=0]{\exists*{z.~}P*{(f(z))}\*Q*{(z,t)}}
	\Vlink[red,l,label={$\scriptstyle[f(x)/y\,,\,x/z]$~~}]{3,2}
	\Vlink[blue,r,label={~$\scriptstyle[t/y\,,\,x/z]$}]{5,4}
  \end{tikzpicture}}
&&
  \vc{
  \begin{tikzpicture}[net]
	\formula[y=1.2]{\forall*{x.}\exists*{y.~}{\dual P}*{(y)}\+{\dual Q}*{(x,y)}}
	\formula[y=0]{\exists*{z.~}P*{(f(z))}\*Q*{(z,t)}}
	\Vlink[violet,l,label={$\scriptstyle\varnothing$~~}]{1,1}
  \end{tikzpicture}} 
\\ \\[-5pt] \downarrow\scriptstyle{\srr+1} && \uparrow\scriptstyle\srr! \\ \\[-5pt]
  \vc{
  \begin{tikzpicture}[net]
	\formula[y=1.2]{\forall*{x.}\exists*{y.~}{\dual P}*{(y)}\+{\dual Q}*{(x,y)}}
	\formula[y=0]{\exists*{z.~}P*{(f(z))}\*Q*{(z,t)}}
	\Vlink[red,l,label={$\scriptstyle[f(x)/y\,,\,x/z]$~~~}]{4,2}
	\Vlink[blue,r,label={~$\scriptstyle[t/y\,,\,x/z]$}]{5,4}
  \end{tikzpicture}}
&&
  \vc{
  \begin{tikzpicture}[net]
	\formula[y=1.2]{\forall*{x.}\exists*{y.~}{\dual P}*{(y)}\+{\dual Q}*{(x,y)}}
	\formula[y=0]{\exists*{z.~}P*{(f(z))}\*Q*{(z,t)}}
	\Vlink[violet,l,label={$\scriptstyle\varnothing$~~}]{2,1}
  \end{tikzpicture}} 
\\ \\[-5pt] \downarrow\scriptstyle\srr?,f(x) && \uparrow\scriptstyle\srr?,x \\ \\[-5pt]  
  \vc{
  \begin{tikzpicture}[net]
	\formula[y=1.2]{\forall*{x.}\exists*{y.~}{\dual P}*{(y)}\+{\dual Q}*{(x,y)}}
	\formula[y=0]{\exists*{z.~}P*{(f(z))}\*Q*{(z,t)}}
	\Vlink[red,l,label={$\scriptstyle[x/z]$~~}]{2,2}
	\Vlink[blue,r,label={~$\scriptstyle[t/y\,,\,x/z]$}]{5,4}
  \end{tikzpicture}} 
&&  
  \vc{
  \begin{tikzpicture}[net]
	\formula[y=1.2]{\forall*{x.}\exists*{y.~}{\dual P}*{(y)}\+{\dual Q}*{(x,y)}}
	\formula[y=0]{\exists*{z.~}P*{(f(z))}\*Q*{(z,t)}}
	\Vlink[violet,l,label={$\scriptstyle[x/z]$~~~}]{[2]2,3}
  \end{tikzpicture}}
\\ \\[-5pt] \downarrow\scriptstyle{\srr+2} && \uparrow\scriptstyle\srr* \\ \\[-5pt]
  \vc{
  \begin{tikzpicture}[net]
	\formula[y=1.2]{\forall*{x.}\exists*{y.~}{\dual P}*{(y)}\+{\dual Q}*{(x,y)}}
	\formula[y=0]{\exists*{z.~}P*{(f(z))}\*Q*{(z,t)}}
	\Vlink[red,l,label={$\scriptstyle[x/z]$~~}]{2,2}
	\Vlink[blue,r,label={~$\scriptstyle[t/y\,,\,x/z]$}]{4,4}
  \end{tikzpicture}}
& \stackrel{\srr?,t}\scoal & 
  \vc{
  \begin{tikzpicture}[net]
	\formula[y=1.2]{\forall*{x.}\exists*{y.~}{\dual P}*{(y)}\+{\dual Q}*{(x,y)}}
	\formula[y=0]{\exists*{z.~}P*{(f(z))}\*Q*{(z,t)}}
	\Vlink[red,l,label={$\scriptstyle[x/z]$~~}]{2,2}
	\Vlink[blue,r,label={~~~$\scriptstyle[x/z]$}]{[4]2,4}
  \end{tikzpicture}}
\end{array}
\\ \\[20pt]
	\color{violet}\infer[\color{black}\Qrr!]{\color{black}\Seq{\forall x.\exists y.\dual P(y)\+\dual Q(x,y)}{\exists z.P(f(z))\*Q(z,t)}}{
	 \color{violet}\infer[\color{black}\Qrr?,x]{\color{black}\Seq{\exists y.\dual P(y)\+\dual Q(x,y)}{\exists z.P(f(z))\*Q(z,t)}}{
	  \color{violet}\infer[\color{black}\Qrr*]{\color{black}\Seq{\exists y.\dual P(y)\+\dual Q(x,y)}{P(f(x))\*Q(x,t)}}{
	   \color{red}\infer[\color{black}\Qrr?{f(x)}]{\color{black}\Seq{\exists y.\dual P(y)\+\dual Q(x,y)}{P(f(x))}}{
	    \color{red}\infer[\color{black}\Qrr+1]{\color{black}\Seq{\dual P(f(x))\+\dual Q(x,f(x))}{P(f(x))}}{
	     \color{red}\infer[\color{black}\Qrr1]{\color{black}\Seq{\dual P(f(x))}{P(f(x))}}{}
	   }}&
	   \color{blue}\infer[\color{black}\Qrr?t]{\color{black}\Seq{\exists y.\dual P(y)\+\dual Q(x,y)}{Q(x,t)}}{
	    \color{blue}\infer[\color{black}\Qrr+2]{\color{black}\Seq{\dual P(t)\+\dual Q(x,t)}{Q(x,t)}}{
	     \color{blue}\infer[\color{black}\Qrr1]{\color{black}\Seq{\dual Q(x,t)}{Q(x,t)}}{}
	}}}}}
\end{array}
\]
\caption{A coalescence and sequentialization example}
\label{fig:main example}
\end{figure}

% ..................................................

\begin{figure}
\[
\vc{\begin{tikzpicture}[net]
	\formula[y=1.2]{{\dual P}*{(x,x)}}
	\formula[y=0]{P*{(y,s)}\*P*{(t,z)}}
	\Vlink[red]{[-2]1,1}
	\Vlink[blue]{[2]1,3}
\end{tikzpicture}}
\quad\ucoal\quad
\vc{\begin{tikzpicture}[net]
	\formula[y=1.2]{{\dual P}*{(x,x)}}
	\formula[y=0]{P*{(y,s)}\*P*{(t,z)}}
	\Vlink[red,l,label={$\scriptstyle[s/y,s/x]$~~}]{[-2]1,1}
	\Vlink[blue,r,label={~~$\scriptstyle[t/z,t/x]$}]{[2]1,3}
\end{tikzpicture}}
\quad\stackrel{\urr*}\ucoal\quad
\vc{\begin{tikzpicture}[net]
	\formula[y=1.2]{{\dual P}*{(x,x)}}
	\formula[y=0]{P*{(y,s)}\*P*{(t,z)}}
	\Vlink[violet,r,label={~~$\scriptstyle[u/z,u/y,u/x]$}]{1,2}
\end{tikzpicture}}
\]
\caption{Coalescence with unification; $u$ is the smallest term unifying $s$ and $t$} 
\label{fig:unification example}
\end{figure}

% ..................................................

% --------------------------------------------------

\subsection{Proof identity}

At the heart of a theory of proof nets is the question of \emph{proof identity}: when are two proofs equivalent? The answer determines which proofs should map onto the same proof net. The introduction of quantifiers creates an interesting issue: if two proofs differ by an immaterial choice of existential witness, should they be equivalent? For example, to prove the sequent $\seq{\exists x.P(x)}{\exists y.\dual P(y)}$ both quantifiers must receive \emph{the same} witness, as in the following two proofs, but any witness will do.
\[
	\vc{\infer={\seq{\exists x.P(x)}{\exists y.\dual P(y)}}{\infer{\seq{P(s)}{\dual P(s)}}{}}}
	\quad
	\stackrel?\equiv
	\quad
	\vc{\infer={\seq{\exists x.P(x)}{\exists y.\dual P(y)}}{\infer{\seq{P(t)}{\dual P(t)}}{}}}
\]
The issue is more pronounced where quantifiers are \emph{vacuous}, $\exists x.A$ where $x$ is not free in $A$. The proofs below left can only be distinguished even syntactically because the $\exists R$-rule makes the instantiating witness explicit. Below right is an interesting intermediate variant: the witness $s$ or $t$ can be observed without explicit annotation in the $\exists R$-rule, but the choice is equally immaterial to the content of the proof as when the quantifier were vacuous.
\[
	\vc{\infer[\!\!\scriptstyle{\qrr?s}]{\seq{\exists x.P}{\dual P}}{\infer{\seq P{\dual P}}{}}}
	\quad
	\stackrel?\equiv
	\quad	
	\vc{\infer[\!\!\scriptstyle{\qrr?t}]{\seq{\exists x.P}{\dual P}}{\infer{\seq P{\dual P}}{}}}\qquad
	\vc{
	 \infer[\!\!\scriptstyle{\qrr?s}]{\seq{\exists x.P\+Q(x)}{\dual P}}{
	  \infer[\!\!\scriptstyle{\qrr+1}]{\seq{P\+Q(s)}{\dual P}}{\infer{\seq P{\dual P}}{}}
	}}
	\quad
	\stackrel?\equiv
	\quad	
	\vc{
	 \infer[\!\!\scriptstyle{\qrr?t}]{\seq{\exists x.P\+Q(x)}{\dual P}}{
	  \infer[\!\!\scriptstyle{\qrr+1}]{\seq{P\+Q(t)}{\dual P}}{\infer{\seq P{\dual P}}{}}
	}}
\]
In this paper we will not attempt to settle the question of proof identity. Rather, our two notions of proof net each represent a natural and coherent perspective, at either end of the spectrum. \emph{Witness nets} make all existential witnesses explicit, including those to vacuous quantifiers, rejecting all three equivalences above. \emph{Unification nets} leave all witnesses implicit, thus identifying all proofs modulo witness assignment, and validating all three equivalences.

\subsection{Monomial nets and Expansion Trees}

Proof nets with additives and quantifiers exist as \emph{monomial nets} \cite{Girard-1996}, where slices are managed explicitly with \emph{monomial weights}. These nets are not generally canonical: they admit the permutation (and duplication) of proof rules past implicit \emph{contractions}. We believe (though have not technically verified) that additive monomial nets include, and could be restricted to, canonical forms. These would be equivalent to the following approach.

\emph{Expansion tree proofs} for classical logic \cite{Miller-1987,Heijltjes-2010} provide another approach to canonicity for first-order quantification. In this formulation, a formula $\exists x.A$ is interpreted as the sum (or classically, \emph{disjunction}) over a fixed number of instantiations $A[t_1/x]\+\dots\+A[t_n/x]$. The idea can be traced to Herbrand's Theorem \cite{Herbrand-1930}: $\exists x.A$ is equivalent to the infinite sum over $A[t/x]$ for all terms $t$ in the language, but for any given proof a finite set of terms suffices.

We use this approach implicitly for our geometric correctness condition for witness nets (Section~\ref{sec:correctness}), which generalizes the existing \emph{slice-based} criterion for additive proof nets \cite{Hughes-vanGlabbeek-2005}. In constructing the slices over a formula, we interpret $\exists x.A$ by an \emph{expansion} $A[t_1/x]\+\dots\+A[t_n/x]$, where the witnesses $t_1$ through $t_n$ are collected from the substitution maps of the links in the proof net.

For the definition of proof nets, we did not pursue this direction, and consider the present choice preferable. Witness nets and unification nets retain the original structure of \emph{formulas + links}, without requiring expansion trees. More importantly, as in the propositional case, they enable direct (non-inductive) composition by a relational composition of links. Finally, available space does not permit a detailed exposition, which we defer to a journal version.



%$\seq{\,\exists x.P(x,x,s)\,}{\,\exists y.\dual P(t,y,y)}$ 

% --------------------------------------------------

%\subsection{Background and related work}

%
%\cite{Whitman-1941}
%\cite{Girard-1987}
%\cite{Galmiche-Marion-1995} % ALL proof search
%\cite{Girard-1996}
%\cite{Hu-1999}
%\cite{Galmiche-2000} % ALL connection net (Def 6.11.)
%\cite{Cockett-Seely-2001}
%\cite{Hughes-VanGlabbeek-2005}
%%\cite{Cockett-Santocanale-2009}
%\cite{Heijltjes-2011, Heijltjes-2012}
%\cite{Heijltjes-Hughes-2015}
%
%\cite{Danos-1990}
%\cite{Guerrini-Masini-2001}
%\cite{Guerrini-2011}
%
%\cite{Herbrand-1930}
%\cite{Heijltjes-2010}

% ==================================================

\section{Proof nets for first-order additive linear logic}

% ..................................................

\begin{figure}
\[
	\infer[\Qrr1]{\vphantom B\seq a{\dual a}}{}
\quad
	\infer[\Qrr+i]{\seq A{B_1\+B_2}}{\seq A{B_i}}
\quad
	\infer[\Qrr*]{\seq A{B\*C}}{\seq AB & \seq AC}
\quad
	\infer[\Qrr?t]{\seq A{\exists x.B}}{\seq A{B[t/x]}}
\quad
	\infer[\Qrr!~(x\,\notin\,\textsc{fv}(A))]{\seq A{\forall x.B}}{\seq AB}
\]
\caption{A sequent calculus for $\all1$}
\label{fig:sequent calculus}
\end{figure}

% ..................................................

% --------------------------------------------------

\subsection{First-order additive linear logic}

First-order terms and the formulas of first-order $\all$ are generated by the following grammars.
%
\setMidspace{5pt}
\[
\begin{array}{@{}l@{}l}
	t &\Coloneqq x \Mid f(t_1,\dots,t_n)
\\[10pt]
	a &\Coloneqq P(t_1,\dots,t_n) \Mid \dual P(t_1,\dots,t_n)
\\[10pt]
	A &\Coloneqq a \Mid A\+A \Mid A\*A \Mid \exists x.A \Mid \forall x.A
\end{array}
\]
%
Negation $(\dual{\,\cdot\,})$ is applied to predicate symbols, $\dual P$ as a matter of convenience. The \defn{dual} $\dual A$ of an arbitrary formula $A$ is given by DeMorgan. We use the following notational conventions.
%
\[
\begin{tabular}{@{}lll@{}} 
	$x,y,z$ & $\in \var$		& first-order variables\\
	$f,g,h$ & $\in \Sigma_f$	& $n$-ary $(n\geq 0)$ function symbols from a fixed alphabet $\Sigma_f$\\
	$P,Q,R$ & $\in \Sigma_p$	& $n$-ary $(n\geq 0)$ predicate symbols from a fixed alphabet $\Sigma_p$ \\
	$s,t,u$ & $\in \terms$ 		& first-order terms over $\var$ and $\Sigma_f$ \\
	$a,b,c$ & $\in \atom$		& atomic propositions \\
	$A,B,C$ & $\in \form$		& \all1 formulas \\
\end{tabular}
\]
%
A \defn{sequent} $\seq AB$ is a pair of formulas $A$ and $B$. A sequent calculus for \all1 is given in Figure~\ref{fig:sequent calculus}, where each rule has a symmetric counterpart for the first formula in the sequent. We write $\prf\pi AB$ for a proof $\pi$ with conclusion sequent $\seq AB$. Two proofs are \defn{equivalent} $\pi\sim\pi'$ if one is obtained from the other by rule permutations (Appendix, Figure~\ref{fig:permutations}.

By a \defn{subformula} we will mean a subformula \defn{occurrence}. For instance, a formula $A\*A$ has two subformulas $A$, one on the left and one on the right. The \defn{subformulas} $\subs A$ of a formula are defined as follows; we write $B\sub A$ if $B$ is a subformula of $A$, i.e.\ if $B\in\subs A$.
\[
	\subs A = \{A\} \cup
	\left\{\begin{array}{ll}
		\subs B\uplus\subs C	& \text{if $A=B\+C$ or $A=B\*C$} \\[5pt]
		\subs B					& \text{if $A=\exists x.B$ or $A=\forall x.B$}
	\end{array}\right.
\]

A \defn{link} $(C,D)$ on a sequent $\seq AB$ is a pair of subformulas $C\leq A$ and $D\leq B$. 
A \defn{linking} $\lambda$ on the sequent $\seq AB$ is a set of links on it.

% ..................................................

\begin{definition}
A \defn{pre-net} $\net\lambda AB$ is a sequent $\seq AB$ with a linking $\lambda$ on it.
\end{definition}

% --------------------------------------------------

\subsection{Witness maps}

A \defn{witness map} $\sigma\colon\var\rightharpoonup\terms$ is a substitution map which assigns terms to variables, given as a (finite) partial function. We define it as $\sigma=[t_1/x_1,\dots,t_n/x_n]$, where its \defn{domain} $\dom\sigma$ is $\{x_1,\dots,x_n\}$. We abbreviate by $y\in\sigma$ that a variable $y$ occurs free in the range of~$\sigma$, i.e., $y\in\fv(t_i)$ for some $i\leq n$. The map $\sigma\minus x$ is undefined on $x$ and otherwise as $\sigma$, we write $\sigma\res V$ for the restriction of $\sigma$ to a set of variables $V$, and $\varnothing$ for the empty witness map.
%
We write $A\sigma$ for the application of the substitutions in $\sigma$ to the formula $A$, and $\pi\sigma$ for its application to the proof $\pi$, where it is applied to each formula in the proof, and to each existential witness $t$ recorded with a rule $\qrr?t$. The \defn{composition} of two maps is written $\sigma\tau$, where $A(\sigma\tau)=(A\sigma)\tau$.

%\[
%\begin{tabular}{@{}ll@{}} 
%	$\varnothing$ & is the empty witness map \\
%	$\sigma\minus x$ & is the witness map that is undefined on $x$ and otherwise as $\sigma$ \\
%	$\sigma\res V$ & is the restriction of $\sigma$ to a set of variables $V$\\
%	$y\in\sigma$ & abbreviates $y\in\fv(\sigma(x))$ for some $x$ ($y$ occurs in the range of $\sigma$) \\
%	$A\sigma$ & is the application of $\sigma$ as a substitution map to a formula $A$ \\
%	$\pi\sigma$ & is the application of $\sigma$ to a proof $\pi$ (applied to all its formulas) \\
%\end{tabular}
%\]

A \defn{witness linking} $\lambda_\Sigma$ is a linking $\lambda$ with a \defn{witness labelling} $\Sigma\colon\lambda\to\var\rightharpoonup\terms$ that assigns each link $(C,D)$ a witness map. We may use and define $\lambda_\Sigma$ as a set of \defn{witness links} $\link CD$, where $(C,D)\in\lambda$ and $\Sigma(C,D)=\sigma$.

% ..................................................

\begin{definition}
A \defn{witness pre-net} $\net{\lambda_\Sigma}AB$ is a sequent $\seq AB$ with a witness linking $\lambda_\Sigma$.
\end{definition}

% ..................................................

We will assume the following variable naming conventions.
%
\begin{description}
	\item
[Barendregt's convention] All quantifiers in a sequent $\seq AB$ have a distinct binding variable, and no bound variable shares a name with a free one.

	\item[Eigenvariables not free]
For a link $\link CD$ over $\seq AB$, if a variable $x$ in the range of $\sigma$ is universally quantified as $\forall x.X$ in $\seq AB$ (an \emph{eigenvariable}), then $C\leq X$ or $D\leq X$.

	\item[Freshness]
For a link $\link CD$ over $\seq AB$, other variables in the range of $\sigma$ are \emph{fresh}: distinct from existentially quantified variables in $\seq AB$, and from those of other links.
\end{description}

% ..................................................

\begin{definition}
The \defn{de-sequentialization} $[\pi]$ of a sequent proof $\prf\pi AB$ is the witness pre-net $\net{\deseq[\varnothing]\pi AB}AB$ where the function $\deseq-AB$ is defined inductively as follows.
%
\newcommand\XDeseq[4][\sigma]{
 \left[\vcenter{#2}\right]%
 _{\makebox[0pt][l]{$\scriptstyle{#1}$}}%
 ^{\makebox[0pt][l]{$\scriptstyle{#3\,,\,#4}$}}%
 \rule{35pt}{0pt}
}
%
\begin{align*}
	\XDeseq{\,\infer[\Qrr1]{\seq a{\dual a}}{}}bc &\quad=\quad \{\link bc\}
\\ \\[-5pt]
	\XDeseq{\infer[\Qrr+i]{\seq A {B_1\+B_2}}{\Prf \pi A{B_i}}} {A'} {B_1'\+B'_2} 
	& \quad=\quad 
	\Deseq{\Prf\pi A{B_i}} {A'} {B_i'}
\\ \\[-5pt]
	\XDeseq{\infer[\Qrr*]{\seq A{B\*C}}{
	 \Prf\pi AB && \Prf{\pi'}AC 
	}}{A'}{B'\*C'}
	& \quad=\quad 
	  \Deseq{\Prf\pi AB}{A'}{B'}
	  \cup
	  \Deseq{\Prf{\pi'}AC}{A'}{C'}
\\ \\[-5pt]
	\XDeseq{\infer[\Qrr?t]{\seq A{\exists x.B}}{\Prf\pi A{B[t/x]}}}{A'}{\exists x.B'}
	& \quad=\quad 
	 \Deseq[{\sigma[t/x]}]{\Prf\pi A{B[t/x]}} {A'} {B'}
\\ \\[-5pt]
	\XDeseq{\infer[\Qrr!]
	  {\seq A{\forall x.B}}{\Prf\pi AB}} {A'} {\forall x.B'}
	& \quad=\quad 
	  \Deseq{\Prf\pi AB}{A'}{B'}
\end{align*}
\end{definition}

% ..................................................

A function call $\deseq{\prf\pi AB}{A'}{B'}$ expects that $A=A'\sigma$ and $B=B'\sigma$: the translation separates a sequent $\seq AB$ into subformulas $A'$, $B'$ of the ultimate conclusion of the proof, and the accumulated existential witnesses $\sigma$.
%While a sequent proof instantiates existential witnesses immedately into the premise of the existential quantifier rule, for proof nets we delay the substitutions $\sigma$. Hence we need to maintain the separation of the conclusion $\seq AB$ into the substitutions $\sigma$ and the subformulas $A'$ and $B'$ of the conclusion of the proof net.
%
For an example, we refer to Figure~\ref{fig:main example}: the de-sequentialization of the sequent proof is the first pre-net of the coalescence sequence.

We will conclude this section by discussing two further essential notions. First, a witness link $\link ab$ on two atomic formulas $a$ and $b$ is an \defn{axiom} link if $a\sigma=\dual b\sigma$. A witness \defn{axiom} linking is one where every link is an axiom link. Observe that the de-sequentialization $[\pi]$ is a witness pre-net with axiom linking $\deseq[\varnothing]\pi AB$.

Second, given a link $\link CD$ in a pre-net $\net\lambda AB$, we expect the domain of $\sigma$ to be exactly the existentially quantified variables in $A$ and $B$ in whose scope $C$ and $D$ occur. 
%That is, any variable $x$ such that there is a subformula $\exists x.X$ in $A$ that in turn has $C$ as a subformula, and likewise for $B$ and $D$. We will now make this precise. 
For a subformula $C$ of $A$, let the \defn{existential variables} $\ex[A]C$ of $C$ in $A$ be the set
\[
	\ex[A]C = \{\,x \mid C < \exists x.X \sub A\,\}~.
\]

% ..................................................

\begin{definition}
A witness pre-net $\net{\lambda_\Sigma}AB$ has \defn{exact coverage} if for every link $\link CD$ in $\lambda_\Sigma$ the domain of $\sigma$ is exactly the existential variables of $C$ and $D$:
\[
	\dom\sigma = \ex[A]C\cup\ex[B]D~.
\]
\end{definition}
%
% ..................................................
%
The witness map $\sigma$ of a link $\link CD$ then has two natural components, $\sigma\res{\ex[A]C}$ and $\sigma\res{\ex[B]D}$, which we will abbreviate by $\sigma_C$ and $\sigma_D$ respectively.

Finally, note that a de-sequentialization $[\pi]$ has exact coverage.

%Note that the property is carefully formulated, using the \emph{exisential cover} rather than simply the free existential variables in a formula, to account for vacuous quantifiers.
% ($\exists x.C$ where $x$ is not free in $C$). The issue becomes even more pertinent for a link $\link CD$ where $C$ occurs in $\exists x.C{\+}C'$ and $x$ is free in $C'$ but not $C$, where we still would like an explicit witness for $x$ in $\sigma$. 


% --------------------------------------------------

\subsection{Correctness and sequentialization}

For sequentialization, the links in a pre-net will be labelled with a sequent proof. An axiom link will carry an axiom, and each coalescence step introduces one proof rule. Formalizing this, a \defn{proof linking} $\lambda_\Sigma^\Pi$ is a witness linking $\lambda_\Sigma$ with a \defn{proof labelling} $\Pi\colon \lambda\to\proofs$ assigning a sequent proof to each link. We will use and define $\lambda_\Sigma^\Pi$ as a set of \defn{proof links} $\link CD^\pi$, where we require that $\prf\pi{C\sigma}{D\sigma}$, i.e.\ that $\pi$ proves the conclusion $\seq{C\sigma}{D\sigma}$. A \defn{labelled pre-net} $\net{\lambda_\Sigma^\Pi}AB$ is a witness pre-net $\net{\lambda_\Sigma}AB$ with a proof labelling $\Pi$ on $\lambda_\Sigma$. If $\lambda_\Sigma$ is an axiom linking, we assign an \defn{initial proof labelling} $\lambda_\Sigma^\star$ as follows.
\[
	\lambda_\Sigma^\star = \{~\link ab^\pi~\mid~\link ab\in\lambda_\Sigma~,~\pi=~\infer{\seq{a\sigma}{b\sigma}}{}~\}
\]
For correctness we may coalesce a pre-net directly, without constructing a proof. 
%
Then to recap, we have accumulated the following further notational conventions.
%
\[
\begin{tabular}{@{}lcll@{}}
	$\pi,\phi,\psi$		& $\in$ & $\proofs$ & \all1 sequent proofs\\
	$\kappa,\lambda$ 	& $\subset$ & $\form\times\form$ 	& linkings (sets of pairs of formulas)\\
	$\rho,\sigma,\tau$	& $\colon$  & $\var\rightharpoonup\terms$ & witness maps\\
	$\Sigma,\Theta$		& $\colon$  & $\lambda\to\var\rightharpoonup\terms$		& witness labellings on a linking $\lambda$ \\
	$\Pi,\Phi,\Psi$		& $\colon$  & $\lambda\to\proofs$		& proof labellings on a linking $\lambda$
\end{tabular}
\]
\begin{definition}
%
\renewcommand{\+}{{+}}
\renewcommand{\*}{{\times}}
%
\defn{Strict sequentialization} ($\scoal$) is the rewrite relation on labelled pre-nets generated by the following rules, that replace one or two links by another in a pre-net $\net{\lambda_\Sigma^\Pi}AB$ (where $B$ has a subformula $D_1\+D_2$, $D_1\*D_2$, $\exists x.D$, and $\forall x.D$ respectively).
%
\begin{align*}
\tag{$\srr+i$}
	\link C{D_i}^\pi
	& \scoal
	\link C{D_1\+D_2}^\psi
	&& 
 	\psi=~\vc{\infer[\Qrr+i]{\seq{C\sigma}{D_1\sigma\,\+\,D_2\sigma}}  {\Prf \pi {C\sigma} {D_i\sigma}}}
\\ \\[-8pt]
\tag{$\srr*$}
	%\{\link C{D_1}^\pi\,,\,\link C{D_2}^\phi\}
	\left.\begin{array}{@{}l@{\,}}\link C{D_1}^\pi \\ \link C{D_2}^\phi\end{array}\right\}
	& \scoal
	\link C{D_1\*D_2}^\psi
	&&
	\psi=~\vc{\infer[\Qrr*]{\seq{C\sigma}{D_1\sigma\,\*\,D_2\sigma}}  {
      \Prf \pi {C\sigma} {D_1\sigma} & \Prf \phi {C\sigma} {D_2\sigma}    
	}}
\\ \\[-8pt]
\tag{$\srr?$}
	\link CD^\pi
	& \scoal
	\link[\sigma\minus x]C{\exists x.D}^\psi
	\quad{\scriptstyle(x\,\in\,\dom\sigma)}
	&&
	\psi=~\vc{\infer[\Qrr?{\sigma(x)}]{\seq{C(\sigma\minus x)}{\exists x.D(\sigma\minus x)}} {\Prf \pi {C\sigma} {D\sigma} }}	
\\ \\[-8pt]
\tag{$\srr!$}
	\link CD^\pi
	& \scoal
	\link C{\forall x.D}^\psi
	\quad{\scriptstyle\phantom{\minus x}(x\,\notin\,\sigma)}
	&&
	\psi=~\vc{\infer[\Qrr!]{\seq{C\sigma}{\forall x.D\sigma}} {\Prf \pi {C\sigma} {D\sigma}}}
\end{align*}
%
\defn{Strict coalescence} is the same relation on witness pre-nets, ignoring proof labels, illustrated in Figure~\ref{fig:coalescence}. A witness pre-net $\net{\lambda_\Sigma}AB$ \defn{strict-coalesces} if it reduces to $\net{\{\link[\varnothing]AB\}}AB$. It \defn{strongly} strict-coalesces if any coalescence path terminates at $\net{\{\link[\varnothing]AB\}}AB$.
\end{definition}
% ..................................................

For an example of coalescence, see Figure~\ref{fig:main example} in the introduction.

% ..................................................
\begin{definition}
An \all1 \defn{witness proof net} or \defn{witness net} is a witness pre-net $\net{\lambda_\Sigma}AB$ with $\lambda_\Sigma$ an axiom linking, that strict-coalesces. It \defn{sequentializes} to a proof $\pi$ if its initial labelling $\net{\lambda_\Sigma^\star}AB$ reduces in $(\scoal)$ to $\net{\{\link[\varnothing]AB^\pi\}}AB$.
\end{definition}
% ..................................................

We conclude this section by establishing that sequentialization and de-sequentialization for witness nets are inverses.

% ..................................................
\begin{theorem}
\label{thm:proof->net->proof}
For any \all1 proof $\pi$, the witness net $[\pi]$ sequentializes to $\pi$. 
\end{theorem}

\begin{proof}
It follows by induction on $\pi$ that if $\lambda_\Sigma=\deseq{\prf\pi AB}{A'}{B'}$ where $A'\sigma=A$ and $B'\sigma=B$, then $\net{\lambda^\star_\Sigma}AB$ reduces in $(\scoal)$ to $\net{\{\link{A'}{B'}^\pi\}}AB$. The statement is the case $\sigma=\varnothing$.
\end{proof}
% ..................................................

% ..................................................
\begin{theorem}
\label{thm:net->proof->net}
If $\net{\lambda_\Sigma}AB$ sequentializes to $\pi$, then $[\pi]$ is $\net{\lambda_\Sigma}AB$. 
\end{theorem}

\begin{proof}
By induction on the sequentialization path 
$\net{\lambda_\Sigma^\star}AB\scoal^*\net{\{\link[\varnothing]AB^\pi\}}AB$
it follows that in every pre-net $\net{\kappa_\Theta^\Phi}AB$ on this path, $\lambda_\Sigma$ is equal to the union over the de-sequentialization of all proof labels $\phi$ in $\Phi$:
\[
	\lambda_\Sigma=\bigcup~\{\,\deseq\phi CD \mid \link CD^\phi \in \kappa_\Theta^\Phi\,\}~.
\]
The statement is then the case $\kappa_\Theta^\Phi=\{\link[\varnothing]AB^\pi\}$.
\end{proof}
% ..................................................

% ..................................................
\begin{figure}
\[
\begin{array}{ccc@{\hspace{50pt}}ccc}
    \vc{\begin{tikzpicture}[net]
    	\formula[y=2]{C}
    	\formula[y=1]{D*{_1}{{\+}}D*{_2}}
    	\Vlink[red,label={$\scriptstyle\sigma~$},l]{1,1}
    \end{tikzpicture}}
&\scoal& 
    \vc{\begin{tikzpicture}[net]
    	\formula[y=2]{C}
    	\formula[y=1]{D*{_1}{{\+}}D*{_2}}
    	\Vlink[red,label={~$\scriptstyle\sigma$}]{1,2}
    \end{tikzpicture}}
&
    \vc{\begin{tikzpicture}[net]
    	\formula[y=2]{C}
    	\formula[y=1]{\exists x.D}
    	\Vlink[red,label={~$\scriptstyle\sigma$}]{1,4}
    \end{tikzpicture}}
&\stackrel{x\,\in\,\dom\sigma}\scoal&
    \vc{\begin{tikzpicture}[net]
    	\formula[y=2]{C}
    	\formula[y=1]{\exists x.D}
    	\Vlink[red,label={~$\scriptstyle{\sigma\minus x}$}]{1,1}
    \end{tikzpicture}}   
\\ \\   
    \vc{\begin{tikzpicture}[net]
    	\formula[y=2]{C}
    	\formula[y=1]{D*{_1}{{\*}}D*{_2}}
    	\Vlink[red,label={$\scriptstyle\sigma~$},l]{[-1]1,1}
    	\Vlink[red,label={~$\scriptstyle\sigma$}]{[1]1,3}
    \end{tikzpicture}}
&\scoal&
	\vc{\begin{tikzpicture}[net]
    	\formula[y=2]{C}
    	\formula[y=1]{D*{_1}{{\*}}D*{_2}}
    	\Vlink[red,label={~$\scriptstyle\sigma$}]{1,2}
    \end{tikzpicture}}   
&
    \vc{\begin{tikzpicture}[net]
    	\formula[y=2]{C}
    	\formula[y=1]{\forall x.D}
    	\Vlink[red,label={~$\scriptstyle\sigma$}]{1,4}
    \end{tikzpicture}}
&\stackrel{x\,\notin\,\sigma}\scoal&
    \vc{\begin{tikzpicture}[net]
    	\formula[y=2]{C}
    	\formula[y=1]{\forall x.D}
    	\Vlink[red,label={~$\scriptstyle\sigma$}]{1,1}
    \end{tikzpicture}}
\end{array}
\]
\caption{Coalescence rules}
\label{fig:coalescence}
\end{figure}
% ..................................................



% --------------------------------------------------

\section{Geometric correctness}
\label{sec:correctness}

A \defn{slice} is the fraction of a proof that depends on a given choice of one branch (or projection) on each product formula $A\*B$. Important to additive proof theory is that many operations can be performed on a per-slice basis, such as normalization, or proof net correctness. We will here use slices for the latter purpose.

As in the propositional case \cite{Hughes-vanGlabbeek-2005}, we define a \defn{slice} of a sequent $\seq AB$ as a set of potential links, of which exactly one must be realized in a proof net $\net{\lambda_\Sigma}AB$. We extend the propositional criterion in two ways:
\begin{itemize}
	\item
When defining slices, we interpret an existential quantification $\exists x.A$ as a sum over all witnesses $t_i$ to $x$ that occur in the pre-net,  $A[t_1/x]\+\dots\+A[t_n/x]$. This captures the non-permutability of a product rule over distinct existential instantiations, as below left.

	\item
We define a \defn{dependency} relation between a universal quantification $\forall x.A$ and an instantiation $B[t/y]$ of $\exists x.B$ where the \defn{eigenvariable} $x$ occurs free in $t$ (a standard approach to first-order quantification \cite{Miller-1987,Girard-1996,Heijltjes-2010,Mimram-2011}). For each link, we require the dependency to be \emph{acyclic}, which amounts to \emph{slice-wise} first-order correctness. It captures the non-permutability of universal and existential sequent rules due to the \emph{eigenvariable condition} of the former, as below right.
\end{itemize}

\[
\vc{
	\infer[\Qrr*]{\Seq{\exists x.A}{B\*C}}{
	 \infer[\Qrr?s]{\Seq{\exists x.A}B}{\Seq{A[s/x]}B}
	 &
	 \infer[\Qrr?t]{\Seq{\exists x.A}C}{\Seq{A[t/x]}C}
   }}
\qquad\qquad
\vc{
	\infer[\Qrr!]{\Seq{\forall x.A}{\exists y.B}}{
	 \infer[\Qrr?t]{\Seq A{\exists x.B}}{\Seq AB[t/y]}
   }}
   \quad\text{where }x\in\fv(t)
\]

For a witness linking $\lambda_\Sigma$ and variable $x$, write $\Sigma(x)\subseteq\terms$ for the \defn{witness set} of $x$, which collects the terms assigned to $x$ by the witness maps in $\Sigma$ for every link in $\lambda$:
\[
	\Sigma(x) = \{~\sigma(x)~\mid~\link CD\in\lambda_\Sigma~\}~.
\]

% ..................................................
\begin{definition}[Slice]
Given a witness linking $\lambda_\Sigma$, a \defn{slice} $S$ of a formula $A$ and a witness map $\sigma$ is a set of pairs $(A',\sigma')$, where $A'\leq A$ and $\sigma'\supseteq\sigma$, given by $S=\{(A,\sigma)\}\cup S'$ where:
%
\begin{itemize}
	\item
\makebox[65pt][l]{If $A = a$}			then $S'=\varnothing$.
	\item
\makebox[65pt][l]{If $A = B\+C$} 		then $S'=S_B\uplus S_C$ with $S_B$ a slice of $B$ and $\sigma$, and $S_C$ one of $C$ and $\sigma$.
	\item
\makebox[65pt][l]{If $A = B\*C$}		then $S'$ is a slice of $B$ and $\sigma$ or a slice of $C$ and $\sigma$.
	\item
\makebox[65pt][l]{If $A = \exists x.B$}	then $S'=\uplus_{t\,\in\,\Sigma(x)}S_t$ where each $S_t$ is a slice of $B$ and $\sigma[t/x]$.
	\item
\makebox[65pt][l]{If $A = \forall x.B$}	then $S'$ is a slice of $B$ and $\sigma$.
\end{itemize}
%
\noindent
A \defn{slice} of a sequent $\seq AB$ is a set of links
\[
	\{~\link[\sigma\cup\tau]CD~\mid~(C,\sigma)\in S_A~,~(D,\tau)\in S_B~\}
\]
where $S_A$ is a slice of $A$ and $\varnothing$, and $S_B$ a slice of $B$ and $\varnothing$.  A \defn{slice} of a witness pre-net $\net{\lambda_\Sigma} AB$ is the intersection $\lambda_\Sigma\cap S$ of $\lambda_\Sigma$ with a slice $S$ of $\seq AB$.
\end{definition}
% ..................................................

As in the propositional case, for correctness we will require that each slice is a singleton. We will further define a \defn{dependency} condition to ensure that the order in which quantifiers are instantiated is sound, corresponding to the \emph{eigenvariable condition} on the $\qrr!$-rule of sequent calculus. For simplicity, we define the condition on individual links rather than slices.

%As in the propositional case, for correctness we will require that each slice is a singleton. In addition, within each slice we need to ensure that the order in which quantifiers are instantiated is sound. That is, we need to have an analogue of the \emph{eigenvariable condition} on the $\qrr!$-rule of sequent calculus, that the quantified variable $x$ of the formula $\forall x.A$ does not occur free in the context. We will define a \defn{dependency} condition to this effect. For simplicity we will define the dependency condition on individual links rather than on slices.

\begin{definition}
In a pre-net $\net{\lambda_\Sigma}AB$, let the \defn{column} of a link $\link CD$ be the set of pairs 
\[
	\{~(X,\sigma\res{\ex[A]X})~\mid~C\sub X\sub A~\}\cup\{~(Y,\sigma\res{\ex[B]Y})~\mid~D\sub Y\sub B~\}~,
\]
with a \defn{dependency} order $(\dep)$: $(X,\rho)\dep(Y,\tau)$ if $X\leq Y$ or $Y$ occurs as $\forall x.Y$ and $x\in\rho$.
\end{definition}

% ..................................................
\begin{definition}
A witness pre-net is \defn{correct} if
\begin{itemize}
\item it has exact coverage,
\item it is \defn{slice-correct}: every slice is a singleton, and
\item it is \defn{dependency-correct}: every column is a partial order (i.e.\ is acyclic/antisymmetric).
\end{itemize}
\end{definition}
% ..................................................

In the remainder of this section we will establish that the two correctness conditions, by coalescence and by slicing, are equivalent. The geometric condition further gives \emph{strong} coalescence: since it is preserved, no coalescence step yields an incorrect, and thus non-coalescing, pre-net. Finally, canonicity follows by inspecting the critical pairs of sequentialization.

% ..................................................
\newcounter{lemcorrect}
\setcounter{lemcorrect}{\thetheorem}
\begin{lemma}
\label{lem:correct}
Strict coalescence preserves and reflects correctness.
\end{lemma}

\begin{proof}
See the appendix.
\end{proof}

% ..................................................

\newcounter{lemcoalescenceapplies}
\setcounter{lemcoalescenceapplies}{\thetheorem}
\begin{lemma}
\label{lem:coalescence applies}
To a correct witness pre-net $\net{\lambda_\Sigma}AB$ a coalescence step applies, unless it is fully coalesced already, $\lambda_\Sigma=\{\link[\varnothing]AB\}$.
\end{lemma}

\begin{proof}
See the appendix.
\end{proof}

% ..................................................

\begin{theorem}
\label{thm:correctness}
A witness pre-net is correct if and only if it strict-coalesces. 
\end{theorem}

\begin{proof}
From right to left, we proceed by induction on the coalescence path from $\net{\lambda_\Sigma}AB$ to $\net{\{\link[\varnothing]AB\}}AB$, with the end result as the base case. It is slice-correct: every slice of $\seq AB$ contains $\link[\varnothing]AB$, so every slice of $\net{\{\link[\varnothing]AB\}}AB$ is the singleton $\{\link[\varnothing]AB\}$. It is also dependency-correct: the column of $\link[\varnothing]AB$ is the set $\{(A,\varnothing),(B,\varnothing)\}$, where $A$ and $B$ are unrelated in $(\dep)$. For the inductive step, by Lemma~\ref{lem:correct} coalescence reflects correctness, so that any pre-net along the coalescence path is correct, in particular $\net{\lambda_\Sigma}AB$.

From left to right, let $\net{\lambda_\Sigma}AB$ be correct. By Lemma~\ref{lem:coalescence applies} either the net has coalesced, or a coalescence step applies. By Lemma~\ref{lem:correct} the result of this coalescence step is again correct. It follows that the pre-net $\net{\lambda_\Sigma}AB$ strict-coalesces.
\end{proof}

% ..................................................

\begin{corollary}
A correct witness pre-net with axiom linking is a witness proof net.
\end{corollary}

% ..................................................

\begin{corollary}
\label{cor:strong coalescence}
A correct witness pre-net strongly strict-coalesces.
\end{corollary}

\begin{proof}
By Theorem~\ref{thm:correctness} a correct pre-net coalesces, and by Lemma~\ref{lem:correct} any coalescence step preserves correctness.
\end{proof}

% ..................................................

\begin{theorem}
Proof nets are canonical: $[\pi]=[\phi]$ if and only if $\pi\sim\phi$.
\end{theorem}

\begin{proof}
From left to right is by inspection of the critical pairs of sequentialization ($\scoal$). From right to left is by inspection of the rule permutations in Figure~\ref{fig:permutations} in the appendix.
\end{proof}


% --------------------------------------------------

\section{Composition}

\newlength\seriously
\setlength\seriously\parindent
\parbox[b]{.6\columnwidth}{
We will describe the composition of two witness nets by a global operation. It consists of the relational composition of both linkings, as in the propositional case, where for each pair of links that are being connected, their witness maps are composed. As links correspond to slices, the operation is effectively first-order composition~\cite{Mimram-2011} applied slice-wise.

\setlength\parindent\seriously
\indent
Cut-elimination rules for \all1 are given in Figure~\ref{fig:cut-elim}; the needed permutations are in the appendix, Figure~\ref{fig:cut-permutations}.
}
\quad
\begin{tikzpicture}[net]
	\formula[right,y=1.2]{*{\exists v.}{\dual P}}
	\formula[right,y=0]{*{\forall x.\exists y.\forall z.\,}P}
	\Vlink[red,l,label={$\scriptstyle[v/z,f(x)/y]$~~}]{1,1}
	\formula[right,y=-.5]{*{\exists x.\forall y.\exists z.\,}{\dual P}}
	\formula[right,y=-1.7]{P}
	\Vlink[blue,l,label={$\scriptstyle[t/x,g(y)/z]$~~}]{1,1}
	\path(0,-2)--(1,-2);
\end{tikzpicture}
\begin{tikzpicture}[net]
	\formula[right,y=1.2]{*{\exists v.}{\dual P}}
	\formula[right,y=-1.7]{P}
	\Vlink[violet,l,label={$\Rightarrow\quad\scriptstyle[g(f(t))/v]$~~}]{1,1}
	\path(0,-2)--(1,-2);
\end{tikzpicture}

We use the example on the right to illustrate the composition of links. To eliminate the central cut, on $\forall x.\exists y.\forall z.P$ and $\exists x.\forall y.\exists z.\dual P$, the explicit substitutions for both formulas must be effectuated. An inductive procedure, as in sequent calculus, could apply them from outside in: first $[t/x]$, then $[f(t)/y]$ (previously $[f(x)/y]$), then $[g(f(t))/z]$ (previously $[g(y)/z]$). 

For a direct definition, to compose two links $\link ab$ and $\link[\tau]{\dual b}c$, the substitutions into the cut-formula $\sigma_b$ and $\tau\subdual b$ must be applied as often as needed, up to the depth of quantifiers above $b$, to the terms in the range of the remaining substitutions, $\sigma_a$ and $\tau_c$. To formalize this, we will use the following notions:
%
\begin{itemize}
	\item
The \defn{domain-preserving composition} of two witness maps $\sigma\dcom\tau$ is the map $(\sigma\tau)\res{\dom\sigma}$.
	\item
The \defn{least fixed point} $\fix\sigma$ of a witness map $\sigma$ is the least map $\rho$ satisfying $\rho = \rho\sigma$. 
\end{itemize}
%
The latter is the shortest sequence $\fix\sigma=\sigma\sigma\dots\sigma$ such that no variable is both in the domain and range of $\fix\sigma$. This is not necessarily finite; in our composition operations, finiteness is ensured by the correctness conditions on proof nets.

% ..................................................
\begin{definition}
The \defn{composition} $\link AB^\pi\comp\link[\tau]{\dual B}C^\phi$ of two proof links is $\link[\rho]AC^\psi$ where
\[
	\rho = \sigma_A\tau_C\dcom\fix{\sigma_B\tau\subdual B}
\qquad\text{and}\qquad
	\psi = ~\vcenter{
	\infer[\!\!\scriptstyle{\mathrm{cut}}]{\seq {A\rho}{C\rho}}{
	 \left(\vcenter{\deduce{\seq{A\sigma}{B\sigma}\vphantom{\dual B}}{\vphantom\phi\pi}}\right)\fix{\sigma_B\tau\subdual B}
	&\left(\vcenter{\deduce{\seq{\dual B\tau}{C\tau}}{\phi}}\right)\fix{\sigma_B\tau\subdual B}
	}}~.
\]
The \defn{composition} $\lambda_\Sigma^\Pi\comp\kappa_\Theta^\Phi$ of two linkings is the linking
\[
	\{~\link XY^\pi\comp\link[\tau]{\dual Y}Z^\phi
		~\mid~
		\link XY^\pi\in{\lambda_\Sigma^\Pi}
		~,~
		\link[\tau]{\dual Y}Z^\phi\in\kappa_\Theta^\Phi~\}
\]
The \defn{composition} $(\net{\lambda_\Sigma^\Pi}AB)\comp(\net{\kappa_\Theta^\Phi}{\dual B}C)$ of two pre-nets is the pre-net $\net{(\lambda_\Sigma^\Pi\comp\kappa_\Theta^\Phi)}AC$.
These compositions may omit proof annotations and witness annotations.
\end{definition}
% ..................................................


% ..................................................
\begin{figure}
\[
\begin{array}{ccc}
	\vc{
	\infer[\!\!\scriptstyle{\mathrm{cut}}]{\Seq AC}{
	 \infer[\!\!\scriptstyle{\qrr*}]{\Seq A{B_1\*B_2}\vphantom{\dual B}}{\Prf{\pi_1}A{B_1} & \Prf{\pi_2}A{B_2}}
	 &&
	 \infer[\!\!\scriptstyle{\qrr+i}]{\Seq {\dual B_1\+\dual B_2}C}{\Prf\phi{B_i}C}
	}}
	&\Rightarrow&
	\vc{\infer[\!\!\scriptstyle{\mathrm{cut}}]{\Seq AC}{\Prf{\pi_i}A{B_i} && \Prf\phi{\dual B_i}C}}
\\ \\
	\vc{
	\infer[\!\!\scriptstyle{\mathrm{cut}}]{\Seq AC}{
	 \infer[\!\!\scriptstyle{\qrr?t}]{\Seq A{\exists x.B}}{\Prf\pi A{B[t/x]}}
	 &&
	 \infer[\!\!\scriptstyle{\qrr!}]{\Seq {\forall x.\dual B}C}{\Prf\phi{\dual B}C}
	}}
	&\Rightarrow&
	\vc{\infer[\!\!\scriptstyle{\mathrm{cut}}]{\Seq AC}{\Prf\pi A{B[t/x]} && \Prf{\phi[t/x]}{\dual B[t/x]}C}}
\end{array}
\]
\caption{\all1 cut-elimination steps}
\label{fig:cut-elim}
\end{figure}
% ..................................................

The composition of two links is strongly related to composition of strategies in game semantics. There, two strategies on $\seq AB$ and $\seq{\dual B}C$ are composed by \emph{interaction} on the interface of $B$ and $\dual B$, and subsequently \emph{hiding} that interaction. 

In the following we will demonstrate that composition gives the desired result: if a net $L$ sequentializes to $\pi$ and $R$ to $\phi$, then $L\comp R$ sequentializes to a normal form of the composition of $\pi$ and $\phi$ with a cut. To this end we will explore how composition and sequentialization interact. We will consider the critical pairs of sequentialization $(\scoal)$ with composition $(\Rightarrow)$ given in Figures~\ref{fig:crit1}--\ref{fig:crit3}, and demonstrate how they are resolved.

\newcommand\yA{2.7}
\newcommand\yB{1.65}
\newcommand\yC{1.05}

% ..................................................
\begin{figure}[p]
\[
\begin{array}{ccc}
    \vc{
    \begin{tikzpicture}[net]
        \formula[y=\yA]{A}
        \formula[y=\yB]{B*{_1}\*B*{_2}}
        \Vlink[red,label={$\scriptstyle{\pi,\sigma\vphantom)~}$},l]{[-2]1,1}
        \Vlink[red,label={$\scriptstyle{~\pi'\!,\sigma\vphantom)}$},r]{[2]1,3}
        \formula[y=\yC]{{\dual B}*{_1}\+{\dual B}*{_2}}
        \formula[y=0]{C}
        \Vlink[red,label={$\scriptstyle{\phi,\tau\vphantom)~}$},l]{1,1}
    \end{tikzpicture}}
    & 
    \begin{array}{c}\scoal\\[36pt]\scoal\end{array}
    &
    \vc{
    \begin{tikzpicture}[net]
        \formula[y=\yA]{A}
        \formula[y=\yB]{B*{_1}\*B*{_2}}
        \Vlink[red,label={$\scriptstyle{\pi''\!\!,\sigma\vphantom)~}$},l]{1,2}
        \formula[y=\yC]{{\dual B}*{_1}\+{\dual B}*{_2}}
        \formula[y=0]{C}
        \Vlink[red,label={$\scriptstyle{\phi'\!,\tau\vphantom)~}$},l]{2,1}
    \end{tikzpicture}}
\\\\[-7pt] \Downarrow && \Downarrow \\\\[-7pt]
    \vc{
    \begin{tikzpicture}[net]
        \formula[y=\yC]{A}
        \formula[y=0]{C}
        \Vlink[red,label={\makebox[0pt][r]{$\scriptstyle{\psi,\rho\vphantom)~}$}},l]{1,1}
    \end{tikzpicture}}
    &&
    \vc{
    \begin{tikzpicture}[net]
        \formula[y=\yC]{A}
        \formula[y=0]{C}
        \Vlink[red,label={\makebox[0pt][r]{$\scriptstyle{\psi',\rho'\vphantom)~}$}},l]{1,1}
    \end{tikzpicture}}
\end{array}
\qquad
\begin{array}{l@{\,}l}
	\rho  &=~ \sigma_A\tau_C\dcom\fix{\sigma_B\tau\subdual B} 
\\ \\
	\rho' &=~ \sigma_A\tau_C\dcom\fix{\sigma_{B_1}\tau\subdual{B_1}}
\\ \\
	\psi &=~\vc{
	\infer[\!\!\scriptstyle{\mathrm{cut}}]{\seq{A\rho}{C\rho}}{
	 \left(\vc{
	  \deduce{\seq{A\sigma}{B_1\sigma}\vphantom{\dual B_1}}{\pi\vphantom)}
	 }\right)\fix{\sigma_{B_1}\tau\subdual{B_1}}
	 &
	 \left(\vc{
	  \deduce{\seq{\dual B_1\tau}{C\tau}}{\phi\vphantom)}
	 }\right)\makebox[0pt][l]{$\fix{\sigma_{B_1}\tau\subdual{B_1}}$}
	}}
\\ \\ \\
	\psi' &= ~
	\vc{\scalebox{0.75}{$
	\infer[\!\!\scriptstyle{\mathrm{cut}}]{\seq{A\rho'}{C\rho'}}{
	 \left(\vc{\infer{\seq{A\sigma}{B_1\sigma\*B_2\sigma}\vphantom{\dual B}}{
	  \deduce{\seq{A\sigma}{B_1\sigma}\vphantom{\dual B}}{\pi\vphantom)}
	  &
	  \deduce{\seq{A\sigma}{B_2\sigma\vphantom{\dual B}}}{\pi'\vphantom)}
	 }}\right)\fix{\sigma_B\tau\subdual B}
	 &
	 \left(\vc{\infer{\seq{\dual B_1\tau\+\dual B_2\tau}{C\tau}}{
	  \deduce{\seq{\dual B_1\tau}{C\tau}}{\phi\vphantom)}
	 }}\right)\makebox[0pt][l]{$\fix{\sigma_B\tau\subdual B}$}
	}$}\kern5pt}
\end{array}
\]
\caption{The critical pair $\seq A{B_1{\*}B_2}$ ; $\seq{\dual B_1{\+}\dual B_2}C$}
\label{fig:crit1}
\end{figure}
% ..................................................
\begin{figure}[p]
\[
\begin{array}{ccc}
    \vc{
    \begin{tikzpicture}[net]
        \formula[y=\yA]{A}
        \formula[y=\yB]{\exists x.B}
        \Vlink[red,label={\makebox[0pt][l]{~$\scriptstyle{\pi,\sigma\vphantom)}$}},r]{1,4}
        \formula[y=\yC]{\forall x.{\dual B}}
        \formula[y=0]{C}
        \Vlink[red,label={\makebox[0pt][l]{~$\scriptstyle{\phi,\tau\vphantom)}$}},r]{4,1}
    \end{tikzpicture}}
    & 
    \begin{array}{c}\scoal\\[36pt]\stackrel{x\,\notin\,\tau\vphantom)}\scoal\end{array}
    &
    \vc{
    \begin{tikzpicture}[net]
        \formula[y=\yA]{A}
        \formula[y=\yB]{\exists x.B}
        \Vlink[red,label={\makebox[0pt][l]{~$\scriptstyle{\pi',\sigma\minus x\vphantom)}$}},r]{1,1}
        \formula[y=\yC]{\forall x.{\dual B}}
        \formula[y=0]{C}
        \Vlink[red,label={\makebox[0pt][l]{~$\scriptstyle{\phi',\tau\vphantom)}$}},r]{1,1}
    \end{tikzpicture}}
\\\\[-7pt] \Downarrow && \Downarrow \\\\[-7pt]
    \vc{ 
    \begin{tikzpicture}[net]
        \formula[y=\yC]{A}
        \formula[y=0]{C}
        \Vlink[red,label={\makebox[0pt][l]{~$\scriptstyle{\psi,\rho\vphantom)}$}},r]{1,1}
    \end{tikzpicture}}
    &&
    \vc{
    \begin{tikzpicture}[net]
        \formula[y=\yC]{A}
        \formula[y=0]{C}
        \Vlink[red,label={\makebox[0pt][l]{~$\scriptstyle{\psi',\rho'\vphantom)}$}},r]{1,1}
    \end{tikzpicture}}
\end{array}
%
\qquad\quad
%
\begin{array}{l@{\,}l}
	\rho  &=~ \sigma_A\tau_C\dcom\fix{\sigma_B\tau\subdual B}
\\ \\
	\rho' &=~ \sigma_A\tau_C\dcom\fix{(\sigma_B\minus x)\tau\subdual B}
\\ \\
	\psi  &=~
	\vc{
	\infer[\!\!\scriptstyle{\mathrm{cut}}]{\seq{A\rho}{C\rho}}{
	 \left(\vc{
	  \deduce{\seq{A\sigma}{B\sigma}\vphantom{\dual B}}{\pi\vphantom)}
	 }\right)\fix{\sigma_B\tau\subdual B}
	 &
	 \left(\vc{
	  \deduce{\seq{\dual B\tau}{C\tau}}{\phi\vphantom)}
	 }\right)\makebox[0pt][l]{$\fix{\sigma_B\tau\subdual B}$}
	}}
\\ \\ \\
	\psi' &=~ 
	\vc{\scalebox{0.8}{$
	\infer[\!\!\scriptstyle{\mathrm{cut}}]{\seq{A\rho'}{C\rho'}}{
	 \left(\vc{\infer{\seq{A\sigma}{\exists x.B\sigma\minus x}\vphantom{\dual B}}{
	  \deduce{\seq{A\sigma}{B\sigma}\vphantom{\dual B}}{\pi\vphantom)}
	 }}\right)\fix{(\sigma_B\minus x)\tau\subdual B}
	 &
	 \left(\vc{\infer{\seq{\forall x.\dual B\tau}{C\tau}}{
	  \deduce{\seq{\dual B\tau}{C\tau}}{\phi\vphantom)}
	 }}\right)\makebox[0pt][l]{$\fix{(\sigma_B\minus x)\tau\subdual B}$}
	}$}}
	\rule{25pt}{0pt}
\end{array}
\]
\caption{The critical pair $\seq A{\exists x.B}$ ; $\seq{\forall x.\dual B}C$}
\label{fig:crit2}
\end{figure}
% ..................................................
\begin{figure}[p]
\[
\begin{array}{ccc}
    \vc{
    \begin{tikzpicture}[net]
        \formula[y=\yA]{A}
        \formula[y=\yB]{B}
        \Vlink[red,label={\makebox[0pt][l]{~$\scriptstyle{\pi,\sigma\vphantom)}$}},r]{1,1}
        \formula[y=\yC]{{\dual B}}
        \formula[y=0]{\exists x.C}
        \Vlink[red,label={\makebox[0pt][r]{~$\scriptstyle{\phi,\tau\vphantom)}$~}},l]{1,4}
    \end{tikzpicture}}
    & 
    \begin{array}{c}\\[36pt]\scoal\end{array}
    &
    \vc{
    \begin{tikzpicture}[net]
        \formula[y=\yA]{A}
        \formula[y=\yB]{B}
        \Vlink[red,label={\makebox[0pt][l]{~$\scriptstyle{\pi,\sigma\vphantom)}$}},r]{1,1}
        \formula[y=\yC]{{\dual B}}
        \formula[y=0]{\exists x.C}
        \Vlink[red,label={\makebox[0pt][l]{~$\scriptstyle{\phi',\tau\minus x\vphantom)}$}},r]{1,1}
    \end{tikzpicture}}
\\\\[-7pt] \Downarrow && \Downarrow \\\\[-7pt]
    \vc{ 
    \begin{tikzpicture}[net]
        \formula[y=\yC]{A}
        \formula[y=0]{\exists x.C}
        \Vlink[red,label={\makebox[0pt][r]{~$\scriptstyle{\psi,\rho\vphantom)}$~}},l]{1,4}
    \end{tikzpicture}}
    &&
    \vc{
    \begin{tikzpicture}[net]
        \formula[y=\yC]{A}
        \formula[y=0]{\exists x.C}
        \Vlink[red,label={\makebox[0pt][l]{~$\scriptstyle{\psi',\rho'\vphantom)}$}},r]{1,1}
    \end{tikzpicture}}
\end{array}
%
\qquad\quad
%
\begin{array}{l@{\,}l}
	\rho  &=~ \sigma_A\tau_C\dcom\fix{\sigma_B\tau\subdual B}
\\ \\
	\rho' &=~ \sigma_A(\tau_C\minus x)\dcom\fix{\sigma_B\tau\subdual B}
\\ \\
	\psi &=~
	\vc{
	\infer[\!\!\scriptstyle{\mathrm{cut}}]{\seq{A\rho}{C\rho}}{
	 \left(\vc{
	  \Prf\pi{A\sigma}{B\sigma}
	 }\right)\fix{\sigma_B\tau\subdual B}
	 &
	 \left(\vc{
	  \Prf\phi{\dual B\tau}{C\tau}
	 }\right)\fix{\sigma_B\tau\subdual B}
	}}
\\ \\
	\psi' &=~
	\vc{
	\infer[\!\!\scriptstyle{\mathrm{cut}}]{\seq{A\rho'}{C\rho'}}{
	 \left(\vc{
	  \Prf\pi{A\sigma}{B\sigma}
	 }\right)\fix{\sigma_B\tau\subdual B}
	 &
	 \left(\vc{
	  \infer{\Seq{\dual B\tau}{\exists x.C(\tau\minus x)}}{\Prf\phi{\dual B\tau}{C\tau}}
	 }\right)\fix{\sigma_B\tau\subdual B}
	}}
\end{array}
\]
\caption{The critical pair $\seq AB$ ; $\seq{\dual B}{\exists x.C}$}
\label{fig:crit3}
\end{figure}
% ..................................................

\begin{itemize}
	\item 
$\seq A{B_1{\*}B_2}$ ; $\seq{\dual B_1{\+}\dual B_2}C$\quad (Figure~\ref{fig:crit1})
\\
Since the existential covers of $B$ and $B_1$ are the same, $\sigma_B\tau\subdual B = \sigma_{B_1}\tau\subdual{B_1}$ and $\rho=\rho'$. It then follows that $\psi'$ cut-eliminates in one step to $\psi$.

	\item
$\seq A{\exists x.B}$ ; $\seq{\forall x.\dual B}C$\quad (Figure~\ref{fig:crit2})
\\	
Since $x$ is not free in the range of $\tau$, and (by the freshness convention) nor in the range of $\sigma$, we have that $\fix{\sigma_B\tau\subdual B}$ is $\fix[1pt]{(\sigma_B\minus x)\tau\subdual B}$ plus the substitution $[\sigma(x)/x]$. Then $\rho=\rho'$ (as $x$ does not occur in the range of $\sigma_A\tau_C$) and $\psi'$ reduces to $\psi$ in a single cut-elimination step.

	\item
$\seq AB$ ; $\seq{\dual B}{\exists x.C}$\quad (Figure~\ref{fig:crit3})
\\
Observe that since $x$ occurs in $C$ but not $B$, it is not in the domain of $\tau_B$, so that $\tau_B\minus x$ is just $\tau_B$. Then $\rho'=\rho\minus x$, and the diagram is closed by a sequentialization step (from left to right) that extends $\psi$ with an existential introduction rule, to a proof equivalent to $\psi'$:
\[
	\infer[\!\!\scriptstyle{\qrr?{\rho(x)}}]{\Seq{A\rho'}{\exists x.C\rho'}}{
	 \Prf\psi{A\rho}{C\rho}
	}
\]
\end{itemize}
There are three further critical pairs, for a proof net on $\seq AB$ composed with one on $\seq {\dual B}{C_1{\+}C_2}$, one on $\seq {\dual B}{C_1{\*}C_2}$, and one on $\seq {\dual B}{\forall x.C}$. These converge as the one above.

Resolving these critical pairs gives the soundness of the composition operation, per the following theorem. We abbreviate a cut on proofs $\prf\pi AB$ and $\prf\phi{\dual B}C$ by $\pi\comp\phi$.


% ..................................................
\newcounter{thmcomposition}
\setcounter{thmcomposition}{\thetheorem}
\begin{theorem}
\label{thm:composition}
If proof nets $\net{\lambda_\Sigma}AB$ and $\net{\kappa_\Theta}{\dual B}C$ sequentialize to $\pi$ and $\phi$ respectively, then their composition $(\net{\lambda_\Sigma}AB)\comp(\net{\kappa_\Theta}{\dual B}C)$ sequentializes to a normal form $\psi$ of $\pi\comp\phi$.
%the proof
%\[
%	\infer[\Qrr.]{\seq AC}{
%	 \Prf\pi AB & \Prf\phi{\dual B}C
%	}
%\]
\end{theorem}

\begin{proof}
See the appendix.
\end{proof}

% --------------------------------------------------

\section{Unification nets}

In this final section we explore a second notion of \all1 proof net: \defn{unification nets} omit any witness information, which is then reconstructed by coalescence. This yields a natural notion of \emph{most general} proof net, where every other proof net is obtained by introducing more witness information. Conversely, every witness net has an underlying unification net, that sequentializes to a \emph{most general} proof.

We consider a proof $\prf\pi AB$ \defn{more general} than $\prf{\pi'}AB$, written $\pi\leq\pi'$, if there is a substitution map $\rho$ such that $\pi\rho=\pi'$. Unlike for proof nets, this notion is not so natural for sequent proofs: in the permutation of existential and product rules below, from left to right $u$ must be generated as the least term more general than $s$ and $t$; from right to left, $s$ and $t$ cannot be reconstructed from $u$, and must be retrieved from their respective subproofs.
\[
	\vc{
	 \infer[\Qrr*]{\seq{A\*B}{\exists x.C}}{
	  \infer[\Qrr?s]{\seq A{\exists x.C}}{\seq AC}
	  &
	  \infer[\Qrr?t]{\seq B{\exists x.C}}{\seq BC}
	}}
	\quad\sim\quad
	\vc{
	 \infer[\Qrr?u]{\seq{A\*B}C}{
	  \infer[\Qrr*]{\seq{A\*B}{\exists x.C}}{
	   \seq AC
	   &
	   \seq BC
	}}}
\]


To reconstruct witnesses by unification, we define the following operations.
\\
\makebox[28pt][l]{$\sigma\gen\tau$}:\quad
	A witness map $\sigma$ is \defn{more general} than $\tau$ if there is a map $\rho$ such that $\sigma\rho=\tau$.
\\
\makebox[28pt][l]{$\sigma\coh\tau$}:\quad
	Two witness maps $\sigma$ and $\tau$ are \defn{coherent} if there is a map $\rho$ such that $\sigma\rho=\tau\rho$.
\\
\makebox[28pt][l]{$\sigma\join\tau$}:\quad
	The \defn{join} of coherent witness maps is the least map $\rho$ such that $\sigma\gen\rho$ and $\tau\gen\rho$.



%\[
%\begin{tabular}{@{}p{.08\textwidth}@{}p{.92\textwidth}@{}}
%	$\sigma\gen\tau$
%&
%	A witness map $\sigma$ is \defn{more general} than $\tau$,  
%	$\sigma\gen\tau$, if there is a map $\rho$ such that 
%	$\sigma\rho=\tau$.
%\\%[4pt]	
%	$\sigma\coh\tau$
%&
%	Two witness maps $\sigma$ and $\tau$ are \defn{coherent}, $\sigma\coh\tau$,
%	if there is a map $\rho$ such that $\sigma\rho=\tau\rho$.
%\\%[4pt]
%	$\sigma\join\tau$
%&
%	The \defn{join} $\sigma\join\tau$ of two coherent witness maps is the least map $\rho$ such that $\sigma\gen\rho$ and $\tau\gen\rho$.
%\end{tabular}%
%\]

%
A link $\link[]ab$ on two atomic formulas is an \defn{axiom} link if there exists a witness map $\sigma$ such that $a\sigma=\dual b\sigma$. To an axiom link $\link[]ab$ over $\seq AB$ we assign an \defn{initial witness map}, which is the least witness map $\sigma$ over the domain $\ex[\!A]a\cup\ex[\!B]b$ such that $a\sigma=\dual b\sigma$. In other words, $\sigma$ is the most general unifier of $a$ and $\dual b$, over the given domain, written $\init ab$. For an axiom linking $\lambda$ over $\seq AB$ the \defn{initial witness pre-net} $\net{\lambda_\star}AB$ is given by
\[
	\lambda_\star = \{~\link ab~\mid~\link[]ab\in\lambda~,~\sigma=\init ab~\}~.
\]
Observe that the initial witness pre-net satisfies the \emph{exact coverage} condition.

% ..................................................
\begin{definition}
%
\renewcommand{\+}{{+}}
\renewcommand{\*}{{\times}}
%
\defn{Unifying sequentialization} ($\ucoal$) is the rewrite relation on labelled pre-nets generated by the rules
$(\urr+i)$, $(\urr?)$, $(\urr!)$, which are respectively as $(\srr+i)$, $(\srr?)$, and $(\srr!)$, and the rule
\begin{equation}
\tag{$\urr*$}
	\left.\begin{array}{@{}l@{\,}}\link C{D_1}^\pi \\ \link[\tau]C{D_2}^\phi\end{array}\right\}
	\ucoal
	\link[\sigma\join\tau\,=\,\sigma\rho\,=\,\tau\rho]C{D_1\*D_2}^\psi
	\quad~
	{\scriptstyle(\sigma\coh\tau)}
	\quad~
	\psi=~
  \vc{\scalebox{0.9}{$
   \infer  {\seq{A(\sigma{\join}\tau)}{B\sigma\rho\,\*\,C\tau\rho}}  {
    \left( \vc{\Prf \pi {A\sigma} {B\sigma}} \right)\rho
    &
    \left( \vc{\Prf \phi {A\tau} {C\tau}} \right)\rho   
  }$}}
\end{equation} 
%
\defn{Unifying coalescence} is the relation ($\ucoal$) on witness pre-nets, ignoring proof labels. A witness pre-net $\net{\lambda_\Sigma}AB$ \defn{unifying-coalesces} if it reduces to $\net{\{\link[\varnothing]AB\}}AB$ and \defn{strongly} unifying-coalesces if any coalescence path terminates at $\net{\{\link[\varnothing]AB\}}AB$.
\end{definition}
% ..................................................

% ..................................................
\begin{definition}
An \all1 \defn{unification proof net} or \defn{unification net} is a pre-net $\net\lambda AB$ with axiom linking $\lambda$ such that the initial witness pre-net $\net{\lambda_\star}AB$ unifying-coalesces. It \defn{sequentializes} to $\pi$ if $\net{\lambda^\star_\star}AB$ reduces in $(\ucoal)$ to $\net{\{\link[\varnothing]AB^\pi\}}AB$.
\end{definition}
% ..................................................

In the above definition, note that $\lambda_\star^\star=(\lambda_\star)^*$ is the initial proof labelling of $\lambda_\star$, which assigns an axiom rule to each axiom link. For a minimal example, see Figure~\ref{fig:unification example} in the introduction. Observe also that unifying coalescence includes strict coalescence, $(\scoal)\subseteq(\ucoal)$. The following two lemmata relate sequentialization for witness nets and unification nets.

% ..................................................
\begin{lemma}
\label{lem:ucoal generality}
In ($\ucoal$), if $\net{\lambda_\Sigma}AB$ sequentializes to $\pi$ then $\net{\lambda_\star}AB$ sequentializes to $\pi'\leq\pi$.
\end{lemma}

\begin{proof}
The sequentialization path
$
	\net{\lambda_\Sigma^\star}AB = 
	L_1 \ucoal L_2 \ucoal \dots \ucoal L_n = 
	\net{\link[\varnothing]AB^\pi}AB
$
has a corresponding path
$
	\net{\lambda_\star^\star}AB = 
	R_1 \ucoal R_2 \ucoal \dots \ucoal R_n =
	\net{\link[\varnothing]AB^{\pi'}}AB
$
where the same links (but with potentially different witness maps) are coalesced. It follows by induction on this path (where the base case is $L_1$ and $R_1$) that for every corresponding pair of links $\link CD^\phi$ in $L_i$ and $\link[\tau]CD^\psi$ in $R_i$ we have $\tau\leq\sigma$ and $\psi\leq\phi$.
\end{proof}
% ..................................................

% ..................................................
\begin{lemma}
\label{lem:unifying implies strict coalescence}
If $\net{\lambda_\star}AB$ unifying-sequentializes to $\pi$ then there exists a witness assignment $\Sigma$ and substitution $\rho$ such that $\net{\lambda_\Sigma}AB$ strict-sequentializes to $\pi$ and $\lambda_\Sigma=\lambda_\star\rho$.
\end{lemma}

\begin{proof}
By induction on the sequentialization path $\net{\lambda_\star}AB\ucoal^*\net{\link[\varnothing]AB^\pi}AB$. For the end result, the statement holds with $\rho=\varnothing$. For the inductive step, consider a step $L\ucoal R$. We show the case ($\urr*$); the other cases are immediate.
%
\begin{itemize}
	\item
$\link C{D_1},\link[\tau]C{D_2}\ucoal\link[\sigma\join\tau]C{D_1{\*}D_2}$
\\
By the inductive hypothesis, $R\rho'$ strict-sequentializes to $\pi$. Let $\sigma\join\tau=\sigma\rho''=\tau\rho''$ and let $\rho=\rho''\rho'$. Then $L\rho$ strict-sequentializes to $\pi$ by 
\[
	\link[\sigma\rho]C{D_1},\link[\tau\rho]C{D_2}\scoal\link[(\sigma\join\tau)\rho']C{D_1{\*}D_2}~.
\qedhere
\]
\end{itemize}
\end{proof}
% ..................................................

We can then show that sequentialization and de-sequentialization for unification nets are inverses up to generality, and that composition is sound.

% ..................................................
\begin{theorem}
If $[\prf\pi AB]$ is $\net{\lambda_\Sigma}AB$ then $\net\lambda AB$ unifying-sequentializes to $\pi'\leq \pi$.
\end{theorem}

\begin{proof}
By Theorem~\ref{thm:proof->net->proof}, $\net{\lambda_\Sigma}AB$ sequentializes to $\pi$ in $(\scoal)$, and hence also in $(\ucoal)$. Then by Lemma~\ref{lem:ucoal generality} $\net{\lambda_\star}AB$ sequentializes to $\pi'\leq\pi$.
\end{proof}
% ..................................................


% ..................................................
\begin{theorem}
If $\net\lambda AB$ sequentializes to $\pi$, then $[\pi]=\net{\lambda_\Sigma} AB$ for some $\Sigma$.
\end{theorem}

\begin{proof}
By Lemma~\ref{lem:unifying implies strict coalescence}, since $\net\lambda AB$ sequentializes to $\pi$ there is a net $\net{\lambda_\Sigma} AB$ that sequentializes to $\pi$. By Theorem~\ref{thm:net->proof->net}, $[\pi]=\net{\lambda_\Sigma} AB$.
\end{proof}
% ..................................................


% ..................................................
\begin{theorem}
If $\net\lambda AB$ sequentializes to $\pi$ and $\net\kappa{\dual B}C$ to $\phi$ then their composition $\net{\lambda\comp\kappa}AC$ sequentializes to a proof $\psi'\leq\psi$ where $\psi$ is a normal form of $\pi\comp\phi$.
%\[
%	\psi~=~
%	\vc{\infer[\Qrr.]{\seq AC}{
%	 \Prf\pi AB & \Prf\phi{\dual B}C
%	}}~.
%\]
\end{theorem}

\begin{proof}
By Lemma~\ref{lem:unifying implies strict coalescence} there are witness labellings $\Sigma$ and $\Theta$ such that $\net{\lambda_\Sigma}AB$ strict-sequentializes to $\pi$ and $\net{\kappa_\Theta}{\dual B}C$ to $\phi$. By Theorem~\ref{thm:composition} their composition $(\net{\lambda_\Sigma\comp\kappa_\Theta)}AC$ strict-sequentializes to a normal form $\psi$ of $\pi\comp\phi$. By Lemma~\ref{lem:ucoal generality} the net $\net{(\lambda\comp\kappa)_\star}AC$ unifying-sequentializes to $\psi'\leq\psi$.
\end{proof}



\bibliography{ALL1}

% ==================================================

\newpage
\appendix

\section{Postponed proofs}

% ..................................................

\setcounter{theorem}{\thelemcorrect}
\begin{lemma}[Restatement]
Strict coalescence preserves and reflects correctness.
\end{lemma}

\begin{proof}
For a strict coalescence step $L\scoal R$, we will show that the witness pre-net $L$ is correct if and only if $R$ is. Let $L=\net{\lambda_\Sigma}AB$ and $R=\net{\kappa_\Theta}AB$. In each case, exact coverage is immediately preserved and reflected. For slice-correctness, we will demonstrate that the left-hand side and right-hand side of each rule belong to the same slice of $\seq AB$, or in the case of $\srr?$, naturally corresponding slices. For dependency-correctness, we will briefly show how acyclicity of the columns of the involved links is preserved.

\begin{itemize}
	\item
$\link C{D_i}\scoal\link C{D_1{\+}D_2}$
\\
A slice $S_B$ of $B$ and $\varnothing$ containing one of $(D_1,\tau)$, $(D_2,\tau)$, and $(D_1\+D_2,\tau)$ must also contain the other two. A slice $S$ of $\seq AB$ then contains all three of $\link C{D_1}$, $\link C{D_2}$, and $\link C{D_1{\+}D_2}$, or none. It follows that $S\cap \lambda_\Sigma$ is a singleton if and only if $S\cap\kappa_\Theta$ is. Since other slices are unaffected, $L$ is slice-correct if and only if $R$ is.

For dependency-correctness, the column of $\link C{D_i}$ is that of $\link C{D_1{\+}D_2}$ plus the pair $(D_i,\sigma\res{\ex[B]{D_i}})$ itself, which is minimal in the order $\dep$.

	\item
$\link C{D_1},\link C{D_2}\scoal\link C{D_1{\*}D_2}$
\\
A slice $S$ of $\seq AB$ contains $\link C{D_1{\*}D_2}$ if and only if it contains either of $\link C{D_1}$ or $\link C{D_2}$, and cannot contain both. Then $S\cap \lambda_\Sigma$ is a singleton if and only if $S\cap\kappa_\Theta$ is.

Dependency-correctness is immediate, as above.

	\item
$\link CD\scoal\link[\sigma\minus x]C{\exists x.D}$
\\
The witness sets $\Sigma(x)$ and $\Theta(x)$ for $L$ and $R$ need not be the same, since $L$ has $\sigma$ where $R$ has $\sigma\minus x$. Let $\Sigma(x)=\{t,t_1,\dots,t_n\}$ and $\Theta(x)=\{t_1,\dots,t_n\}$, where $\sigma(x)=t$. For every slice $S_B$ of $B$ over $\Sigma$ there is a corresponding slice $S'_B$ over $\Theta$, both (or neither) containing $(\exists x.D,\tau)$ and $(D,\tau[t_i/x])$ for $i\leq n$, but the former in addition having $(D,\tau[t/x])$. Then $S\cap \lambda_\Sigma$ is the singleton $\{\link CD\}$ if and only if $S\cap\kappa_\Theta$ is $\{\link[\sigma\minus x]C{\exists x.D}\}$, where it should be observed that if $t$ is also a witness to $x$ in some other slice of $R$, then $t\in\Theta(x)$ and in fact $S=S'$. 

%\lutz{?}
%\todo{Explain the witness sets better; refer to 1st case above; }

For dependency-correctness, the column of $\link CD$ is that of $\link[\sigma\minus x]C{\exists x.D}$ plus a pair $(D,\tau)$, which is minimal in $(\dep)$.

	\item
$\link CD\scoal\link C{\forall x.D}$
\\
A slice $S$ of $\seq AB$ contains $\link CD$ if and only if it contains also $\link C{\forall x.D}$, and hence $S\cap\lambda_\Sigma$ is a singleton if and only if $S\cap\kappa_\Theta$ is.

For dependency-correctness, the column of $\link CD$ is that of $\link C{\forall x.D}$ plus a pair $(D,\tau)$. The side-condition of the coalescence step is that $x\notin\sigma$; then $x$ does not occur free in any $(X,\rho)$, and $(D,\tau)$ is minimal in $(\dep)$.
\qedhere
\end{itemize}
\end{proof}

% ..................................................

\setcounter{theorem}{\thelemcoalescenceapplies}
\begin{lemma}[Restatement]
To a correct witness pre-net $\net{\lambda_\Sigma}AB$ a coalescence step applies, unless it is fully coalesced already, $\lambda_\Sigma=\{\link[\varnothing]AB\}$.
\end{lemma}

\begin{proof}
Let the \defn{depth} of a link $\link CD$ be a pair of integers $(n,m)$, where $n$ is the distance from $C$ to the root of $A$, and $m$ that from $D$ to $B$. We order link depth in the product order: $(i,j)\leq(n,m)$ if and only if $i\leq n$ and $j\leq m$. We will demonstrate that a link at maximal depth may always be coalesced, unless it is the unique link $\link[\varnothing]AB$ at $(0,0)$. 

To see that a maximally deep link coalesces, first note that a link $\link C{D_i}$ where $D_i$ occurs in $D_0{\+}D_1$ may always coalesce, as may a link $\link CD$ where $D$ occurs in $\exists x.D$. This leaves the following cases:
\begin{itemize}
	\item 
$\link A{D_i}$ with $D_i$ occurring in $D=D_1\*D_2$.
\\ 
Without loss of generality, let $i=1$. A slice $S_1$ of $\seq AB$ containing $\link A{D_1}$ has a counterpart $S_2$ containing $\link A{D_2}$. The depth of $\link A{D_2}$ is the same as that of $\link A{D_1}$. By correctness $S_2\cap \lambda_\Sigma$ is a singleton; by the assumption of maximality it may not contain a deeper link than $\link A{D_2}$; and it may not contain a shallower one since that would be shared with $S_1\cap\lambda_\Sigma$. Then $\net{\lambda_\Sigma}AB$ contains both $\link A{D_1}$ and $\link A{D_2}$, and these contract to $\link AD$.
	
	\item 
$\link AD$ with $D$ in $\forall x.D$. 
\\
The step $\link AD\scoal\link A{\forall x.D}$ applies if $x\notin\sigma$. By way of contradiction, assume $x\in\sigma$. The column of $\link AD$ contains $(D,\sigma_D)$ and $(\forall x.D,\tau)$ where $\tau=\sigma\res{\ex[B]{\forall x.D}}$. By the exact coverage condition, $\sigma=\sigma_A\cup\sigma_D$, and since the existential variables in $D$ and $\forall x.D$ are the same, $\ex[B]D=\ex[B]{\forall x.D}$, so that $\tau=\sigma_D$. (Note that since $\sigma_A=\varnothing$, we get $\sigma=\sigma_D=\tau$, but this is not essential to the argument.) Since $x\in\sigma$ we have $x\in\tau$, and in the column of $\link AD$ we have $(\forall x.D,\tau)\dep(D,\tau)$ since $D$ occurs as $\forall x.D$. But we already have $(D,\tau)\dep(\forall x.D,\tau)$ because $D\sub\forall x.D$, contradicting antisymmetry of $(\dep)$. Then $x\notin\sigma$, and the step $\link AD\scoal\link A{\forall x.D}$ applies.

	\item
$\link{C_i}{D_j}$ in $C=C_1\*C_2$ and $D=D_1\*D_2$.
\\
Without loss of generality, let $i=j=1$. By minimal depth and using similar reasoning to the first case above, the pre-net must contain one of the following three configurations.\\
\begin{tabular}[c]{ll}
	1. & $\link{C_1}{D_1},\link{C_1}{D_2},\link{C_2}{D_1},\link{C_2}{D_2}$
\\	2. & $\link{C_1}{D_1},\link{C_1}{D_2},\link{C_2}{D}$
\\	3. & $\link{C_1}{D_1},\link{C_2}{D_1},\link{C}{D_2}$
\end{tabular}
\hfill
$
\scriptstyle
1:~\vc{
\begin{tikzpicture}[net]
	\formula[y=1]{C*{_1}{{\*}}C*{_2}}
	\formula[y=0]{D*{_1}{{\*}}D*{_2}}
	\Vlink[red]{[-2]1,[-2]1}\Vlink[red]{[2]1,[-2]3}\Vlink[red]{[-2]3,[2]1}\Vlink[red]{[2]3,[2]3}
\end{tikzpicture}}
\qquad
2:~\vc{
\begin{tikzpicture}[net]
	\formula[y=1]{C*{_1}{{\*}}C*{_2}}
	\formula[y=0]{D*{_1}{{\*}}D*{_2}}
	\Vlink[red]{[-2]1,[-2]1}\Vlink[red]{[2]1,[-2]3}\Vlink[red]{3,2}
\end{tikzpicture}}
\qquad
3:~\vc{
\begin{tikzpicture}[net]
	\formula[y=1]{C*{_1}{{\*}}C*{_2}}
	\formula[y=0]{D*{_1}{{\*}}D*{_2}}
	\Vlink[red]{[-2]1,[-2]1}\Vlink[red]{[-2]3,[2]1}\Vlink[red]{2,3}
\end{tikzpicture}}
\quad
$
In the second case, the step $\link{C_1}{D_1},\link{C_1}{D_2}\scoal\link{C_1}D$ applies; in the third case, $\link{C_1}{D_1},\link{C_2}{D_1}\scoal\link C{D_1}$; and in the first case, both.

	\item 
$\link{C_i}D$ in $C=C_1\*C_2$ and $\forall x.D$.
\\
Without loss of generality let $i=1$. If $x\notin\sigma$ the rewrite step $\link{C_1}D\scoal\link{C_1}{\forall x.D}$ applies. Otherwise, let $x\in\sigma$. The slice $S_1$ of $\seq AB$ containing $\link{C_1}D$ has a counterpart $S_2$ containing $\link{C_2}D$, which must include exactly one link of $\lambda_\Sigma$. By the assumption of minimal depth, it cannot have greater depth than $\link{C_2}D$. It cannot be $\link CD$ or any shallower link, since that would be shared with the slice $S_1$ which already contains $\link{C_1}D$. It cannot be $\link{C_2}{\forall x.D}$ or any shallower link $\link[\tau]{C_2}{X}$ (i.e.\ with $\forall x.D\leq X$) because $x\in\sigma$. This would mean either $x\in\tau$ which contradicts the \emph{eigenvariables not free} convention, or $x\in\fv(\sigma(y))$ where $\forall x.D<\exists y.Y\leq X$ which creates a cyclic column, as in the second case above. It follows that $S_2\cap\lambda_\Sigma=\{\link{C_2}{D}\}$, so that the rewrite step $\link{C_1}D,\link{C_2}D\scoal\link CD$ applies.

	\item 
$\link CD$ in $\forall x.C$ and $\forall y.D$.
\\
A rewrite step $\link CD\scoal\link{\forall x.C}D$ or $\link CD\scoal\link C{\forall y.D}$ applies unless $x,y\in\sigma$. But that would generate a cycle in the column of $\link CD$, in one of three ways. If $x\in\sigma_C$ or $y\in\sigma_D$ then, since $\sigma_C=\sigma_{\forall x.C}$ and $\sigma_D=\sigma_{\forall y.D}$, respectively:
\[
	(C,\sigma_C)\dep(\forall x.C,\sigma_C)\dep(C,\sigma_C)
\qquad\quad
	(D,\sigma_D)\dep(\forall y.D,\sigma_D)\dep(D,\sigma_D)~.
\]
Otherwise, if $x\in\sigma_D$ and $y\in\sigma_C$ then
\[
	(C,\sigma_C)\dep(\forall x.C,\sigma_C)\dep(D,\sigma_D)\dep(\forall x.D,\sigma_D)\dep(C,\sigma_C)~.
\qedhere
\]
\end{itemize}
\end{proof}

% ..................................................

\setcounter{theorem}{\thethmcomposition}
\begin{theorem}[Restatement]
If proof nets $\net{\lambda_\Sigma}AB$ and $\net{\kappa_\Theta}{\dual B}C$ sequentialize to $\pi$ and $\phi$ respectively, then their composition $(\net{\lambda_\Sigma}AB)\comp(\net{\kappa_\Theta}{\dual B}C)$ sequentializes to a normal form $\psi$ of $\pi\comp\phi$.
\end{theorem}

\begin{proof}
By Corollary~\ref{cor:strong coalescence} the proof nets $L=\net{\lambda_\Sigma}AB$ and $R=\net{\kappa_\Theta}{\dual B}C$ strongly coalesce. We may then interleave their coalescence sequences as follows: if a synchronized step in $L$ and $R$ on the interface $B$ and $\dual B$ is available, apply it; otherwise perform steps in $L$ on $A$ and in $R$ on $C$ until it is. This gives the following combined sequence.
\[
\begin{array}{c@{~}c@{~}c@{~}l@{~}c@{~}l@{~}c@{~}l@{~}c}
	L &{}={}& L_1 & ~\scoal^?~ & L_2 & ~\scoal^?~ & \dots & ~\scoal^?~ & L_n 
\\
	R &{}={}& R_1 & ~\scoal^?~ & R_2 & ~\scoal^?~ & \dots & ~\scoal^?~ & R_n 
\\[5pt]
	\Downarrow && \Downarrow && \Downarrow &&&& \Downarrow 
\\[5pt]
	L\comp R & = & L_1\comp R_1 & ~\scoal~ & L_2\comp R_2 & ~\scoal~ & \dots & ~\scoal~ & L_n\comp R_n
\end{array}
\]
(Here, $(\scoal^?)$ is the relation $(\scoal)\cup(=)$, but we assume that at least $L_i\scoal L_{i+1}$ or $R_i\scoal R_{i+1}$.)
%
The path along the top and right of this diagram sequentializes $L$ to $\pi'$ and $R$ to $\phi'$ (equivalent to $\pi$ and $\phi$ respectively), and then composes to $L_n\comp R_n = \net{\{\link[\varnothing]AC^{\pi'\comp\phi'}\}}AC$. %with
%\[
%	\psi' = \vc{\infer[\!\!\scriptstyle{\mathrm{cut}}]{\seq AC}{\Prf{\pi'}AB & \Prf{\phi'}{\dual B}C}}~.
%\]

Each square of the diagram converges as one of the critical pairs of sequentialization and composition discussed above. Then each path along the diagram from top left ($L$ and $R$) to bottom right $(L_n\comp R_n)$ gives a sequentialization, with cuts, of $L_n\comp R_n$. Let the path taking the vertical step from $L_i$ and $R_i$ to $L_i\comp R_i$ sequentialize to $\psi_i$, so that $\psi_n=\psi'$. By the way each square converges, we have that $\psi_i$ is reached from $\psi_{i+1}$ by a cut-elimination or permutation step. 

Finally, in $L$ and $R$ every link is an axiom link. Any link in $L\comp R$ is composed from two links $\link ab$ in $L$ and $\link[\tau]{\dual b}c$ in $R$, which yields $\link[\rho]ac$ where $\rho = \sigma_a\tau_c\dcom\fix{\sigma_b\tau\subdual b}$. This sequentializes to the axiom $\infer{\seq{a\rho}{c\rho}}{}$, which is in normal form. Then $L\comp R$ is a proof net (it has an axiom linking and it coalesces), and it sequentializes to a normal form of $\psi$.
%
%which sequentialize respectively to. 
%\[
%	\infer{\seq{a\sigma}{b\sigma}}{} 
%	\qquad\text{and}\qquad
%	\infer{\seq{\dual b\tau}{c\tau}}{}~.
%\]
%We have the link 
\end{proof}

% --------------------------------------------------

\section{Permutations}

% ..................................................

\begin{figure}[!ht]
\[
\renewcommand\+{{+}}
\renewcommand\*{{\times}}
\scalebox{0.85}{$
\begin{array}{c@{\qquad}c}
	\permAC   & \permEC   
\\	\downperm & \downperm \\
	\permCA   & \permCE
\\ \\[20pt]
	\permPC   & \permXC 
\\	\downperm & \downperm \\
	\permCP   & \permCX
\end{array}
\begin{array}{c}
	\permCD \\ \downperm \\ \permDC
\end{array}
$}
\]
\caption{Cut-permutations}
\label{fig:cut-permutations}
\end{figure}

% ..................................................

\begin{figure}[!ht]
\[
\renewcommand\+{{+}}
\renewcommand\*{{\times}}
\scalebox{0.85}{$
\begin{array}{@{}c@{\qquad}c@{\qquad}c@{\qquad}c@{}}
	\permAB   & \permAE   & \permAP   & \permAX 
\\	\downperm & \downperm & \downperm & \downperm \\
	\permBA   & \permEA   & \permPA   & \permXA
\\ \\[20pt]
 			  & \permEF   & \permEP   & \permEX
\\			  & \downperm & \downperm & \downperm \\
 			  & \permFE   & \permPE   & \permXE
\\ \\[20pt]
 			  &			  & \permPQ   & \permPX
\\			  &			  & \downperm & \downperm \\
 			  &			  & \permQP   & \permXP
\\ \\[20pt]
 			  &			  &			  & \permPX
\\			  &			  &			  & \downperm \\
 			  &			  &			  & \permXP
\end{array}$}
\]
\caption{Cut-free rule permutations}
\label{fig:permutations}
\end{figure}


\end{document}






\begin{lemma}
If a labelled pre-net $\net{\lambda_\Sigma^\Pi}AB$ strict-sequentializes to $\pi$, then $(\net{\lambda_\Sigma^\Pi}AB)\rho$ strict-sequentializes to $\pi\rho$.
\end{lemma}

\begin{proof}
By induction on the sequentialization sequence.
\end{proof}


NORMALIZATION PROBLEM

\begin{lemma}
For $\link ab$ with $\sigma=\mgu(a,\dual b)$ in a proof net $\net{\lambda_\star}AB$ and $\link[\tau]{\dual b}c$ with $\tau=\mgu(b,c)$ in $\net{\kappa_\star}{\dual B}C$ the composition $\sigma_a\tau_c\dcom\fix{\sigma_b\tau\subdual b}$ is $\mgu(a,\dual c)$.
\end{lemma}


%
%\begin{lemma}
%For $\link ab$ in a proof net $\net{\lambda_\star}AB$ and $\link[\tau]{\dual b}c$ in $\net{\kappa_\star}{\dual B}C$ the composition $\sigma_a\tau_c\dcom\fix{\sigma_b\tau\subdual b}=\mgu(a,c)$.
%\end{lemma}
%
%\begin{proof}
%Let $\rho=\mgu(a,c)$ and $\rho'=\sigma_a\tau_c\dcom\fix{\sigma_b\tau\subdual b}$. Define the \defn{basis} $\floor{s,t}$ of the $\mgu$ of $s$ and $t$ to be the substitution map defined by $\floor{x,t}=\floor{t,x} = [t/x]$ and
%\[
%	\floor{P(s_1,\dots,s_n),P(t_1,\dots,t_n)} =
%	\floor{f(s_1,\dots,s_n),f(t_1,\dots,t_n)} =
%	\floor{s_1,t_1}\dots\floor{s_n,t_n}~.
%\]
%Write $\floor\sigma$ for the basis $\floor{a,b}$ of $\sigma$, $\floor\tau$ for that of $\tau$, and $\floor\rho$ for that of $\rho$.
%\end{proof}
%
%\end{document}
%



\begin{figure}
\hrule
\par\bigskip
\[
\begin{array}{ccc}
    \vc{\begin{tikzpicture}[net]
    	\formula[y=2]{P}
    	\formula[y=1]{Q}
    	\Vlink[blue]{1,1}
    \end{tikzpicture}}
&\rightsquigarrow&    
    \vc{\begin{tikzpicture}[net]
    	\formula[y=2]{P}
    	\formula[y=1]{Q}
    	\Vlink[red,label={$~\scriptstyle{\sigma(P,\dual Q)}$}]{1,1}
    \end{tikzpicture}}
\\ \\
    \vc{\begin{tikzpicture}[net]
    	\formula[y=2]{A}
    	\formula[y=1]{B\+C}
    	\Vlink[red,label={$\scriptstyle\sigma~$},l]{1,1}
    \end{tikzpicture}}
&\rightsquigarrow& 
    \vc{\begin{tikzpicture}[net]
    	\formula[y=2]{A}
    	\formula[y=1]{B\+C}
    	\Vlink[red,label={~$\scriptstyle\sigma$}]{1,2}
    \end{tikzpicture}}
\\ \\   
    \vc{\begin{tikzpicture}[net]
    	\formula[y=2]{A}
    	\formula[y=1]{B\*C}
    	\Vlink[red,label={$\scriptstyle\sigma~$},l]{[-1]1,1}
    	\Vlink[red,label={~$\scriptstyle\tau$}]{[1]1,3}
    \end{tikzpicture}}
&\stackrel{\sigma\coh\tau}\rightsquigarrow&   
	\vc{\begin{tikzpicture}[net]
    	\formula[y=2]{A}
    	\formula[y=1]{B\*C}
    	\Vlink[red,label={~$\scriptstyle{\sigma\cup\tau}$}]{1,2}
    \end{tikzpicture}}
    
\\ \\
    \vc{\begin{tikzpicture}[net]
    	\formula[y=2]{A}
    	\formula[y=1]{\exists x.B}
    	\Vlink[red,label={~$\scriptstyle\sigma$}]{1,4}
    \end{tikzpicture}}
&\rightsquigarrow&
    \vc{\begin{tikzpicture}[net]
    	\formula[y=2]{A}
    	\formula[y=1]{\exists x.B}
    	\Vlink[red,label={~$\scriptstyle{\sigma\minus x}$}]{1,1}
    \end{tikzpicture}}
\\ \\
    \vc{\begin{tikzpicture}[net]
    	\formula[y=2]{A}
    	\formula[y=1]{\forall a.B}
    	\Vlink[red,label={~$\scriptstyle\sigma$}]{1,4}
    \end{tikzpicture}}
&\stackrel{a\,\notin\,\fv(A\sigma)}\rightsquigarrow&
    \vc{\begin{tikzpicture}[net]
    	\formula[y=2]{A}
    	\formula[y=1]{\forall a.B}
    	\Vlink[red,label={~$\scriptstyle\sigma$}]{1,1}
    \end{tikzpicture}}
\end{array}
\]
\caption{Coalescence rules}
\label{fig:coalescence}
\end{figure}




\begin{equation}
  	\net{\{\link A{B_i}^\pi\}}A{B_1\+B_2} 
 ~\scoal~
	\net{\{\link A{B_1\+B_2} ^\psi\}}A{B_1\+B_2}
\tag{$\srr+i$}
\end{equation}

\hfill $\text{where}\quad \psi=~
  \vc{\infer  {\seq{A\sigma}{B_1\sigma\,\+\,B_2\sigma}}  {\Prf \pi {A\sigma} {B_i\sigma}}}
$ \hspace*{20pt}

\begin{equation}
	\net{\{\link AB^\pi,\link AC^\phi\}}A{B\*C}
 ~\scoal~ 
	\net{\{\link A{B\*C}^\psi\}}A{B\*C}
\tag{$\srr*$}
\end{equation}

\hfill $\text{where}\quad \psi=~
  \vc{
   \infer  {\seq{A\sigma}{B\sigma\,\*\,C\sigma}}  {
    \Prf \pi {A\sigma} {B\sigma}
    &
    \Prf \phi {A\sigma} {C\sigma}    
  }}$ \hspace*{20pt}

\begin{equation}
	\net{\{\link AB^\pi\}}A{\exists x.B}
 ~\scoal~ 
	\net{\{\link[\sigma\minus x]A{\exists x.B}^\psi\}}A{\exists x.B}
\tag{$\srr?$}
\end{equation}

\hfill $\text{if}\quad\sigma(x)\,\neq\,\bot~,
		\quad
		\text{where}\quad \tau=\sigma\minus x
		\quad
		\text{and}\quad \psi=~
  \vc{
   \infer  {\seq{A\tau}{\exists x.B\tau}}  {
    \Prf \pi {A\sigma} {B\sigma}
  }}
$ \hspace*{20pt}

\begin{equation}
	\net{\{\link AB\}}A{\forall a.B}
 ~\scoal~
	\net{\{\link A{\forall a.B}\}}A{\forall a.B}
\tag{$\srr!$}
\end{equation}

\hfill $\text{if}\quad a\notin\sigma~,
		\quad
		\text{where}\quad \psi=~
  \vc{
   \infer  {\seq{A\sigma}{\forall a.B\sigma}}  {
    \Prf \pi {A\sigma} {B\sigma}
  }}
$ \hspace*{20pt}



\begin{proposition}
If $\net{\lambda_\Sigma}AB$ and $\net{\kappa_\Theta}{\dual B}C$ both coalesce, so does their composition.
\end{proposition}

\begin{proof}
By induction on both coalescence sequences. In the base case, $\lambda_\Sigma=\{\link[\varnothing]AB\}$ and $\kappa_\Theta=\{\link[\varnothing]{\dual B}C\}$, and the composition is $\net{\{\link[\varnothing]AC\}}AC$.

If
\todo{}
\end{proof}


\[
\begin{array}{@{}ccccc@{}}
  \vc{
  \begin{tikzpicture}[net]
	\formula[y=1.2]{\forall*{x.}\exists*{y.~}{\dual P}*{(y)}\+{\dual Q}*{(x,y)}}
	\formula[y=0]{\exists*{z.~}P*{(f(z))}\*Q*{(z,t)}}
	\Vlink[red,l,label={$\scriptstyle[f(x)/y\,,\,x/z]$~~}]{3,2}
	\Vlink[blue,r,label={~$\scriptstyle[t/y\,,\,x/z]$}]{5,4}
  \end{tikzpicture}}
& \scoal &
  \vc{
  \begin{tikzpicture}[net]
	\formula[y=1.2]{\forall*{x.}\exists*{y.~}{\dual P}*{(y)}\+{\dual Q}*{(x,y)}}
	\formula[y=0]{\exists*{z.~}P*{(f(z))}\*Q*{(z,t)}}
	\Vlink[red,l,label={$\scriptstyle[f(x)/y\,,\,x/z]$~~~}]{4,2}
	\Vlink[blue,r,label={~$\scriptstyle[t/y\,,\,x/z]$}]{5,4}
  \end{tikzpicture}}
& \scoal &
  \vc{
  \begin{tikzpicture}[net]
	\formula[y=1.2]{\forall*{x.}\exists*{y.~}{\dual P}*{(y)}\+{\dual Q}*{(x,y)}}
	\formula[y=0]{\exists*{z.~}P*{(f(z))}\*Q*{(z,t)}}
	\Vlink[red,l,label={$\scriptstyle[x/z]$~~}]{2,2}
	\Vlink[blue,r,label={~$\scriptstyle[t/y\,,\,x/z]$}]{5,4}
  \end{tikzpicture}}
\\ \\[-10pt] &&&& \downarrow \\ \\[-10pt]
  \vc{
  \begin{tikzpicture}[net]
	\formula[y=1.2]{\forall*{x.}\exists*{y.~}{\dual P}*{(y)}\+{\dual Q}*{(x,y)}}
	\formula[y=0]{\exists*{z.~}P*{(f(z))}\*Q*{(z,t)}}
	\Vlink[violet,l,label={$\scriptstyle[x/z]$~~~}]{[2]2,3}
  \end{tikzpicture}}
& \leftarrow &
  \vc{
  \begin{tikzpicture}[net]
	\formula[y=1.2]{\forall*{x.}\exists*{y.~}{\dual P}*{(y)}\+{\dual Q}*{(x,y)}}
	\formula[y=0]{\exists*{z.~}P*{(f(z))}\*Q*{(z,t)}}
	\Vlink[red,l,label={$\scriptstyle[x/z]$~~}]{2,2}
	\Vlink[blue,r,label={~~~$\scriptstyle[x/z]$}]{[4]2,4}
  \end{tikzpicture}}
& \leftarrow &
  \vc{
  \begin{tikzpicture}[net]
	\formula[y=1.2]{\forall*{x.}\exists*{y.~}{\dual P}*{(y)}\+{\dual Q}*{(x,y)}}
	\formula[y=0]{\exists*{z.~}P*{(f(z))}\*Q*{(z,t)}}
	\Vlink[red,l,label={$\scriptstyle[x/z]$~~}]{2,2}
	\Vlink[blue,r,label={~$\scriptstyle[t/y\,,\,x/z]$}]{4,4}
  \end{tikzpicture}}
\\ \\[-10pt] \downarrow &&&& \\ \\[-10pt]
  \vc{
  \begin{tikzpicture}[net]
	\formula[y=1.2]{\forall*{x.}\exists*{y.~}{\dual P}*{(y)}\+{\dual Q}*{(x,y)}}
	\formula[y=0]{\exists*{z.~}P*{(f(z))}\*Q*{(z,t)}}
	\Vlink[violet,l,label={$\scriptstyle\varnothing$~~}]{2,1}
  \end{tikzpicture}}
& \scoal &
  \vc{
  \begin{tikzpicture}[net]
	\formula[y=1.2]{\forall*{x.}\exists*{y.~}{\dual P}*{(y)}\+{\dual Q}*{(x,y)}}
	\formula[y=0]{\exists*{z.~}P*{(f(z))}\*Q*{(z,t)}}
	\Vlink[violet,l,label={$\scriptstyle\varnothing$~~}]{1,1}
  \end{tikzpicture}}
\end{array}
\]

%\vc{\scalebox{0.5}{$
%	\infer[\Qrr!]{\Seq{\forall x.\exists y.\dual P(y)\+\dual Q(x,y)}{\exists z.P(f(z))\*Q(z,t)}}{
%	 \infer[\Qrr?,x]{\Seq{\exists y.\dual P(y)\+\dual Q(x,y)}{\exists z.P(f(z))\*Q(z,t)}}{
%	  \infer[\Qrr*]{\Seq{\exists y.\dual P(y)\+\dual Q(x,y)}{P(f(x))\*Q(x,t)}}{
%	   \infer[\Qrr?{f(x)}]{\Seq{\exists y.\dual P(y)\+\dual Q(x,y)}{P(f(x))}}{
%	    \infer[\Qrr+1]{\Seq{\dual P(f(x))\+\dual Q(x,f(x))}{P(f(x))}}{
%	     \infer[\Qrr1]{\Seq{\dual P(f(x))}{P(f(x))}}{}
%	   }}&
%	   \infer[\Qrr?t]{\Seq{\exists y.\dual P(y)\+\dual Q(x,y)}{Q(x,t)}}{
%	    \infer[\Qrr+2]{\Seq{\dual P(t)\+\dual Q(x,t)}{Q(x,t)}}{
%	     \infer[\Qrr1]{\Seq{\dual Q(x,t)}{Q(x,t)}}{}
%	}}}}}

%
%\[
%\begin{array}{@{}ccc@{}}
%  \vc{
%  \begin{tikzpicture}[net]
%	\formula[y=1.2]{\forall*{x.}\exists*{y.~}{\dual P}*{(y)}\+{\dual Q}*{(x,y)}}
%	\formula[y=0]{\exists*{z.~}P*{(f(z))}\*Q*{(z,t)}}
%	\Vlink[red,l,label={$\scriptstyle[f(x)/y\,,\,x/z]$~~}]{3,2}
%	\Vlink[blue,r,label={~$\scriptstyle[t/y\,,\,x/z]$}]{5,4}
%  \end{tikzpicture}}
%& \stackrel{\srr+1}\scoal &
%  \vc{
%  \begin{tikzpicture}[net]
%	\formula[y=1.2]{\forall*{x.}\exists*{y.~}{\dual P}*{(y)}\+{\dual Q}*{(x,y)}}
%	\formula[y=0]{\exists*{z.~}P*{(f(z))}\*Q*{(z,t)}}
%	\Vlink[red,l,label={$\scriptstyle[f(x)/y\,,\,x/z]$~~~}]{4,2}
%	\Vlink[blue,r,label={~$\scriptstyle[t/y\,,\,x/z]$}]{5,4}
%  \end{tikzpicture}}
%\\ \\[-5pt] && \downarrow\scriptstyle\srr?,f(x) \\ \\[-5pt]
%  \vc{
%  \begin{tikzpicture}[net]
%	\formula[y=1.2]{\forall*{x.}\exists*{y.~}{\dual P}*{(y)}\+{\dual Q}*{(x,y)}}
%	\formula[y=0]{\exists*{z.~}P*{(f(z))}\*Q*{(z,t)}}
%	\Vlink[red,l,label={$\scriptstyle[x/z]$~~}]{2,2}
%	\Vlink[blue,r,label={~$\scriptstyle[t/y\,,\,x/z]$}]{4,4}
%  \end{tikzpicture}}
%& \stackrel{\srr+2}\leftarrow &
%  \vc{
%  \begin{tikzpicture}[net]
%	\formula[y=1.2]{\forall*{x.}\exists*{y.~}{\dual P}*{(y)}\+{\dual Q}*{(x,y)}}
%	\formula[y=0]{\exists*{z.~}P*{(f(z))}\*Q*{(z,t)}}
%	\Vlink[red,l,label={$\scriptstyle[x/z]$~~}]{2,2}
%	\Vlink[blue,r,label={~$\scriptstyle[t/y\,,\,x/z]$}]{5,4}
%  \end{tikzpicture}}
%\\ \\[-5pt] \downarrow\scriptstyle\srr?,t \\ \\[-5pt]  
%  \vc{
%  \begin{tikzpicture}[net]
%	\formula[y=1.2]{\forall*{x.}\exists*{y.~}{\dual P}*{(y)}\+{\dual Q}*{(x,y)}}
%	\formula[y=0]{\exists*{z.~}P*{(f(z))}\*Q*{(z,t)}}
%	\Vlink[red,l,label={$\scriptstyle[x/z]$~~}]{2,2}
%	\Vlink[blue,r,label={~~~$\scriptstyle[x/z]$}]{[4]2,4}
%  \end{tikzpicture}}
%& \stackrel{\srr*}\scoal & 
%  \vc{
%  \begin{tikzpicture}[net]
%	\formula[y=1.2]{\forall*{x.}\exists*{y.~}{\dual P}*{(y)}\+{\dual Q}*{(x,y)}}
%	\formula[y=0]{\exists*{z.~}P*{(f(z))}\*Q*{(z,t)}}
%	\Vlink[violet,l,label={$\scriptstyle[x/z]$~~~}]{[2]2,3}
%  \end{tikzpicture}}
%\\ \\[-5pt] && \downarrow\scriptstyle\srr?,x \\ \\[-5pt]
%  \vc{
%  \begin{tikzpicture}[net]
%	\formula[y=1.2]{\forall*{x.}\exists*{y.~}{\dual P}*{(y)}\+{\dual Q}*{(x,y)}}
%	\formula[y=0]{\exists*{z.~}P*{(f(z))}\*Q*{(z,t)}}
%	\Vlink[violet,l,label={$\scriptstyle\varnothing$~~}]{1,1}
%  \end{tikzpicture}}
%& \stackrel{\srr!}\leftarrow &
%  \vc{
%  \begin{tikzpicture}[net]
%	\formula[y=1.2]{\forall*{x.}\exists*{y.~}{\dual P}*{(y)}\+{\dual Q}*{(x,y)}}
%	\formula[y=0]{\exists*{z.~}P*{(f(z))}\*Q*{(z,t)}}
%	\Vlink[violet,l,label={$\scriptstyle\varnothing$~~}]{2,1}
%  \end{tikzpicture}}
%\end{array}
%\]


%\[
%	\vc{
%	 \infer[\Qrr*]{\seq{A\*B}{\exists x.C}}{
%	  \infer[\Qrr?y]{\seq A{\exists x.C}}{\seq AC}
%	  &
%	  \infer[\Qrr?z]{\seq B{\exists x.C}}{\seq BC}
%	}}
%	\quad\sim\quad
%	\vc{
%	 \infer[\Qrr?y]{\seq{A\*B}C}{
%	  \infer[\Qrr*]{\seq{A\*B}{\exists x.C}}{
%	   \seq AC
%	   &
%	   \seq BC
%	}}}
%\]
