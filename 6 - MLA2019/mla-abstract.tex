%% FIRST RENAME THIS FILE <yoursurname>.tex. 
%% BEFORE COMPLETING THIS TEMPLATE, SEE THE "READ ME" SECTION 
%% BELOW FOR INSTRUCTIONS. 
%% TO PROCESS THIS FILE YOU WILL NEED TO DOWNLOAD asl.cls from 
%% http://aslonline.org/abstractresources.html. 


\documentclass[a4paper,11pt]{article}

\pagestyle{plain}

\def\urladdr#1{\endgraf\noindent{\it URL Address}: {\tt #1}.}


\newcommand{\NP}{}
%\usepackage{verbatim}

\begin{document}
\thispagestyle{empty}

\title{Proof nets for first-order additive linear logic}

\author{Willem Heijltjes \and Dominic Hughes \and Lutz Stra{\ss}burger}

\maketitle
%% % First author's affiliation
%% \affil{Dept.\ Computer Science, University of Bath, UK}
%% %\meetemail{w.b.heijltjes@bath.ac.uk}
%% \urladdr{http://willem.heijltj.es}
%% %
%% % Second author's affiliation
%% \affil{Logic Group, U.C.\ Berkeley, USA}
%% \urladdr{http://boole.stanford.edu/\textasciitilde dominic}
%% %
%% % Third author's affiliation
%% \affil{Parsifal, Inria, France}
%% \urladdr{http://www.lix.polytechnique.fr/Labo/Lutz.Strassburger}

%% INSERT TEXT OF ABSTRACT DIRECTLY BELOW
In this talk we will present canonical proof nets for first-order additive linear logic, the fragment of linear logic with sum, product, and first-order universal and existential quantification. The central challenge is to combine the witnessing information to existential quantifiers with the behaviour of the additive conjunction. The latter creates multiple "slices", each containing a different version of the same quantifier, with a potentially different witness.

The challenge is met by upending the traditional evaluation of an existential quantifier by an immediate substitution: instead, the substitution is recorded separately, at each axiom link in the proof net. The result is a canonical notion of proof nets for this logic. The main thrust of the work resolves the technical consequences of the design. Efficient and intuitive correctness and sequentialization are given by "coalescence", an additive version of multiplicative contractibility; in essence, this is top-down sequentialization by simple graph rewriting. A main contribution is an intricate geometric correctness condition, which subtly combines "slicing" correctness for propositional additives with "dependency" correctness for quantification. Cut-elimination involves the composition of the witnessing substitutions in two proof nets by "composition + hiding" in the style of game semantics.

A further contribution is the observation --- following recent work by Dominic Hughes for first-order multiplicative linear logic --- that witnessing information can be omitted from first-order additive proof nets altogether, and reconstructed via unification. This yields a further, coarser notion of proof net that factors out any inessential choice in witness assignment.

Details can be found in the technical report \cite{HHS}.
%This is joint work with Dominic Hughes and Lutz Strassburger.

\begin{thebibliography}{10}

\bibitem{HHS}
Willem B.\ Heijltjes, Dominic J.\,D.\ Hughes, and Lutz Stra{\ss}burger.
\emph{Proof nets for first-order additive linear logic.}
Inria Research Report RR-9201,
2018


%% INSERT YOUR BIBLIOGRAPHIC ENTRIES HERE; 
%% SEE (4) BELOW FOR PROPER FORMAT.
%% EACH ENTRY MUST BEGIN WITH \bibitem{citation key}
%%
%% IF THERE ARE NO ENTRIES  
%% DELETE THE LINE ABOVE (\begin{thebibliography}{20}) 
%% AND THE LINE BELOW (\end{thebibliography})

\end{thebibliography}


%\vspace*{-0.5\baselineskip}
% this space adjustment is usually necessary after a bibliography

\end{document}


%% READ ME
%% READ ME
%% READ ME

INSTRUCTIONS FOR SUPPLYING INFORMATION IN THE CORRECT FORMAT: 

1. Author names are listed as First Last, First Last, and First Last.

\absauth{FirstName1 LastName1, FirstName2 LastName2, and FirstName3 LastName3}


2. Titles of abstracts have ONLY the first letter capitalized,
except for Proper Nouns.

\meettitle{Title of abstract with initial capital letter only, except for
Proper Nouns} 


3. Affiliations and email addresses for authors of abstracts are
  listed separately.

% First author's affiliation
\affil{Department, University, Street Address, Country}
\meetemail{First author's email}
%%% NOTE: email required for at least one author
\urladdr{OPTIONAL}
%
% Second author's affiliation
\affil{Department, University, Street Address, Country}
\meetemail{Second author's email}
\urladdr{OPTIONAL}
%
% Third author's affiliation
\affil{Department, University, Street Address, Country}
% Second author's email
\meetemail{Third author's email}
\urladdr{OPTIONAL}


4. Bibliographic Entries

%%%% IF references are submitted with abstract,
%%%% please use the following formats

%%% For a Journal article
\bibitem{cite1}
{\scshape Author's Name},
{\itshape Title of article},
{\bfseries\itshape Journal name spelled out, no abbreviations},
vol.~XX (XXXX), no.~X, pp.~XXX--XXX.

%%% For a Journal article by the same authors as above,
%%% i.e., authors in cite1 are the same for cite2
\bibitem{cite2}
\bysame
{\itshape Title of article},
{\bfseries\itshape Journal},
vol.~XX (XXXX), no.~X, pp.~XX--XXX.

%%% For a book
\bibitem{cite3}
{\scshape Author's Name},
{\bfseries\itshape Title of book},
Name of series,
Publisher,
Year.

%%% For an article in proceedings
\bibitem{cite4}
{\scshape Author's Name},
{\itshape Title of article},
{\bfseries\itshape Name of proceedings}
(Address of meeting),
(First Last and First2 Last2, editors),
vol.~X,
Publisher,
Year,
pp.~X--XX.

%%% For an article in a collection
\bibitem{cite5}
{\scshape Author's Name},
{\itshape Title of article},
{\bfseries\itshape Book title}
(First Last and First2 Last2, editors),
Publisher,
Publisher's address,
Year,
pp.~X--XX.

%%% An edited book
\bibitem{cite6}
Author's name, editor. % No special font used here
{\bfseries\itshape Title of book},
Publisher,
Publisher's address,
Year.

